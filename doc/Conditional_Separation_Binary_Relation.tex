% START
\documentclass[12pt]{article}
\def\citep#1{\cite{#1}}

\newif\ifpreprint
% select preprinttrue (by uncommenting) if you want to generate the
% preprint version with aditional materials
\preprintfalse
% select (by uncommenting) preprintfalse if you want to obtain a version without moral relation
% ready for submission.
%\preprintfalse

\usepackage{fullpage}

% START
% macros

\usepackage[utf8]{inputenc} % allow utf-8 input
\usepackage[T1]{fontenc}    % use 8-bit T1 fonts

\usepackage{amsfonts}       % blackboard math symbolsG
\usepackage{amssymb}
\usepackage{amsmath}
\usepackage{euscript}
\usepackage{stmaryrd} % used by \ic 
\usepackage{color}

\usepackage[colors]{optsys}
% we can pass options to optsys \PassOptionsToPackage{colors}{optsys}
% and let causal load optsys
\usepackage{causal}
\usepackage{snippets}

% margin bars 
\usepackage{tikz}
\usetikzlibrary{tikzmark}

\newcounter{coqbar}
\newcommand{\coqbarcolor}{red}
\newcommand{\coqstart}[1][red]{%
    \renewcommand{\coqbarcolor}{#1}%
    \stepcounter{coqbar}%
    \tikzmarknode{chbar-\thecoqbar-start}{\strut}%
}
\newcommand{\coqend}{%
  \tikzmarknode{chbar-\thecoqbar-end}{\strut}%
  \begin{tikzpicture}[remember picture, overlay]
    \draw[very thick, \coqbarcolor] ([xshift={\oddsidemargin+1in-10pt}]current page.west |- chbar-\thecoqbar-start.north) -- ([xshift={\oddsidemargin+1in-10pt}]current page.west |- chbar-\thecoqbar-end.south);
  \end{tikzpicture}%
}

\newcommand{\pinX}{\PY{n}{p}~\PY{n}{$[\in]$}~\PY{n}{X}}
\newcommand{\Liftp}{\PY{n}{Lift}~\PY{n}{p}}
\newcommand{\pSucR}{\PY{n}{p}~\PY{o}{$[\text{Suc}{\in}]$}~\PY{n}{R}}
\newcommand{\Liftpin}{\PY{n}{p}~\PY{o}{$[\text{L}{\in}]$}~\PY{n}{R}}

% START
% macros to remove in submissions
\usepackage{amsmath}

\newcommand{\nb}[3]{
		{\colorbox{#2}{\bfseries\sffamily\tiny\textcolor{white}{#1}}}
		{\textcolor{#2}{$\blacktriangleright${#3}$\blacktriangleleft$}}}

\newcommand{\mdl}[1]{\nb{Michel}{cyan}{#1}}
\newcommand{\jpc}[1]{\nb{JPC}{blue}{#1}}
\newcommand{\Benjamin}[1]{\nb{Benjamin}{green}{#1}}

% START
\title{Conditional Separation as a Binary Relation}% Between Vertices}

% START
\author{%
  Jean-Philippe Chancelier$^\dagger$,
  Michel De Lara\footnote{CERMICS, Ecole des Ponts, Marne-la-Vall\'ee, France},
  Benjamin Heymann\footnote{Criteo AI Lab, Paris, France}
}

\date{\today}

% START
\begin{document}

\maketitle

% START
\begin{abstract}
  Pearl's d-separation is a foundational notion to study conditional
  independence between random variables on directed acyclic graphs.
  Pearl defined the d-separation of two subsets conditionally on a third one.
  We show how the d-separation can be extended beyond acyclic graphs
  and can be expressed and characterized as a binary relation between vertices. 
\end{abstract}

% START
% \pagebreak 
\setcounter{tocdepth}{2}
\pagebreak\tableofcontents \pagebreak 

% START
\section{Introduction}
As the world shifts toward more and more data-driven decision-making, causal
inference is taking more space in applied sciences, statistics and machine
learning.  This is because it allows for better, more robust decision-making,
and provides a way to interpret the data that goes beyond correlation
\citep{pearl2018book}.  In his seminal work~\citep{pearl1995causal}, Pearl
builds on graphical models~\citep{cowell2006probabilistic} to introduce the
so-called do-calculus and the notion of d-separation on a directed acyclic graph
(DAG).

This paper was written in parallel to two other papers
\citep{De-Lara-Chancelier-Heymann-2021,Heymann-De-Lara-Chancelier-2021}, all of
which aim at providing another perspective on conditional independence and
do-calculus.  Here we show how the d-separation can be extended beyond DAGs and
can be expressed and characterized as a binary relation between vertices.  We
think this work will allow for new proof strategies in the developpement of the
theory of causal inference.  In particular, the main result is an intermediate
step that we reuse in the two other above mentionned papers.  Last, we think the
developpement of a theory based on binary relations makes it amenable to
computer aided proof~\citep{pous2013}. %

In Sect.~\ref{Conditional_active_relation_induced_by_endpoints_of_non-blocking_paths},
we present our extended definition of d-separation and then characterize it as
the complementary of the conditional active binary relation between vertices in
Sect.~\ref{Characterization__of_the_conditional_directional_separation_binary_relation},
we state and sketch the proof of our main result, the characterization of the
d-separation relation as the complementary of the conditional active relation.
Part of the proof is relegated in Appendices~\ref{Additional_material}
and~\ref{Proof_of_Theorem}%
\ifpreprint , as well as a treatment of the moral
  relation~\citep{Lauritzen-et-al-1990} with binary relations in
  Appendices~\ref{Moral_relation}.  \else .  \fi

\section{A formal Pearl's $d$-separation definition}
%\section{Conditional active relation induced by endpoints of non-blocking paths}
\label{Conditional_active_relation_induced_by_endpoints_of_non-blocking_paths}

In~\S\ref{UndirectedEdgePaths_in_a_graph}, we deal with
graphs but using the concepts of binary relations;
we formally define what we call \undirectedEdgePaths\ in a graph.
Thus equipped, in~\S\ref{Definition_of_the_conditional_directional_separation_binary_relation}
we formally adapt Pearl's definition of active (and blocked)
\undirectedEdgePaths\ in a graph,
from which we deduce the (conditional) d-separation binary relation.
Then, we also introduce elementary binary relations that are the bricks
to define a new binary relation that we call conditional active relation.

\subsection{\UndirectedEdgePaths\ in a graph}
\label{UndirectedEdgePaths_in_a_graph}

We employ the vocabulary and concepts both of binary relations and of graph theory.
% 
We take inspiration from~\citep{PEARL1986357} to define the notion of blocked paths
on a graph, not necessarily finite nor acyclic.
% 
We use the notation \( \ic{r,s}=\na{r,r+1,\ldots,s-1,s} \) for two integers $r \leq s$.

\subsubsection{Background on binary relations}

Let $\AGENT$ be a nonempty set (finite or not). 
We recall that a \emph{(binary) relation}~$\relation$ on~$\AGENT$ is
a subset $\relation \subset \AGENT\times\AGENT $ and that 
\( \bgent\, \relation\, \cgent \) means 
\( \np{\bgent,\cgent} \in \relation \).
For any subset \( \Bgent \subset \AGENT \), 
the \emph{(sub)diagonal relation} is \( \Delta_{\Bgent} = \bset{ \np{\bgent,\cgent} \in \AGENT\times\AGENT }%
{ \bgent=\cgent \in \Bgent } \)
and the \emph{diagonal relation} is \( \Delta=\Delta_{\AGENT} \).
%
A \emph{foreset} of a relation~$\relation$ is
any set of the form \( \relation \, \cgent = 
\defset{ \bgent \in  \AGENT }{ \bgent\, \relation \, \cgent } \),
where \( \cgent \in \AGENT \), 
or, by extension, of the form \( \relation \, \Cgent = 
\defset{ \bgent \in  \AGENT }{ \exists \cgent \in \Cgent \eqsepv \bgent\,
  \relation \, \cgent } \), where \( \Cgent \subset \AGENT \).
%
An \emph{afterset} of a relation~$\relation$ is
any set of the form \( \bgent \, \relation = 
\defset{ \cgent \in  \AGENT }{ \bgent\, \relation \, \cgent } \),
where \( \bgent \in \AGENT \), 
or, by extension, of the form \( \Bgent \, \relation = 
\defset{ \cgent \in  \AGENT }{ \exists \bgent \in \Bgent \eqsepv \bgent\,
  \relation \, \cgent } \), where \( \Bgent \subset \AGENT \).
%
The \emph{opposite} or \emph{complementary~$\Complementary{\relation}$} of a binary
relation~$\relation$ is the relation~$\Complementary{\relation}=\AGENT\times\AGENT\setminus\relation$,
that is, defined by \( \bgent\, \relation^{\mathsf{c}} \, \cgent \iff 
\neg \np{ \bgent\, \relation \, \cgent } \).
The \emph{converse~$\Converse{\relation}$} of a binary relation~$\relation$ is
defined by \( \bgent\, \Converse{\relation} \, \cgent \iff \cgent\, \relation \, \bgent
\) (and $\relation$ is  \emph{symmetric} if \( \Converse{\relation}=\relation \)).
The \emph{composition}
$\relation\relation'$ of two
binary relations~$\relation, \relation'$ on~$\AGENT$ is defined by
\( \bgent (\relation\relation') \cgent \iff
\exists \delta \in  \AGENT \), \( \bgent\, \relation \, \delta \)
and \( \delta\, \relation' \, \cgent \);
then, by induction we define\footnote{%
  In what follows, when we consider a binary relation
  as a subset $\relation \subset \AGENT\times\AGENT $, we will use the notation
  \( \SetProd{\relation}{n} \subset \SetProd{\AGENT\times\AGENT}{n}\), where $n$ is
  a positive integer, to
  denote a product subset of the product set~\( \AGENT^{2n} \),
  thus making the distinction with the binary relation
  $\relation^{n} \subset \AGENT\times\AGENT $ obtained by $n$~compositions.}
\( \relation^{n+1}=\relation\relation^{n} \) for \( n \in \NN^* \). 
The \emph{transitive closure} of a binary relation~$\relation$ is
\( \TransitiveClosure{\relation} = \cup_{k=1}^{\infty} \relation^{k} \)
(and $\relation$ is  \emph{transitive} if \( \TransitiveClosure{\relation}=\relation \))
and the \emph{reflexive and transitive closure} is 
\( \TransitiveReflexiveClosure{\relation}= \TransitiveClosure{\relation} \cup
\Delta = \cup_{k=0}^{\infty} \relation^{k} \) with the convention $\relation^0=\Delta$. 
% 
A \emph{partial equivalence relation} is a symmetric and transitive binary
relation (generally denoted by~$\sim$ or~$\equiv$).
An \emph{equivalence relation} is a reflexive, symmetric and transitive binary
relation.

\coqstart
Relations are implemented as sets on a product space using the classical sets implemented in
\texttt{classical\_sets.v} from the Coq mathcomp library~\cite{MathComp:2022}
using ssreflect tactics~\cite{Gonthier-Mahboubi-Tassi:2016} 
\inputsnippetsE{ssrel/relation.tex}{.}
As described in more details below, we have developed a library for relations taking into account all the
definitions recalled at the begining of this section.\coqend

% ,ssrel/clos_refl_trans.tex,ssrel/clos_rt.tex}
\subsection{Sequences sets and relations}
\label{sec:sequences}
\newcommand{\fullset}{T}

To formalize graph paths we use sequences (as defined in mathcomp
\texttt{seq.v}) combined with set formalization (defined in mathcomp
\texttt{classical\_sets}). To describe the formalization we start by introducing some
notations.

\paragraph{$\bullet$ \protect\pinX.} We consider a set $\fullset$, and for any set ${\Set} \subset \fullset$ and $n\in \NN$,
we denote by
$\Sequence_n(\Set)$ the set of sequences of length $n$ of elements of the
set~$\Set$, $\Sequence_n(\Set)= \SetProd{\Set}{n}$ ($\Sequence_0(\Set)$ being the singleton set with the empty sequence) and by
$\Sequence_{\ge n}(\Set)$ the set of finite sequences of length greater or equal
to $n$ of elements of the set $\Set$, that is the disjoint union
$\sqcup_{k\ge n} \Sequence_k(\Set)$. The set $\Sequence_{\ge 0}(\Set)$ will be denoted by $\Sequence(\Set)$.
\medskip

\coqstart The sets $\Sequence_{\ge n}(\fullset)$ and $\Sequence_{n}(\fullset)$ are
formalized in Coq in the mathcomp library as sequences of elements of type
\texttt{T} and the restriction to elements in a subset \texttt{(\PY{n+nv}{D}:
  set T)} is obtained using the function \texttt{all} (in mathcomp
 mathcomp library\texttt{seq.v}).  As an example $\Sequence_{n}(D)$ is implemented as follows
{\small \inputsnippets{seq1/all_notation-n.tex,seq1/Sn-n.tex}}\coqend

\paragraph{$\bullet$ (\protect\Liftp) and (\protect\Liftpin).} Then, we define a lift operator
\[ \Lift: \Sequence(\fullset) \to \Sequence(\fullset{\times}\fullset)
  \eqfinv\] such that, for all $n \ge 2$, the restriction of the operator
$\Lift$ to the set $\Sequence_{n}(\fullset)$ coincide with the
mapping \(\Lift_n: \Sequence_{n}(\fullset)\to \Sequence_{n{-}1}(\fullset{\times}\fullset)\)
transforming a sequence of elements of $\fullset$ of length $n$
into a sequence of oriented pairs in $\fullset{\times}\fullset$ of length $n{-}1$
\begin{align}
  \forall (\vertex_1,\dots, \vertex_n)\in \SetProd{\fullset}{n}
  \eqsepv
  \Lift_n(\vertex_1,\dots, \vertex_n) = \bp{\np{\vertex_1, \vertex_2},\np{\vertex_2, \vertex_3},\dots,
  \np{\vertex_{n-1},\vertex_n}}
  \eqfinv
\end{align}
and the restriction of the operator $\Lift$ on
$\Sequence_{0}(\fullset) \cup\Sequence_{1}(\fullset)$ is the constant mapping giving the empty list on $\fullset{\times}\fullset$.

\coqstart The lift operator $\Lift$ is implemented in Coq as a recursive mapping denoted
\texttt{Lift}: {\small \inputsnippetsE{seq1/Lift.tex}{,} }
and note that using Coq polymorphism the list operator is parameterized by a type
and thus can be used also to Lift a sequence of edeges (in $\fullset\times\fullset$) into
a sequence of pair of edges.
\coqend

The notation $\protect\Liftpin$ is used to denote the
expression {\PY{n}{(Lift p)}~\PY{o}{$[{\in}]$}~\PY{n}{R}}.

\paragraph{$\bullet$ \protect\pSucR.} We must be able to check that successive elements of a
sequence whose elements are in $\fullset$ belong to a subset of
$\fullset{\times}\fullset$, that is satisfy a relation on $\fullset$.

\coqstart This is easily implemented
with the help of an inductive predicate (\texttt{RPath} in \texttt{seq1.v})
{\small \inputsnippetsE{seq1/RPath.tex}{,}}
that we do not detail here as we prove that it can be equivalently implemented with the Lift mapping
(which enable more computational proofs) as we have
{\small \inputsnippets{seq1/RPath_equiv-n.tex}}\coqend

As a first example, consider the
relation $C_{\textsc{H}}$ -- denoted by \texttt{Chrel} in Coq -- on the product set
$T{\times}T$ defined by $((v_1, v_2)\, C_{\textsc{H}}\, (v_3, v_4) \iff v_2 = v_3)$,
{\small \inputsnippetsE{seq1/Chrel}{.}}
Now, the fact that lifted sequences are well chained sequences can be stated as proving the following lemma
{\small \inputsnippets{seq1/Lift_Suc-n.tex}}

As a second example, if the elements of a sequence belong to a set $X$, then the 
elements of the lifted sequence belong to the product relation $X{\times}X$ as proved in the following lemma
{\small \inputsnippets{seq1/Rpath_L1-n.tex}}

\paragraph{$\bullet$ \texttt{Lift} bijection.} The Lift operation when restricted to the subset \texttt{D} is bijective
onto its image \texttt{I}
{\small \inputsnippets{seq1/DI.tex,seq1/Lift_inj-n.tex,seq1/Lift_surj-n.tex}}
The inverse of \texttt{Lift} is obtained with the recursive mapping \texttt{UnLift}. 

\paragraph{$\bullet$ (\texttt{seq T}) and (\texttt{seq (T*T)}).}
XXXX four equivalent ways -- two are
expressed as image of set and two as preimage of set -- of representing sequences
of pairs of elements of $T$ such that successive pairs satisfy the
$C_{\textsc{H}}$ relation. We obtain that \texttt{(Im 2) = (Im1 2) = (Pre 0) =
  (Pre1 1)}, where the four paths sets are defined by {\small
  \inputsnippets{seq1/Lift_lemma2-n.tex}}

The Lift operator satisfy the equality $\Lift\bp{\Sequence(T)}= \Lift^{(-1)}\bp{ \Sequence(\texttt{Chrel})}$\footnote{%
  Note that mathematically, two different Lift operators appear in this equation as their domain is not the same.
  Nevertheless, we use the same notation as in Coq they are implemented with the same \texttt{List} function using
  type polymorphism}
proved in Coq in Lemma \texttt{Lift\_lemma}
{\small \inputsnippets{seq1/Lift_lemma}}

\subsubsection{Edge paths in a (directed simple) graph (permitting loops)}

Let $\VERTEX$ be a nonempty set (finite or not), whose elements are called \emph{vertices}.  Let
\( \EDGE \subset \VERTEX\times\VERTEX \) be a relation on~$\VERTEX$, whose
elements are ordered pairs (that is, couples) of vertices called \emph{edges}.
The first element of an edge is the \emph{tail of the edge},
whereas the second one is the \emph{head of the edge}.
Both tail and head are called \emph{endpoints} of the edge,
and we say that the edge connects its endpoints.
We define a \emph{loop} %in the graph~$(\VERTEX,\EDGE)$
as an element of \( \Delta \cap \EDGE \), that is, a loop is an edge that
connects a vertex to itself.

A \emph{graph}, as we use it throughout this paper, is a
couple~$(\VERTEX,\EDGE)$.
This definition is very basic and we now stress proximities and differences with
classic notions in graph theory.
As we define a graph, it may hold a finite or infinite number of vertices;
there is at most one edge that has a couple of ordered vertices as single endpoints,
hence a graph (in our sense) is not a multigraph (in graph theory);
loops are not excluded (since we do not impose $\Delta \cap \EDGE=\emptyset$).
Hence, what we call a graph would be called a directed simple graph permitting
loops in graph theory. 

To define blocked and active \undirectedEdgePaths\ relative to the
graph~\( \npOrientedGraph \), we need to fix additional vocabulary and
notation.
In the graph~$(\VERTEX,\EDGE)$, 
the \emph{undirected edges} are the elements of $\EDGE \cap \Converse{\EDGE}$
--- that is, edges with both $(\cgent,\bgent)\in \EDGE$
and $(\bgent, \cgent)\in \EDGE$ (hence, including loops).
Then, the graph~$(\VERTEX,\EDGE)$ is said to be \emph{undirected} if
all edges are undirected edges, 
or, equivalently, if \( \EDGE=\EDGE \cap \Converse{\EDGE} \)
or if \( \Converse{\EDGE} = \EDGE \).
The \emph{undirected extension} of a graph~$(\VERTEX,\EDGE)$
is the graph~$(\VERTEX,\EDGE \cup \Converse{\EDGE})$.

In the graph~$(\VERTEX,\EDGE)$, 
the \emph{directed edges} are the elements of
$\EDGE \cap \npComplementary{ \Converse{\EDGE} }$ --- that is, 
edges with $(\cgent,\bgent)\in \EDGE$ such that $(\bgent, \cgent)\not\in \EDGE$
(recall that we do not assume that $\EDGE \cap \Converse{\EDGE}=\emptyset$).
Then, the graph~$(\VERTEX,\EDGE)$ is said to be \emph{directed} if
all edges are directed edges, 
or, equivalently, if 
\( \EDGE \cap \Converse{\EDGE}=\emptyset \), that is, 
when no two edges have the same endpoints.

\coqstart A graph $(\VERTEX,\EDGE)$ is given in Coq by a type \texttt{(T: Type)} and a relation on
\texttt{T} which is a set in the product type \texttt{set T*T}, that is
$(\EDGE:\texttt{set T*T})$. Thus, the (classical) set definition of mathcomp
analysis \texttt{classical\_sets.v} is used to formalize a graph. \coqend
\medskip

In graph theory, one finds the notions of path, chain and walk. 
To avoid ambiguities,
we formally define an \EdgePath\ in $(\VERTEX,\EDGE)$ (in our sense) 
as the classical notion of path in a graph~\citep{Diestel}.
We also define an \undirectedEdgePath\ in $(\VERTEX,\EDGE)$ (in our sense)
as what corresponds to chain path in~\citep{Lauritzen-et-al-1990}.

\paragraph{$\bullet$ Definition of \emph{\EdgePath}.}
The set of \emph{\EdgePath} of length~$n$ ($n\ge 1$) relative to the graph~$\npOrientedGraph$ is defined by
\begin{align}
  \PATH_n \np{\OrientedGraph} %    \PATH^n\np{\graph}
  &=
    \Bset{ \bseqa{ \np{\tail{\vertex_i},\head{\vertex_i}} }{i\in\ic{1,n}} \in \SetProd{\EDGE}{n} }
    { \head{\vertex_i}=\tail{\vertex_{i+1}} \text{ for } i\in\ic{1,n{-}1}}
    \eqfinp
\end{align}
The set of finite (edge) paths of length greater or equal to $n\ge1$ relative to the graph~$\npOrientedGraph$ is defined by
\begin{equation}
  \PATH_{\ge n}\np{\OrientedGraph} = \sqcup_{n'\ge n} \PATH_{n'}\np{\OrientedGraph}
  \label{def:epaths_gt1}
  \eqfinp
\end{equation}

\coqstart Using the tools introduced in Section~\ref{sec:sequences} and Equation~\eqref{def:epaths_gt1} we obtain 
the following formalization of $P_{> n}$
{\small \inputsnippetsE{seq1/EPath_gt-n.tex}{,}}
where \texttt{E} is the edge relation $\EDGE$ and where \texttt{Chrel} we defined in Section~\ref{sec:sequences}. \coqend

We denote by $\cardinal{\path}$ the length of a path $\path \in
\PATH\np{\OrientedGraph}$.
An \emph{\EdgeSubPath} of the \EdgePath~$\path$ is an \EdgePath\ obtained by a
subsequence of consecutive indices.

\paragraph{$\bullet$ Definition of endpoints.} The first element~$\tail{\vertex_1}$ of an \EdgePath\
\( \path=\bseqa{ \np{\tail{\vertex_i},\head{\vertex_i}} }{i\in\ic{1,n}} \) is the
\emph{tail of the \EdgePath}, whereas the last one~$\head{\vertex_n}$ is the
\emph{head of the \EdgePath}.  Both tail (obtained with function \texttt{head}
in Coq) and head (obtained with function \texttt{last} in Coq) are called
\emph{endpoints} of the \EdgePath.  When an edge path of length greater or equal
to one is given as a Lifted sequence of elements of $T$
as for example in \texttt{\PY{n}{Lift}~\PY{o}{(}\PY{n}{x}\PY{o}{::(}\PY{n}{rcons}~\PY{n}{p}~\PY{n}{y}\PY{o}{))}~\PY{o}{=}~\PY{n}{sp}\PY{o}{.}} the endpoints \texttt{(x,y)} and the
intermediate nodes \texttt{p} of the \EdgePath\ \texttt{sp} are immediately obtained.

\paragraph{$\bullet$ Transitive closure and edge paths.} As a first result linking (edge) paths and relations we can prove that 
{\small \inputsnippetsE{erel3/TCP-n.tex}{,}
which asserts that two nodes \texttt{(v1,v2)} are in relation through the transitive closure of a relation 
\texttt{S}, that is \texttt{(v1,v2) $\in$ S.+} if and only if 
there exists an edge path with endpoints \texttt{(v1,v2)} in the graph $(T,S)$.

\paragraph{$\bullet$ Deployment in paths.} We denote by \( \Projection^n: \PATH^n \np{\OrientedGraph}
\to\VERTEX\times\VERTEX \) the \emph{projection
  on the tail and head endpoints} of an \EdgePath\ of length~$n$.
It is defined, for \( \path\in\PATH^n\np{\OrientedGraph}\), by 
\(\Projection^n\np{\path}=
\Projection^n\np{(\tail{\vertex_i},\head{\vertex_i})_{i\in\ic{1,n}}} 
= \np{\tail{\vertex_1},\head{\vertex_{n}}}\).
We denote by \( \Projection: \PATH \np{\OrientedGraph} \to \VERTEX\times\VERTEX \) 
the projection mapping on the tail and head endpoints defined by
\begin{subequations}
  \begin{equation}
    \forall \path\in \PATH\np{\OrientedGraph}
    \eqsepv
    \Projection\np{\path} = \Projection^{\cardinal{\path}}\np{\path} \in \VERTEX\times\VERTEX 
    \eqfinp
    \label{eq:Projection_PATH}
  \end{equation}
  We also distinguish the tail and the head endpoints projection mappings by
  \begin{equation}
    \Projection=\np{\Tail{\Projection},\Head{\Projection}}
    \mtext{ where }
    \Tail{\Projection}: \PATH \np{\OrientedGraph} \to \VERTEX
    \mtext{ and }
    \Head{\Projection}: \PATH \np{\OrientedGraph} \to \VERTEX
    \eqfinp 
    \label{eq:Projection_PATH_first_and_last}
  \end{equation}  
\end{subequations}

Now, with any binary relation \( \relation \subset \VERTEX{\times}\VERTEX \),
we associate the subset~\( \DeploymentInPaths{\relation}{\OrientedGraph} \)
of~\( \PATH\np{\OrientedGraph} \),
that we call the \emph{deployment in paths},
defined by 
\begin{equation}
  \forall \relation \subset \VERTEX\times\VERTEX 
  \eqsepv 
  \DeploymentInPaths{\relation}{\OrientedGraph}
  = \Projection^{-1}(\relation) \subset \PATH\np{\OrientedGraph} 
  \eqfinv
  \label{eq:DeploymentInPaths}
\end{equation}
where the projection~\( \Projection \) has been defined
in~\eqref{eq:Projection_PATH}.
The deployment in paths~\( \DeploymentInPaths{\relation}{\OrientedGraph} \)
is made of the \EdgePaths\  whose endpoints satisfy the 
binary relation~\( \relation \).

ZZZZ here we need Coq code to illustrate. 

\subsubsection{Extended-oriented paths in a (directed simple) graph (permitting loops)}

\noindent $\bullet$ Definition of \emph{\undirectedEdgePath}. To define
\emph{\undirectedEdgePath}, we consider a finite set ${\Orientation}=\na{-1,+1}$
taking two possible values and implemented in Coq a an inductive type taking two
values \texttt{N} and \texttt{P}
(\texttt{\PY{k+kn}{Inductive}~\PY{n+nf}{O}~\PY{o}{:=}~\PY{o}{|}~\PY{n}{P}~\PY{o}{|}~\PY{n}{N}\PY{o}{.}})
which will serve as an orientation specification of an edge.

Now, for $n\ge 2$, we consider an oriented lift operator
\begin{equation}
  \Lifto_n: \Sequence_{n}(\fullset)\times\Sequence_{n-1}(\Orientation) \to
  \Sequence_{n{-}1}(\fullset{\times}\fullset{\times}\Orientation)
  \eqsepv
\end{equation}
transforming a sequence of elements of $\fullset$ of length $n$ and a sequence of orientations of length $n{-}1$ into a
sequence of oriented triples in $\fullset{\times}\fullset{\times}\Orientation$ of length
$n{-}1$ as follows
\begin{align}
  \forall \vertex\in \Sequence_n(\fullset)
  \eqsepv
  \forall o\in \Sequence_{n-1}(\Orientation)
  \eqsepv
  \Lifto_n(v,o) = \bp{\np{\vertex_1, \vertex_2,o_1},\dots,
  \np{\vertex_{n-1},\vertex_n,o_{n-1}}}
  \eqfinp
  \label{def:Lifto}
\end{align}
Proceeding as in the previous paragraph, using the sequence of mapping
$\nseqa{\Lifto_n}{n\ge 2}$ we define a lift operator
$$\Lifto: \sqcup_{n\ge 2} \Sequence_{n}(\fullset){\times}\Sequence_{n-1}(\Orientation)
\to \Sequence_{\ge 1}(\fullset{\times}\fullset{\times}\Orientation)$$ such that for all
$n \ge 2$ the restriction of $\Lifto$ to the set
$\Sequence_{n}(\fullset){\times}\Sequence_{n-1}(\Orientation)$ coincide with
$\Lifto_n$.

The set of \emph{\undirectedEdgePath} of length~$n$ ($n\ge 1$) relative to the
graph~$\npOrientedGraph$ is defined by
\begin{equation}
  \UPATH^n\np{\OrientedGraph} =
  \Lifto_n \bp{ \Sequence_{n+1}(\VERTEX){\times}\Sequence_{n}(\Orientation)}
  \cap \Sequence_{n}(\EDGEo)
  \text{ and satisfy } \PATH^n\np{\OrientedGraph} \subset \Sequence_{n}(\EDGEo)
  \eqfinv
\end{equation}
where the relation $\EDGEo$ on the set $\EDGE\times\Orientation$ is defined by 
\begin{align}
  (\vertex,\vertexbis,o) \in \EDGEo
  \iff (\vertex, \vertexbis) \in \EDGE^{(o)},
  \text{ with } \EDGE^{(+1)}= \EDGE \eqsepv \EDGE^{(-1)}= \Converse{\EDGE}
  \eqfinp
  \label{def:Eo_relation}
\end{align}

Finally, the set of finite \undirectedEdgePaths\footnote{%
  It is to be noted that an \undirectedEdgePath\ is \emph{not a path in the graph
    $\npOrientedGraph$}, neither in the undirected graph $\npUnorientedGraph$.
  However, considering a couple $(\path,\orient)\in \UPATH^n\np{\OrientedGraph}$, we obtain that
  $\path \in \PATH^n\np{\UnorientedGraph}$,
  that is, $\path$ is an (edge) path in the unoriented graph
  \( \npUnorientedGraph \).
  We thus obtain a natural surjection
  $(\path,\orient)\mapsto \path$
    from $\UPATH\np{\OrientedGraph}$ to $\PATH\np{\UnorientedGraph}$.
    This canonical surjection is not necessary injective because
    a path in $\PATH\np{\UnorientedGraph}$ that has an edge in $\EDGE \cap \Converse{\EDGE}$
    is the image of two distinct \undirectedEdgePaths.
    The surjection $(\path,\orient)\mapsto \path$ 
    is a bijection in the special case when 
    the graph~\( \npOrientedGraph \) is directed,
    that is, when $\EDGE \cap \Converse{\EDGE}=\emptyset$, that is, 
    when no two edges have the same endpoints. 
  } of length greater or equal to one
  relative to the graph~$\npOrientedGraph$ 
  is defined by
\begin{align}
  \UPATH_{\ge 1}\np{\OrientedGraph}
  &=
    \Lifto \bp{\sqcup_{n\ge 2} \Sequence_{n}(\Set){\times}\Sequence_{n-1}(\Orientation)}
    \cap \Sequence_{\ge 1}(\EDGEo)
    \text{ and satisfy } \UPATH_{\ge 1}\np{\OrientedGraph}  \subset \Sequence_{\ge 1}(\EDGEo) 
    \eqfinp
    \label{def:eopaths_gt1}
\end{align}
An \emph{\undirectedEdgeSubPath} of the \undirectedEdgePath~$\UndirectedPath
\in\UPATH\np{\OrientedGraph}$ is an \undirectedEdgePath\ obtained by a
subsequence of consecutive indices. 

The coq formalization of \undirectedEdgePaths\ paths of size greater or equal to
one follows the same path as the one used for (edge) paths except that we use
the oriented lift operator defined in Equation~\eqref{def:Lifto} implemented as 
in Coq as follows
{\small \inputsnippets{seq1/pair.tex,seq1/LiftO.tex}}
Moreover, we have to
check that elements of paths satisfy the $\EDGEo$ relation defined in
Equation~\eqref{def:Eo_relation}. For that purpose,  the mapping \texttt{Oedge} is used to
construct a relation $\EDGEo$ given a relation $\EDGE$
{\small \inputsnippets{erel3/Oedge.tex}
Finally, the set of
\undirectedEdgePaths\ of size greater or equal to one,
$\UPATH_{\ge 1}\np{\OrientedGraph}$ is defined by
{\small \inputsnippets{erel3/U_ge_1.tex,erel3/U_ge_1p.tex}}

\medskip

\noindent $\bullet$ 
We denote by \( \Projection^n: \PATH^n \np{\OrientedGraph}
\to\VERTEX\times\VERTEX \) the \emph{projection
  on the tail and head endpoints} of an \EdgePath\ of length~$n$.
It is defined, for \( \path\in\PATH^n\np{\OrientedGraph}\), by 
\(\Projection^n\np{\path}=
\Projection^n\np{(\tail{\vertex_i},\head{\vertex_i})_{i\in\ic{1,n}}} 
= \np{\tail{\vertex_1},\head{\vertex_{n}}}\).
We denote by \( \Projection: \PATH \np{\OrientedGraph} \to \VERTEX\times\VERTEX \) 
the projection mapping on the tail and head endpoints defined by
\begin{subequations}
  \begin{equation}
    \forall \path\in \PATH\np{\OrientedGraph}
    \eqsepv
    \Projection\np{\path} = \Projection^{\cardinal{\path}}\np{\path} \in \VERTEX\times\VERTEX 
    \eqfinp
    \label{ZZeq:Projection_PATH}
  \end{equation}
  We also distinguish the tail and the head endpoints projection mappings by
  \begin{equation}
    \Projection=\np{\Tail{\Projection},\Head{\Projection}}
    \mtext{ where }
    \Tail{\Projection}: \PATH \np{\OrientedGraph} \to \VERTEX
    \mtext{ and }
    \Head{\Projection}: \PATH \np{\OrientedGraph} \to \VERTEX
    \eqfinp 
    \label{ZZeq:Projection_PATH_first_and_last}
  \end{equation}  
  We use the notation \( \Projection_{\UPATH}: \UPATH \np{\OrientedGraph} \to\VERTEX\times\VERTEX \) 
  to denote the projection mappings on the tail and head endpoints of an \undirectedEdgePath. 
  It is defined, for $\np{\path,\orient}\in \UPATH\np{\OrientedGraph}$, by 
  \begin{equation}
    \Projection_{\UPATH}\bp{\np{\path,\orient}} =  \Projection\np{\path} 
    \eqsepv
    \forall \np{\path,\orient}\in \UPATH\np{\OrientedGraph}
    \eqfinp
    \label{ZZeq:Projection}
  \end{equation}
\end{subequations}
\medskip

\ifpreprint 
\noindent $\bullet$ We denote by \( \Internals^n: \PATH^n \np{\OrientedGraph}
\rightrightarrows\VERTEX \) the set-valued mapping giving the 
\emph{intermediate vertices} of an \EdgePath\ of length~$n$ ($n\ge 0$).
It is defined, for %\( \path=\sequence{\tail{\vertex_i},\head{\vertex_i}}{i\in\ic{1,n}}
\( \path=\sequence{\vertex_i,\vertex_{i+1}}{i\in\ic{1,n}}
\in\PATH^n\np{\OrientedGraph}\) in~\eqref{eq:PATHOrientedGraph}, by
\begin{subequations}
  \begin{equation}
    \Internals^n\np{\path}=
    \Internals^n\bp{\sequence{\vertex_i,\vertex_{i+1}}{i\in\ic{1,n}}} =  
    % \Internals^n\bp{\sequence{\tail{\vertex_i},\head{\vertex_i}}{i\in\ic{1,n}}} =
    \na{\vertex_2,\ldots,\vertex_{n}}
  \end{equation}
  (which reduces to the empty set for $n=0$ or $1$).
  Notice that
  \[
    \na{\vertex_1,\vertex_2,\ldots,\vertex_{n},\vertex_{n{+}1}}
    =   \na{\vertex_1} \cup \na{\vertex_2,\ldots,\vertex_{n}} \cup \na{\vertex_{n{+}1}}
    = \Tail{\Projection}\np{\path} \cup \Internals^n\np{\path}
    \cup \Head{\Projection}\np{\path}
    \eqfinp
  \]
  We denote by \( \Internals: \PATH \np{\OrientedGraph}\rightrightarrows \VERTEX
  \)
  the set-valued mapping giving the intermediate vertices of an \EdgePath\ by
  \begin{equation}
    \forall \path \in \PATH \np{\OrientedGraph}
    \eqsepv
    \Internals\np{\path} = \Internals^{\cardinal{\path}}\np{\path} 
    \eqfinp
    \label{eq:Internals_PATH}
  \end{equation}
  We use the notation   \( \Internals_{\UPATH}: \UPATH \np{\OrientedGraph} \rightrightarrows\VERTEX \)
  to denote the set-valued mapping giving the intermediate vertices of an \undirectedEdgePath. 
  It is defined, for $\np{\path,\orient}\in\UPATH\np{\OrientedGraph}$
  in~\eqref{eq:undirectedEdgePaths}, by 
  \begin{equation}
    \forall \np{\path,\orient}\in \UPATH\np{\OrientedGraph}
    \eqsepv
    \Internals_{\UPATH}\bp{\np{\path,\orient}} =  \Internals\np{\path} 
    \eqfinp
    \label{eq:Internals_UPATH}
    % \label{eq:Projection}
  \end{equation}
\end{subequations}
\medskip
\fi 

\noindent \(\bullet\)
To define blocked \undirectedEdgePaths, 
the following notions prove useful. 
With any binary relation \( \relation \subset \VERTEX\times\VERTEX \),
we associate the subset~\( \DeploymentInUPaths{\relation}{\OrientedGraph} \)
of~\( \UPATH\np{\OrientedGraph} \) in~\eqref{eq:undirectedEdgePaths},
that we call the \emph{deployment in \undirectedEdgePaths}, 
defined by 
\begin{equation}
  \forall \relation \subset \VERTEX\times\VERTEX 
  \eqsepv 
  \DeploymentInUPaths{\relation}{\OrientedGraph}
  = \Projection_{\UPATH}^{-1}(\relation) \subset \UPATH\np{\OrientedGraph} 
  \eqfinv
  \label{eq:DeploymentInUPaths}
\end{equation}
where the projection~\( \Projection_{\UPATH} \) has been defined
in~\eqref{eq:Projection}.
The deployment \(\DeploymentInUPaths{\relation}{\OrientedGraph} \)
is made of the \undirectedEdgePaths\ whose endpoints satisfy the 
binary relation~\( \relation \).
%
We also associate with~\( \relation \)
the subset~\( \DeploymentInPaths{\relation}{\OrientedGraph} \)
of~\( \PATH\np{\OrientedGraph} \),
that we call the \emph{deployment in paths},
defined by 
\begin{equation}
  \forall \relation \subset \VERTEX\times\VERTEX 
  \eqsepv 
  \DeploymentInPaths{\relation}{\OrientedGraph}
  = \Projection^{-1}(\relation) \subset \PATH\np{\OrientedGraph} 
  \eqfinv
  \label{eq:DeploymentInPaths}
\end{equation}
where the projection~\( \Projection \) has been defined
in~\eqref{eq:Projection_PATH}.
The deployment in paths~\( \DeploymentInPaths{\relation}{\OrientedGraph} \)
is made of the \EdgePaths\  whose endpoints satisfy the 
binary relation~\( \relation \).


\subsection{Definitions of active \undirectedEdgePaths, of  conditional
  directional separation and of conditional active binary relation}
\label{Definition_of_the_conditional_directional_separation_binary_relation}

% \subsubsection{Definition of active \undirectedEdgePaths\ in a graph}

We define active \undirectedEdgePaths\ relative to the graph~$\npOrientedGraph$
in Definition~\ref{de:ActivePaths}.
Then, we  obtain the definition of blocked \undirectedEdgePaths\ relative to the
graph~$\npOrientedGraph$, as defined by~\citep{PEARL1986357}, by switching to the
complementary set.
% We define blocked \undirectedEdgePaths\ relative to the graph~$\npOrientedGraph$ as follows.

\begin{definition}
  % \label{de:BlockedUndirectedPaths}
  \label{de:ActivePaths}
  Let \( \npOrientedGraph \) be a graph, 
  and $\AgentSubsetW\subset\VERTEX$ be a subset of vertices.
  % 
  We say that an  \undirectedEdgePath\
  \(\UndirectedPath=\np{\path,\orient} \in
  \UPATH\np{\OrientedGraph} \) in~\eqref{eq:undirectedEdgePaths}
  relative to the graph~\(\npOrientedGraph\)
  is an \emph{active \undirectedEdgePath} (\wrt\footnote{With respect to} the subset~$\AgentSubsetW$)
  if %it is an \undirectedEdgePath\ and that,
  either its length is~0 or~1,
  or (when its length is at least~$2$)
  \(\UndirectedPath=\np{\path,\orient}=\bp{\nseqp{(\tail{\vertex_i},\head{\vertex_i})}{i \in
      \ic{1,\cardinal{\path}}},\nseqp{\orient_i}{i \in \ic{1,\cardinal{\path}}}}\)
  is such that, for all $i\in \ic{1,\cardinal{\path}-1}$, the consecutive extended-oriented subpaths
  \(\bp{\np{\tail{\vertex_i},\head{\vertex_i}}, \orient_i}\)  and
  \(\bp{\np{\tail{\vertex_{i+1}},\head{\vertex_{i+1}}}, \orient_{i+1}}\)
  satisfy one of the four following cases
  \begin{enumerate}
  \item
    $\orient_i = +1$, $\orient_{i+1}= +1$ and $\head{\vertex_i} = \tail{\vertex_{i+1}} \in \Complementary{\AgentSubsetW}$,
    \label{it:ActivePaths_case1}
  \item
    $\orient_i = -1$, $\orient_{i+1}= -1$ and $\head{\vertex_i} = \tail{\vertex_{i+1}} \in \Complementary{\AgentSubsetW}$,
    \label{it:ActivePaths_case2}
  \item
    $\orient_i = -1$, $\orient_{i+1}= +1$ and $\head{\vertex_i} = \tail{\vertex_{i+1}} \in \Complementary{\AgentSubsetW}$,
    \label{it:ActivePaths_case3}
  \item
    $\orient_i = +1$, $\orient_{i+1}= -1$ and $\head{\vertex_i} = \tail{\vertex_{i+1}} \in \TransitiveReflexiveClosureOfSet{\AgentSubsetW}$,
    \label{it:ActivePaths_case4}
  \end{enumerate}
  where \( \TransitiveReflexiveClosure{\EDGE}= \TransitiveClosure{\EDGE} \cup \Delta \)
  is the reflexive and transitive closure of the relation~$\EDGE$,
  and where
  
  \begin{equation}
    \TransitiveReflexiveClosureOfSet{\AgentSubsetW}=
    \TransitiveReflexiveClosure{\EDGE}\AgentSubsetW 
    \eqfinp
    \label{eq:TransitiveReflexiveClosure_AgentSubsetW}
  \end{equation}
  We denote by \( \ActiveUndirectedPaths{\OrientedGraph} \subset \UPATH\np{\OrientedGraph}\)
  the subset of all active \undirectedEdgePaths\ (\wrt\ the subset~$\AgentSubsetW$). 
  We say that an \undirectedEdgePath\ is \emph{blocked} if it is not active
  and we denote by \( \BlockedUndirectedPaths{\OrientedGraph} = 
  \bpComplementary{\ActiveUndirectedPaths{\OrientedGraph}}\)   
  the subset of all blocked \undirectedEdgePaths\ (\wrt\ the subset~$\AgentSubsetW$).
\end{definition} 
Notice that any \undirectedEdgePath\ of length~0 or 1 is active by definition.

{\small \inputsnippets{erel3/ActiveOe.tex,erel3/Aeop.tex}}

We introduce in Definition~\ref{de:vertices-d-separated} a new binary relation
between vertices: we say that 
two vertices are (conditionally) directionally separated if and only if
all the \undirectedEdgePaths, having them as endpoints, are blocked.
This definition mimics Pearl's d-separation, but with two differences:
the graph is not supposed to be acyclic, and the separation is between vertices
and not between disjoint subsets. 

\begin{definition}
  \label{de:vertices-d-separated}
  Let \( \npOrientedGraph \) be a graph, 
  and $\AgentSubsetW\subset\VERTEX$ be a subset of vertices.
  % 
  % Let $\bgent$, $\cgent\in \AGENT$ be two vertices.
  We denote
  \begin{equation}
    \bgent \ConditionalDirectionalSeparation \cgent \mid \AgentSubsetW
    \iff 
    \bcDeploymentInUPaths{\na{\np{\bgent,\cgent}}}{\graph} \subset
    \BlockedUndirectedPaths{\graph}
    \qquad \bp{ \forall \bgent,\cgent \in \AGENT }
    \eqfinv
    \label{eq:vertices-d-separated}
  \end{equation}
  and we say that the vertices~$\bgent$ and $\cgent$ are 
  (conditionally) \emph{directionally separated}  (\wrt\ the subset~$\AgentSubsetW$).
\end{definition}
{\small \inputsnippets{paper_csbr_paths/dsepnota-n.tex,erel3/Active-n}}



\section{Characterization  of $d$-separation by means of binary relations}
\label{Characterization__of_the_conditional_directional_separation_binary_relation}

Our main result is the characterization of the conditional directional
separation relation (extension of the d-separation)
as the complementary of the conditional active relation defined below in
Equation~\eqref{eq:conditional_active_relation}.
% We just give a sketch of proof here, as details are to be found in
% Appendix~\ref{Proof_of_Theorem}. 

For the purpose of characterizing the conditional directional separation relation~\(
\ConditionalDirectionalSeparation \), we introduce the following binary relations on~$\AGENT$.

\begin{definition}
  \label{de:all_the_relations}
  Let \( \npOrientedGraph \) be a graph, 
  and $\AgentSubsetW\subset\VERTEX$ be a subset of vertices.
  % 
  We define the \emph{conditional parental relation}~\( \ParentalPrecedence \)
  as
  \begin{subequations}
    \begin{align}
      \ParentalPrecedence 
      &= \Delta_{\Complementary{\AgentSubsetW}}\Precedence
        \mtext{ \qquad that is, }
        \bgent\ParentalPrecedence\cgent \iff
        \bgent\in\Complementary{\AgentSubsetW} \mtext{ and }
        \bgent\Precedence\cgent \qquad \bp{\forall \bgent,\cgent \in \AGENT }
        \eqfinv
        \label{eq:conditional_parental_relation}
        % 
        \intertext{the \emph{conditional ascendent relation}
        \( \ConditionalAscendent \) as }
        \ConditionalAscendent &=
                                \ConditionalDown = \Precedence \TransitiveReflexiveClosureParentalPrecedence
                                \label{eq:conditional_ascendent_relation}
                                % 
                                \intertext{which relates a descendent with
                                an ascendent by means of %a bridge made of
                                elements in~$\Complementary{\AgentSubsetW}$u. 
                                We define their converses~\( \ConverseParentalPrecedence \) and
                                \( \ConverseConditionalAscendent \) as }
                                \ConverseParentalPrecedence
      &= \npConverse{\ParentalPrecedence}
        = \Converse{\Precedence} \Delta_{\Complementary{\AgentSubsetW}}
        \eqfinv
        \label{eq:converse_conditional_parental_relation}
      \\
      \ConverseConditionalAscendent
      &= \Converse{\bp{\ConditionalAscendent}}
        = \ConditionalUp
        = \TransitiveReflexiveClosureConverseParentalPrecedence \Converse{\Precedence}
        \eqfinp
        \label{eq:converse_conditional_ascendent_relation}
        \intertext{With these elementary binary relations,
        we define the \emph{conditional common cause relation}~$\ConditionalCommonCause$ 
        as the symmetric relation}
        \ConditionalCommonCause 
                              &=
                                \ConverseConditionalAscendent \Delta_{\Complementary\AgentSubsetW}
                                \ConditionalAscendent
                                = \TransitiveClosureConverseParentalPrecedence \TransitiveClosureParentalPrecedence
                                \label{eq:common_cause}
                                \eqfinv
                                % \\
      \intertext{the \emph{conditional cousinhood relation}~$\Cousinhood$
      as the partial equivalence relation} 
      \Cousinhood
      &=
        \bpTransitiveClosure{\Delta_{\AgentSubsetW} \ConditionalCommonCause
        \Delta_{\AgentSubsetW} }
        \cup
        \Delta_{\AgentSubsetW}
        \eqfinv
        \label{eq:Cousinhood}
        % \\
      \intertext{
      and the \emph{conditional active relation}~$\ConditionalActive$ 
      as the symmetric relation}
      \ConditionalActive
      &= \Delta \cup 
        \ConditionalAscendent \cup \ConverseConditionalAscendent \cup \ConditionalCommonCause
        \cup
        \bp{\ConditionalAscendent \cup \ConditionalCommonCause}
        \Cousinhood
        \bp{\ConverseConditionalAscendent \cup \ConverseConditionalCommonCause}
        \eqfinp
        \label{eq:conditional_active_relation}
    \end{align}
  \end{subequations}
\end{definition}


\begin{theorem}
  \label{th:ConditionalDirectionalSeparation_IFF_relation}
  Let \( \npOrientedGraph \) be a graph, 
  and $\AgentSubsetW\subset\VERTEX$ be a subset of vertices.
  % 
  The conditional directional separation relation~$\ConditionalDirectionalSeparation$
  (Definition~\ref{de:vertices-d-separated})
  is the complementary~$\bpComplementary{ \ConditionalActive }$ of the
  conditional active relation~$\ConditionalActive$
  (Equation~\eqref{eq:conditional_active_relation} in Definition~\ref{de:all_the_relations}):
  \begin{equation}
    \bp{  \;\protect\ConditionalDirectionalSeparation \; \mid \AgentSubsetW }
    =
    \bpComplementary{ \ConditionalActive }
           \mtext{ \qquad that is, }
    \bgent\ConditionalDirectionalSeparation\cgent
      \; \mid \AgentSubsetW  \iff
      \neg \np{ \bgent\ConditionalActive\cgent }
      \qquad \bp{ \forall \bgent,\cgent \in \AGENT }
    \eqfinp 
  \end{equation}
  In other words, we have that 
  \begin{equation}
    \bset{ \np{\bgent,\cgent} \in \AGENT\times\AGENT }{  \DeploymentInUPaths{\na{\np{\bgent,\cgent}}}{\graph} \subset
      \BlockedUndirectedPaths{\graph} }
    = \npComplementary{\ConditionalActive}
    \eqfinp
  \end{equation}
\end{theorem}

\begin{proof}
  Let \( \bgent,\cgent \in \AGENT \) be two vertices. 
  In Proposition~\ref{pr:ConditionalDirectionalSeparation_seilpmi_relation},
  we show that 
  \( \bgent \ConditionalActive \cgent \implies 
  \neg \bp{\bgent \ConditionalDirectionalSeparation \cgent \mid \AgentSubsetW}
  \) or, equivalently (see~\eqref{eq:vertices-d-separated} in
  Definition~\ref{de:vertices-d-separated}), that 
  \(  \bgent {\ConditionalActive} \cgent \implies
  \DeploymentInUPaths{\na{\np{\bgent,\cgent}}}{\graph} 
  \cap \ActiveUndirectedPaths{\graph} 
  \neq\emptyset \).
  % 
  We simply give a sketch of proof here, as details are to be found in the Appendix~\ref{Proof_of_Theorem}. 
  % 
  As the binary relation~\(  \ConditionalActive \) is given by the union of five
  relations in~\eqref{eq:conditional_active_relation},
  the proof examines five cases and exhibits an active path
  (one in~\( \ActiveUndirectedPaths{\graph} \), see Definition~\ref{de:ActivePaths})
  that joins the vertices~\( \bgent \) and \( \cgent \) in the five cases
  when \( \bgent \ConditionalActive \cgent \). 
  \medskip

  % In Proposition~\ref{pr:ConditionalDirectionalSeparation_implies_relation},
  Now,  we show that 
  \( \neg \bp{\bgent \ConditionalDirectionalSeparation \cgent \mid \AgentSubsetW}
  \implies \bgent \ConditionalActive \cgent \) or, equivalently (see~\eqref{eq:vertices-d-separated} in
  Definition~\ref{de:vertices-d-separated}), that 
  \(
  \DeploymentInUPaths{\na{\np{\bgent,\cgent}}}{\graph} 
  \cap \ActiveUndirectedPaths{\graph} 
  \neq\emptyset 
  \implies
  \bgent {\ConditionalActive} \cgent \).
  For this purpose, we assume that there exists an \undirectedEdgePath\
  \( \UndirectedPath \in \UPATH^{n}\np{\graph} \)
  joining the two vertices~$\bgent$ and $\cgent$ and such that \( \UndirectedPath
  \) is active.
  We are going to prove that $\bgent \ConditionalActive\cgent$.
  
  If the path length of~\( \UndirectedPath \) is smaller or equal to one, then,
  by definition~\eqref{eq:undirected_(edge)_paths_of_length_n} of
  $\UPATH^{1}\np{\OrientedGraph}$, we necessarily have that
  either $\bgent \Precedence\cgent$, or $\bgent \Converse{\Precedence}\cgent$,
  or $\bgent \Delta\cgent$.
  Now, as \( \Precedence \subset \ConditionalAscendent \subset
  \ConditionalActive \)
  by~\eqref{eq:conditional_ascendent_relation}
  and~\eqref{eq:conditional_active_relation}, 
  as \( \Converse{\Precedence} \subset \ConverseConditionalAscendent \subset
  \ConditionalActive \)
  by~\eqref{eq:converse_conditional_ascendent_relation}
  and~\eqref{eq:conditional_active_relation}, 
  and as \( \Delta \subset \ConditionalActive \)
  by~\eqref{eq:conditional_ascendent_relation}
  and~\eqref{eq:conditional_active_relation},
  we conclude that $\bgent \ConditionalActive \cgent$.
  
  If the path length of~\( \UndirectedPath \) is $\ge 2$,
  we use Lemma~\ref{lem:induction-lemma} (proof by induction on the path length)
  and we obtain that $\bgent \ConditionalActivePlus \cgent$ or $\bgent \ConditionalActiveMinus
  \cgent$.
  We therefore conclude that $\bgent \ConditionalActive \cgent$
  since $\ConditionalActive =
  \Delta \cup \ConditionalActivePlus \cup \ConditionalActiveMinus$
  by~\eqref{eq:ConditionalActivePlus_cup_Minus}. 
  \medskip 
  
  This ends the proof.
  % 
\end{proof}









\section{Conclusion}

Together with its two companion
papers~\citep{De-Lara-Chancelier-Heymann-2021,Heymann-De-Lara-Chancelier-2021}, 
this paper is a contribution to providing
another perspective on conditional independence and do-calculus.
In this paper, we have considered directed graphs (DGs) not necessarily acyclic,
and we have shown how the d-separation can be extended beyond acyclic
graphs and can be expressed and characterized as a binary relation
between vertices.
%
The results in this paper are instrumental in proving those
in~\citep{De-Lara-Chancelier-Heymann-2021} on topological conditional separation (t-separation),
hence in the use of t-separation to establish conditional independence
in~\citep{Heymann-De-Lara-Chancelier-2021}.

Moreover, there are other perspectives.
First, such developpement of a theory based on binary relations is interesting in
itself as it makes it amenable to computer aided proof.
Second, there are other notions of separation (between subsets) in graph theory that can also 
be expressed by means of binary relations (between vertices).
We illustrate how. 
  Let \( \npOrientedGraph \) be a graph, 
  and $\AgentSubsetW\subset\VERTEX$ be a subset of vertices,
  and $\bgent$, $\cgent\in \AGENT$ be two vertices.
  Then, the three following statements are equivalent:
  any path from~$\bgent$ to~$\cgent$ passes through~$\AgentSubsetW$;
  there does not exist a path from~$\bgent$ to~$\cgent$ which passes
  through~$\Complementary{\AgentSubsetW}$;
  \( \neg \np{ \bgent \npTransitiveClosure{ \EDGE \Delta_{\Complementary{\AgentSubsetW}} \EDGE} \cgent } \).
  
% \mdl{keep below?}
% Let $\relation  \subset \VERTEX\times\VERTEX$ be a relation on~$\VERTEX$.
% Any (vertex) path in the graph \( \np{\VERTEX,\EDGE} \) is a path in the
% subgraph \( \np{ \TopologicalClosure{\Converse{\EDGE},\EDGE}{\relation}, \EDGE } \).

% \begin{itemize}
% \item
%   There exists a path in the deployement of the relation~$\relation$   which passes
%   through~$\Complementary{\AgentSubsetW}$  
% \item
%   \( \TransitiveReflexiveClosureConverse{\EDGE}\relation
%   \TransitiveReflexiveClosureConverse{\EDGE} \cap 
%   \Delta_{\AgentSubsetW} = \emptyset \)
% \end{itemize}

\appendix


\section{Additional material}
\label{Additional_material}

\subsection{Material on \undirectedEdgePaths\ in a graph}

In~\S\ref{appendix_Definitions_undirectedEdgePaths}, we introduce additional definitions
related to \undirectedEdgePaths\ in a graph.
In~\S\ref{appendix_Lemmas_undirectedEdgePaths}, we show Lemmas that will be instrumental
in the proof of Theorem~\ref{th:ConditionalDirectionalSeparation_IFF_relation}.

\subsubsection{Definitions}
\label{appendix_Definitions_undirectedEdgePaths}

We refer the reader to~\S\ref{UndirectedEdgePaths_in_a_graph}
for the basic definitions. 


\subsubsubsection{Concatenation of \undirectedEdgePaths\ in a graph}

We recall that \( \PATH\np{\OrientedGraph} \) in~\eqref{eq:PATHOrientedGraph}
is the set of finite \EdgePaths\ in the graph~$\npOrientedGraph$,
and that \( \UPATH\np{\OrientedGraph} \) in~\eqref{eq:undirectedEdgePaths}
is the set of finite \undirectedEdgePaths\ relative to the
graph~$\npOrientedGraph$. 

\noindent $\bullet$ 
The \EdgePaths\ \( \pathbis \in \PATH\np{\OrientedGraph}\)
and \( \pathter \in \PATH\np{\OrientedGraph}\)
can be \emph{concatenated} when 
\( \Head{\Projection}\np{\pathbis}=\Tail{\Projection}\np{\pathter} \)
(see Equation~\eqref{eq:Projection_PATH_first_and_last}) to give the new 
\EdgePath
\begin{subequations}
  \begin{equation}
    \path = \pathbis \ltimes \pathter = \np{\pathbis,\pathter}
    \in \PATH\np{\OrientedGraph} 
    \eqfinv
    \label{eq:concatenation_Path}
  \end{equation}
  called \emph{concatenation} of~$\pathbis$ and \( \pathter \).
  It is easily seen that the operation~$\ltimes$ is associative.

  The \undirectedEdgePaths\
  \( \UndirectedPathbis= \np{\pathbis,\orientbis} 
  % =\bp{\np{\vertexbis_i,\vertexbis_i^+}_{i\in\ic{1,n}},\orientbis} 
  \in \UPATH\np{\OrientedGraph}\)
  and
  \( \UndirectedPathter= \np{\pathter,\orientter} 
  % =\bp{\np{\vertexter_j,\vertexter_j^+}_{j=1,\ldots,m},\orientter}
  \in \UPATH\np{\OrientedGraph}\)
  can be \emph{concatenated} when 
  \( \Head{\Projection}\np{\UndirectedPathbis}=
  \Tail{\Projection}\np{\UndirectedPathter} \) to give the new 
  \undirectedEdgePath
  % \begin{subequations}
  \begin{equation}
    \UndirectedPath = \UndirectedPathbis \ltimes\UndirectedPathter
    = \bp{ \np{\pathbis,\pathter}, \np{\orientbis,\orientter} }
    = \bp{ \pathbis \ltimes \pathter, \np{\orientbis,\orientter} }
    % \Bp{ \bp{\np{\vertexbis_i,\vertexbis_i^+}_{i\in\ic{1,n}},
    % \np{\vertexter_j,\vertexter_j^+}_{j=1,\ldots,m}},
    % \np{\orientbis,\orientter} } 
    \in \UPATH\np{\OrientedGraph} 
    \eqfinv
    \label{eq:concatenation_UndirectedPath}
  \end{equation}
  called \emph{concatenation} of~$\UndirectedPathbis$
  and \( \UndirectedPathter \).
  It is easily seen that the operation~$\ltimes$ is associative.
  For any subsets \( \SubsetPathbis \subset \UPATH\np{\OrientedGraph} \) 
  and \( \SubsetPathter \subset \UPATH\np{\OrientedGraph} \) of
  \undirectedEdgePaths, we define 
  \begin{equation}
    \SubsetPathbis  \ltimes \SubsetPathter 
    = \bset{ \UndirectedPathbis \ltimes\UndirectedPathter }%
    { \UndirectedPathbis \in \SubsetPathbis
      \eqsepv
      \UndirectedPathter \in \SubsetPathter } 
    \subset \UPATH\np{\OrientedGraph} 
    \eqfinv
    \label{eq:concatenation_SubsetPath}
  \end{equation}
  with \( \SubsetPathbis  \ltimes \SubsetPathter = \emptyset \)
  when \( \Head{\Projection}\np{\SubsetPathbis} 
  \cap 
  \Tail{\Projection}\np{\SubsetPathter} = \emptyset \).
\end{subequations}
\ifpreprint
It is easily obtained that the intermediate vertices~\eqref{eq:Internals_PATH}
of a concatenation are given by
\begin{equation}
  \Internals_{\UPATH}\np{\UndirectedPath} = 
  \Internals_{\UPATH}\np{\UndirectedPathbis \ltimes\UndirectedPathter}
  =
  \Internals_{\UPATH}\np{\UndirectedPathbis} \cup
  \na{\Head{\Projection}\np{\UndirectedPathbis}} \cup
  \Internals_{\UPATH}\np{\UndirectedPathter} 
  =
  \Internals_{\UPATH}\np{\UndirectedPathbis} \cup
  \na{\Tail{\Projection}\np{\UndirectedPathter}} \cup
  \Internals_{\UPATH}\np{\UndirectedPathter} 
  % \Internals\np{\pathbis \ltimes \pathter}=
  % \Internals\np{\pathbis} \cup \na{\Head{\Projection}\np{\pathbis}} \cup \Internals\np{\pathter}
  % =\Internals\np{\pathbis} \cup \na{\Tail{\Projection}\np{\pathter}} \cup \Internals\np{\pathter}
  \eqfinp
  \label{eq:Internals_PATH_concatenation}
\end{equation}
\fi



\subsubsubsection{More on deployment in paths}\quad

\noindent $\bullet$
For any binary relation \( \relation \subset \VERTEX\times\VERTEX \),
the subset \( \DeploymentInUPaths{\relation}{\OrientedGraph} \subset \UPATH\np{\OrientedGraph} \) 
is made of \undirectedEdgePaths\ (relative to the graph~$\npOrientedGraph$)
whose ordered endpoints (head and tail)
satisfy the relation~$\relation$.
The mapping $\relation \mapsto \DeploymentInUPaths{\relation}{\OrientedGraph} $ is nondecreasing
\wrt\ the inclusion order, as it satisfies
% \begin{subequations}
\begin{equation}
  \forall \relation \subset \VERTEX{\times}\VERTEX 
  \eqsepv 
  \forall \relationbis \subset \VERTEX{\times}\VERTEX 
  \eqsepv 
  \DeploymentInUPaths{\relation \cup \relationbis}{\OrientedGraph}
  =
  \DeploymentInUPaths{\relation}{\OrientedGraph}
  \cup 
  \DeploymentInUPaths{\relationbis}{\OrientedGraph}
  \eqfinp
  \label{eq:DeploymentInPaths_property_union}
\end{equation}
\medskip

\noindent $\bullet$ 
By the very definition~\eqref{eq:DeploymentInUPaths} of the deployment in \undirectedEdgePaths, 
and by definition~\eqref{eq:concatenation_SubsetPath} of the concatenation
between subsets, we have the inclusion
\begin{equation}
  \forall \relation \subset \VERTEX{\times}\VERTEX 
  \eqsepv 
  \forall \relationbis \subset \VERTEX{\times}\VERTEX 
  \eqsepv 
  \DeploymentInUPaths{\relation}{\OrientedGraph}
  \ltimes 
  \DeploymentInUPaths{\relationbis}{\OrientedGraph}
  \subset
  \DeploymentInUPaths{\relation\relationbis}{\OrientedGraph}
  \eqfinp
  \label{eq:DeploymentInPaths_property_composition}
\end{equation}  
% \end{subequations}
We use the following notations
\begin{subequations}
  \begin{align}
    \DeploymentInUPaths{\relation}{\OrientedGraph}^n 
    &=
      \overbrace{\DeploymentInUPaths{\relation}{\OrientedGraph} \ltimes \cdots \ltimes
      \DeploymentInUPaths{\relation}{\OrientedGraph}}^{\text{$n$ times}}
      \eqfinv 
    \\
    \TransitiveClosure{\DeploymentInUPaths{\relation}{\OrientedGraph}} 
    &=
      \bigcup_{n \ge 1} \DeploymentInUPaths{\relation}{\OrientedGraph}^n
      \eqfinp
  \end{align}
\end{subequations}

\subsubsubsection{Endpoint relation in \undirectedEdgePaths}\quad

\noindent $\bullet$ 
With any subset \( \SubsetPath \subset \UPATH\np{\OrientedGraph} \) of
\undirectedEdgePaths, we associate the following binary
relation~\( \BoundaryU{\SubsetPath}{\OrientedGraph} \) on~$\VERTEX$, 
that we call \emph{endpoint relation}, %\emph{boundary relation}, 
% \emph{endpoint relation} \( \Endpoint \subset \VERTEX\times\VERTEX \) 
defined by 
\begin{equation}
  \forall \SubsetPath \subset \UPATH\np{\OrientedGraph}
  \eqsepv 
  \BoundaryU{\SubsetPath}{\OrientedGraph} = \Projection\np{\SubsetPath}
  = \bset{ \Projection\np{\UndirectedPath} }{ \UndirectedPath\in \SubsetPath} \subset \VERTEX\times\VERTEX 
  % = \bset{ \Projection\bp{\np{\path,\orient}} }{ \np{\path,\orient}\in \SubsetPath} \subset \VERTEX\times\VERTEX 
  \eqfinv
  \label{eq:BoundaryU}
\end{equation}
where the projection~\( \Projection \) has been defined
in~\eqref{eq:Projection}. 
% \label{de:endpoint_relation}

The subset \( \BoundaryU{\SubsetPath}{\OrientedGraph} \subset \VERTEX\times\VERTEX \) 
is the subset of couples of vertices which are ordered endpoints (head and tail)
of an \undirectedEdgePath\ contained in~$\SubsetPath$.
By construction, the mapping $\SubsetPath
\mapsto\BoundaryU{\SubsetPath}{\OrientedGraph}$ is nondecreasing \wrt\
the inclusion order, and takes values that are binary relations included in
the binary relation~\( \npTransitiveClosure{\EDGE\cup \Converse{\EDGE}} \), that
is,
\begin{equation}
  \forall \SubsetPath \subset \UPATH\np{\OrientedGraph}
  \eqsepv 
  \BoundaryU{\SubsetPath}{\OrientedGraph} \subset
  \npTransitiveClosure{\EDGE\cup \Converse{\EDGE}} 
  \eqfinp
\end{equation}

\noindent $\bullet$
Joint properties between boundary relation~$\BoundaryU{\cdot}{\OrientedGraph}$
and deployment in \undirectedEdgePaths~\( \DeploymentInUPaths{\cdot}{\OrientedGraph} \)
are as follows:
\begin{subequations}
  \label{boundary-deployment-properties}
  \begin{align}
    \bcBoundaryU{\DeploymentInUPaths{\relation}{\OrientedGraph}}{\OrientedGraph}
    & \subset \relation
      \eqsepv \forall \relation \subset \VERTEX{\times}\VERTEX
    \\
    \bcDeploymentInUPaths{\BoundaryU{\SubsetPath}{\OrientedGraph}}{\OrientedGraph}
    & \supset
      \SubsetPath
      \eqsepv \forall \SubsetPath \subset \UPATH\np{\OrientedGraph}
      \eqfinp
  \end{align}
\end{subequations}



\subsubsection{Lemmas}
\label{appendix_Lemmas_undirectedEdgePaths}

We develop useful links between active \undirectedEdgePaths\
  (see Definition~\ref{de:ActivePaths})
and binary relations between ordered endpoints (head and tail) of the path.
%
In the next lemma, we
give a first insight in that direction for \undirectedEdgePaths\ of length~$2$.

\begin{lemma}
  \label{le:active-length-2}
  Let \( \npOrientedGraph \) be a graph, 
  and $\AgentSubsetW\subset\VERTEX$ be a subset of vertices.
  % 
  Let  \(\UndirectedPath=\np{\path,\orient} \in  \ActiveUndirectedPaths{\OrientedGraph}  \) be
  a given active \undirectedEdgePath\ of length~2
  (see Definition~\ref{de:ActivePaths})
  with   $\bgent$ as head endpoint and $\cgent$ as tail endpoint, that is,
  there exists a vertex $\dgent\in\VERTEX$ such that 
  \begin{equation}
    \UndirectedPath = \Bp{ \bp{ \np{\bgent,\dgent},
        \np{\dgent,\cgent} }, \np{\orient_1,\orient_2} }
    \text{ with }
    \np{\orient_1,\orient_2} \in \na{-1,+1}^2
    \eqfinp
    \label{eq:le:active-length-2}    
  \end{equation}
  % (  \( \Projection\np{\path} = \np{\bgent, \cgent} \)).
  Then, one of the following two possibilities holds true:
  \begin{enumerate}
  \item
    the \undirectedEdgePath~$\UndirectedPath$ ends with $\orient_2=+1$ orientation,
    and then %we have that
    either
    \(\bgent  \EDGE \Delta_{\Complementary{\AgentSubsetW}}\EDGE  \cgent\)
    or
    \(\bgent   \Converse{\EDGE} \Delta_{\Complementary{\AgentSubsetW}} \EDGE  \cgent\),
  \item
    the \undirectedEdgePath~$\UndirectedPath$ ends with $\orient_2=-1$ orientation,
    and then %we have that
    either
    \( \bgent\EDGE \Delta_{\TransitiveReflexiveClosureOfSet{\AgentSubsetW}} \Converse{\EDGE}\cgent \)
    or \( \bgent \Converse{\EDGE} \Delta_{\Complementary{\AgentSubsetW}}\Converse{\EDGE} \cgent\).
  \end{enumerate}
\end{lemma}


\begin{proof}
  As \(\UndirectedPath=\np{\path,\orient} \) in~\eqref{eq:le:active-length-2}
  belongs to~\( \ActiveUndirectedPaths{\OrientedGraph} \) --- 
  hence to \( \UPATH\np{\OrientedGraph} \),
  the set of finite \undirectedEdgePaths\ in~\eqref{eq:undirectedEdgePaths} --- 
  we have that %  \(  \UndirectedPath = \Bp{ \bp{ \np{\bgent,\dgent},
  % \np{\dgent,\cgent} }, \np{\orient_1,\orient_2} } \)
  % where 
  \( \np{\bgent,\dgent} \in \EDGE \) if $\orient_1={+1}$,
  \( \np{\bgent,\dgent} \in \Converse{\EDGE} \) if $\orient_1={-1}$,
  \( \np{\dgent,\cgent} \in \EDGE \) if $\orient_2={+1}$,
  and \( \np{\dgent,\cgent} \in \Converse{\EDGE} \) if $\orient_2={-1}$.
  Now, we consider the four conditions enumerated in Definition~\ref{de:ActivePaths} which must be satisfied
  for the \undirectedEdgePath~$\UndirectedPath$ to be active and which impose constraints on the
  vertex~$\dgent$ according to the possible orientations.

  \begin{enumerate}
  \item
    First, we consider the case when the \undirectedEdgePath~$\UndirectedPath$
    ends with $+1$ orientation, that is, when $\orient_2=+1$.
    \begin{itemize}
    \item
      Item~\ref{it:ActivePaths_case1} in Definition~\ref{de:ActivePaths}
      corresponds to 
      $\orient_1 = +1$, $\orient_{2}= +1$ and $\dgent \in
      \Complementary{\AgentSubsetW}$, which gives that 
      $\bgent \EDGE^{(+1)} \Delta_{\Complementary{\AgentSubsetW}}\dgent$ and $\dgent\Delta_{\Complementary{\AgentSubsetW}}
      \EDGE^{(+1)}\cgent$. Hence, by composition of binary relations, we get that
      $\bgent \EDGE \Delta_{\Complementary{\AgentSubsetW}} \EDGE\cgent$,
      using the property \( \Delta_{\Complementary{\AgentSubsetW}}\Delta_{\Complementary{\AgentSubsetW}}=\Delta_{\Complementary{\AgentSubsetW}}\).
    \item
      Item~\ref{it:ActivePaths_case3} in Definition~\ref{de:ActivePaths} corresponds to 
      $\orient_1 = -1$, $\orient_{2}= +1$ and $\dgent \in
      \Complementary{\AgentSubsetW}$, which gives that 
      $\bgent \EDGE^{(-1)} \Delta_{\Complementary{\AgentSubsetW}}\dgent$ and $\dgent\Delta_{\Complementary{\AgentSubsetW}}
      \EDGE^{(+1)}\cgent$. Hence, we get that 
      $\bgent \Converse\EDGE \Delta_{\Complementary{\AgentSubsetW}} \EDGE\cgent$.
    \end{itemize}
  \item
    Second, we consider the case when the \undirectedEdgePath~$\UndirectedPath$ ends with $-1$ orientation,
    that is, when $\orient_2=-1$:
    \begin{itemize}
    \item
      Item~\ref{it:ActivePaths_case2} in Definition~\ref{de:ActivePaths} corresponds to 
     $\orient_1 = -1$, $\orient_{2}= -1$ and $\dgent \in \Complementary{\AgentSubsetW}$, which gives that 
       $\bgent \EDGE^{(-1)} \Delta_{\Complementary{\AgentSubsetW}}\dgent$ and $\dgent\Delta_{\Complementary{\AgentSubsetW}}
      \EDGE^{(-1)}\cgent$. Hence, we get that 
      $\bgent \Converse\EDGE \Delta_{\Complementary{\AgentSubsetW}}
      \Converse\EDGE\cgent$,
    \item
      Item~\ref{it:ActivePaths_case4} in Definition~\ref{de:ActivePaths} corresponds to 
      $\orient_1 = +1$, $\orient_{2}= -1$ and $\dgent \in \TransitiveReflexiveClosureOfSet{\AgentSubsetW}$, which gives that 
       $\bgent \EDGE^{(+1)}
       \Delta_{\TransitiveReflexiveClosureOfSet{\AgentSubsetW}}\dgent$
       and $\dgent\Delta_{\TransitiveReflexiveClosureOfSet{\AgentSubsetW}} \EDGE^{(-1)}\cgent$. Hence, we get that 
      $\bgent \EDGE \Delta_{\TransitiveReflexiveClosureOfSet{\AgentSubsetW}}
      \Converse\EDGE\cgent$,
      using the property \(
      \Delta_{\TransitiveReflexiveClosureOfSet{\AgentSubsetW}}\Delta_{\TransitiveReflexiveClosureOfSet{\AgentSubsetW}}
      = \Delta_{\TransitiveReflexiveClosureOfSet{\AgentSubsetW}} \). 
    \end{itemize}
    % 
  \end{enumerate}

  This ends the proof.
\end{proof}

Now, we develop the machinery to analyze active \undirectedEdgePaths\ 
by considering decomposition into subpaths and junctions when
reconcatenating.
For this purpose, we need notation. 

% \noindent $\bullet$ 
We denote by \( \Tail{\ProjectionUndirectedPath}: \UPATH\np{\OrientedGraph} \to \UPATH^{1}\np{\OrientedGraph} \) 
(resp. \( \Head{\ProjectionUndirectedPath}: \UPATH\np{\OrientedGraph} \to \UPATH^{1}\np{\OrientedGraph} \)
the projection on the tail (resp. head) endpoint of an \undirectedEdgePath,
defined, for \(\np{\path,\orient}\in\UPATH^n\np{\OrientedGraph}\) and $n\ge 1$, by 
\begin{subequations}
  \label{eq:ProjectionUndirectedPath}
  \begin{align}
    \Tail{\ProjectionUndirectedPath}\bp{\np{\path,\orient}}
    &= 
      \Tail{\ProjectionUndirectedPath}
      \bp{ \nseqa{\tail{\vertex_i},\head{\vertex_i}}{i\in\ic{1,n}},
      \nseqa{\orient_i}{i\in\ic{1,n}} }
      = \bp{\np{\tail{\vertex_1},\head{\vertex_1}},\orient_1 } 
      \eqfinv
    \\
    \Head{\ProjectionUndirectedPath}\bp{\np{\path,\orient}}
    &= 
      \Head{\ProjectionUndirectedPath}
      \bp{ \nseqa{\tail{\vertex_i},\head{\vertex_i}}{i\in\ic{1,n}},
      \nseqa{\orient_i}{i\in\ic{1,n}} }
      = \bp{\np{\tail{\vertex_n},\head{\vertex_n}},\orient_n } 
      \eqfinp
  \end{align}
\end{subequations}

The following Lemma is a straightforward consequence of the definitions
of active \undirectedEdgePaths\ in~\( \ActiveUndirectedPaths{\OrientedGraph} \)
(see Definition~\ref{de:ActivePaths}),
of concatenation~$\ltimes$ in~\eqref{eq:concatenation_UndirectedPath}
and of the projection mappings~\(\Tail{\ProjectionUndirectedPath}\) and 
\( \Head{\ProjectionUndirectedPath} \) in~\eqref{eq:ProjectionUndirectedPath}.
The proof is left to the reader. 

\begin{lemma}
  \label{lem:ActiveUndirectedPaths_subpaths_junctions_when_reconcatenating}
  Let \( \npOrientedGraph \) be a graph, 
  and $\AgentSubsetW\subset\VERTEX$ be a subset of vertices.
  Let \(\UndirectedPath \in \UPATH\np{\OrientedGraph} \)
  be an  \undirectedEdgePath\ of length~\( n \geq 2 \). We have that
  \begin{equation}
    \UndirectedPath \in \ActiveUndirectedPaths{\OrientedGraph}
    \iff
    \exists \UndirectedPathbis, \UndirectedPathter
    \in \ActiveUndirectedPaths{\OrientedGraph}
    \eqsepv 
    \UndirectedPath = \UndirectedPathbis \ltimes \UndirectedPathter
    \eqsepv
    \Head{\ProjectionUndirectedPath}\np{\UndirectedPathbis}
    \ltimes
    \Tail{\ProjectionUndirectedPath}\np{\UndirectedPathter}
    \in \ActiveUndirectedPaths{\OrientedGraph}
    \eqfinp
    \label{eq:ActiveUndirectedPaths_subpaths_junctions_when_reconcatenating}    
  \end{equation}
\end{lemma}



\subsection{Material on binary relations}
\label{Material_on_binary_relations}

In~\S\ref{appendix_Lemmas_binary_relations},
we prove Lemmas on the binary relations of
Definition~\ref{de:all_the_relations}.
In~\S\ref{appendix_Definitions_binary_relations},
we define new binary relations and we establish properties.

\coqstart The Coq proofs follows the mathematical proofs with the help of a developed library encoding most of usefull 
binary relations and needed operators. As relations are coded as sets the library we have developed is based on the 
mathcomp implementation of sets \texttt{classical\_sets.v} and contains a ssreflect reimplementation of transitive (reflexive) 
closures of relation that are found in the Coq standard library.
Moreover, as proofs use intensively associative and commutative properties of union and intersection of relations 
we use the \texttt{AAC\_tactics} to ease equality proofs between long expressions with relations containing unions intersections 
composition and diagonal relations. For this to be possible, some specific propeties of relations with respect to \texttt{AAC} 
are to be listed and proved in Coq. This is done in \texttt{aacset.v}.\coqend

\subsubsection{Lemmas on the binary relations of Definition~\ref{de:all_the_relations}}
\label{appendix_Lemmas_binary_relations}

We refer the reader to Definition~\ref{de:all_the_relations}
for the definitions of basic binary relations. 
\coqstart As already explained the Coq proofs closely follow the mathematical proofs given here and we will 
give for each lemma its Coq name. \coqend

\begin{lemma}[\texttt{L7}]
  \label{le:L7}
  We have that
  \begin{subequations}
    \begin{align}
      \ConverseConditionalAncestor \ConditionalAncestor
      &=
        {\Delta 
        \cup
        \Delta_{\Complementary{\AgentSubsetW}} \ConditionalAscendent 
        \cup \ConverseConditionalAscendent \Delta_{\Complementary{\AgentSubsetW}}
        \cup \ConditionalCommonCause
        }
        \eqfinv
        \label{eq:AA}
      \\
      \Cousinhood \ConverseConditionalAncestor \ConditionalAncestor
      &=
        \Cousinhood \bp{
        \Delta \cup
        \ConverseConditionalAscendent
        \Delta_{{\Complementary{\AgentSubsetW}}}
        \cup \ConditionalCommonCause}
        \eqfinv
        \label{eq:CEmEsubset}
     \\
      \Cousinhood \ConverseConditionalAncestor \ConditionalAncestor \Cousinhood
      &= \Cousinhood
        \eqfinv
        \label{eq:gammaAAgamma}
      \\
      \Cousinhood \ConverseConditionalAncestor \ConditionalAncestor
      \Delta_{\Complementary{\AgentSubsetW}}
      &= \Cousinhood \bp{
        \ConverseConditionalAscendent 
        \cup \ConditionalCommonCause}
        \Delta_{\Complementary{\AgentSubsetW}}
        \eqfinv
        \label{eq:gammaAADelta}
      \\
      \Delta_{\Complementary{\AgentSubsetW}} \ConverseConditionalAncestor \ConditionalAncestor    \Cousinhood 
      &=
        \Delta_{\Complementary{\AgentSubsetW}}
        \bp{
        \ConditionalAscendent 
        \cup \ConditionalCommonCause}
        \Cousinhood
        \eqfinp
        \label{eq:DeltaAAgamma}
     \end{align}
  \end{subequations}
\end{lemma}

\coqstart We give here, as an example, the Lemma~\ref{le:L7} as written in Coq to 
{\small \inputsnippets{paper_csbr/L7_csbr-n.tex}} \coqend.

\begin{proof}\quad

  \noindent $\bullet$
  We prove Equation~\eqref{eq:AA} (\texttt{ZZZ}) as follows:
  \begin{align*}
    \ConverseConditionalAncestor \ConditionalAncestor
    &= \bp{\Delta \cup  \TransitiveClosureConverseParentalPrecedence}
      \bp{\Delta \cup \TransitiveClosureParentalPrecedence}
      \intertext{as \( \ConditionalAncestor=\Delta \cup \TransitiveClosureParentalPrecedence\)
      by definition of the reflexive and transitive closure of a binary relation}
    % \\
    &= \Delta \cup \TransitiveClosureParentalPrecedence \cup  \TransitiveClosureConverseParentalPrecedence \cup
      \TransitiveClosureConverseParentalPrecedence  \TransitiveClosureParentalPrecedence
      \tag{by developing}
    \\
    &= \Delta \cup \TransitiveClosureParentalPrecedence \cup  \TransitiveClosureConverseParentalPrecedence \cup
      \ConditionalCommonCause 
      \tag{by definition of $\ConditionalCommonCause$ in~\eqref{eq:common_cause}}
    \\
    &= \Delta \cup
      \bp{\Delta_{\Complementary{\AgentSubsetW}}\ConditionalDown}
      \cup
      \bp{ \ConditionalUp \Delta_{\Complementary{\AgentSubsetW}}}
      \cup
      \ConditionalCommonCause 
      \intertext{as $\TransitiveClosure{\relation}=
      \relation\TransitiveReflexiveClosure{\relation}
      = \TransitiveReflexiveClosure{\relation}\relation $
      for any binary relation~$\relation$, used here with
      \( \relation= \Delta_{\Complementary\AgentSubsetW}\Precedence \)
      and with     \( \relation= \Converse{\Precedence}\Delta_{\Complementary\AgentSubsetW} \)}
    % \\
    &=
      \Delta 
      \cup
      \Delta_{\Complementary{\AgentSubsetW}} \ConditionalAscendent 
      \cup \ConverseConditionalAscendent \Delta_{\Complementary{\AgentSubsetW}}
      \cup \ConditionalCommonCause
      \eqfinp
      \tag{by definition of $\ConditionalAscendent$ in~\ref{eq:converse_conditional_parental_relation}}
  \end{align*}
  

  \noindent $\bullet$
  We prove Equation~\eqref{eq:CEmEsubset} as follows:
  \begin{align*}
    \Cousinhood \ConverseConditionalAncestor \ConditionalAncestor
    &=
      \Cousinhood \Delta_{\AgentSubsetW}
      \ConverseConditionalAncestor \ConditionalAncestor
      \intertext{as $\Cousinhood$ ends with $\Delta_{\AgentSubsetW}$
      in~\eqref{eq:Cousinhood}, hence \( \Cousinhood=\Cousinhood\Delta_{\AgentSubsetW}\)}
%    \\
    &=
      \Cousinhood \Delta_{\AgentSubsetW}
      \bp{\Delta 
      \cup
      \Delta_{\Complementary{\AgentSubsetW}} \ConditionalAscendent 
      \cup \ConverseConditionalAscendent \Delta_{\Complementary{\AgentSubsetW}}
      \cup \ConditionalCommonCause}
      \tag{by~\eqref{eq:AA}}
    \\
    &=
      \Cousinhood 
      \bp{ \Delta_{\AgentSubsetW}\Delta
      \cup
      \Delta_{\AgentSubsetW}\Delta_{\Complementary{\AgentSubsetW}} \ConditionalAscendent
      \cup
      \Delta_{\AgentSubsetW}\ConverseConditionalAscendent \Delta_{\Complementary{\AgentSubsetW}}
      \cup
      \Delta_{\AgentSubsetW}\ConditionalCommonCause}
      \tag{by developing}
    \\
    &=
      \Cousinhood \Delta_{\AgentSubsetW}
      \bp{\Delta \cup
      \ConverseConditionalAscendent \Delta_{\Complementary{\AgentSubsetW}}
      \cup \ConditionalCommonCause}
      \tag{as $\Delta_{\AgentSubsetW}\Delta_{\Complementary{\AgentSubsetW}}=\emptyset$}
    \\
    &=
      \Cousinhood \bp{
      \Delta \cup
      \ConverseConditionalAscendent \Delta_{\Complementary{\AgentSubsetW}}
      \cup \ConditionalCommonCause}
      \tag{as \( \Cousinhood\Delta_{\AgentSubsetW}=\Cousinhood\)}
      \eqfinp
  \end{align*}

  \noindent $\bullet$
  We prove Equation~\eqref{eq:gammaAAgamma} as follows:
  \begin{align*}
    \Cousinhood \ConverseConditionalAncestor \ConditionalAncestor \Cousinhood
    &=
      \Cousinhood \bp{
      \Delta \cup
      \ConverseConditionalAscendent \Delta_{\Complementary{\AgentSubsetW}}
      \cup \ConditionalCommonCause} \Cousinhood
      \tag{by~\eqref{eq:CEmEsubset}}
     \\
    &=
      \Cousinhood\Delta_{\AgentSubsetW} \bp{
      \Delta \cup
      \ConverseConditionalAscendent \Delta_{\Complementary{\AgentSubsetW}}
      \cup \ConditionalCommonCause} \Delta_{\AgentSubsetW} \Cousinhood
      \intertext{as $\Cousinhood$ begins and ends with $\Delta_{\AgentSubsetW}$
      in~\eqref{eq:Cousinhood}, hence \( \Cousinhood=\Cousinhood\Delta_{\AgentSubsetW}=\Delta_{\AgentSubsetW}\Cousinhood\)}
%      \\
    % &=
    %   \Cousinhood \Delta_{\AgentSubsetW}\ConverseConditionalAncestor \ConditionalAncestor\Delta_{\AgentSubsetW} \Cousinhood
    %   \tag{as $\Cousinhood$ begins and ends with $\Delta_{\AgentSubsetW}$ in~\eqref{eq:Cousinhood}}
    % \\
    % &=\Cousinhood  \Delta_{\AgentSubsetW}
    %   \bp{\Delta 
    %   \cup
    %   \Delta_{\Complementary{\AgentSubsetW}} \ConditionalAscendent 
    %   \cup \ConverseConditionalAscendent \Delta_{\Complementary{\AgentSubsetW}}
    %   \cup \ConditionalCommonCause
    %   }
    %   \Delta_{\AgentSubsetW}
    %   \Cousinhood
    %   \tag{by~\eqref{eq:AA}}
    % \\
    &=
      \Cousinhood  \Delta_{\AgentSubsetW}
      \bp{\Delta 
      \cup \ConditionalCommonCause
      }
      \Delta_{\AgentSubsetW}
      \Cousinhood
      \tag{as %$\Delta_{\AgentSubsetW}\Delta_{\Complementary{\AgentSubsetW}}=$
      $\Delta_{\Complementary{\AgentSubsetW}}\Delta_{\AgentSubsetW}=\emptyset$}
    \\
    &=\Cousinhood \bp{\Delta_{\AgentSubsetW}
      \cup \Delta_{\AgentSubsetW} \ConditionalCommonCause\Delta_{\AgentSubsetW}}
      \Cousinhood
      \tag{by developing}
    \\
    &= \bp{\Delta_{\AgentSubsetW} \cup \bpTransitiveClosure{\Delta_{\AgentSubsetW} \ConditionalCommonCause
      \Delta_{\AgentSubsetW} } }
      \bp{\Delta_{\AgentSubsetW} \cup \Delta_{\AgentSubsetW} \ConditionalCommonCause\Delta_{\AgentSubsetW}}
      \bp{ \Delta_{\AgentSubsetW} \cup \bpTransitiveClosure{\Delta_{\AgentSubsetW} \ConditionalCommonCause
      \Delta_{\AgentSubsetW} } }
      \tag{by definition~\eqref{eq:Cousinhood} of $\Cousinhood$}
    \\
    &=
      \bp{ \Delta_{\AgentSubsetW} \cup \bpTransitiveClosure{\Delta_{\AgentSubsetW} \ConditionalCommonCause
      \Delta_{\AgentSubsetW} } }
      \bp{ \Delta_{\AgentSubsetW} \cup \bpTransitiveClosure{\Delta_{\AgentSubsetW} \ConditionalCommonCause
      \Delta_{\AgentSubsetW} } }
      \intertext{as $(\Delta_{\AgentSubsetW} \cup \TransitiveClosure{\relation})
      (\Delta_{\AgentSubsetW} \cup \relation)=
      \Delta_{\AgentSubsetW} \cup
      \TransitiveClosure{\relation}\Delta_{\AgentSubsetW} \cup
      \Delta_{\AgentSubsetW}\relation \cup \TransitiveClosure{\relation}\relation =
      \Delta_{\AgentSubsetW} \cup
      \TransitiveClosure{\relation}
      \cup \TransitiveClosure{\relation}\relation =
      \Delta_{\AgentSubsetW} \cup \TransitiveClosure{\relation}$ whenever 
      $\relation\Delta_{\AgentSubsetW}=
      \Delta_{\AgentSubsetW}\relation=\relation$,
      which is the case for \( \relation=\Delta_{\AgentSubsetW} \ConditionalCommonCause
      \Delta_{\AgentSubsetW} \)}
    % \\
    &=
      \Delta_{\AgentSubsetW} \cup \bpTransitiveClosure{\Delta_{\AgentSubsetW} \ConditionalCommonCause
      \Delta_{\AgentSubsetW} }
      \tag{by developing}
    \\
    &=
      \Cousinhood
      \eqfinp
      \tag{by definition~\eqref{eq:Cousinhood} of $\Cousinhood$}
  \end{align*}
  
  \noindent $\bullet$
  We prove Equation~\eqref{eq:gammaAADelta} as follows:
  \begin{align*}
    \Cousinhood \ConverseConditionalAncestor \ConditionalAncestor
    \Delta_{\Complementary{\AgentSubsetW}}
    &=
      \Cousinhood \bp{
      \Delta \cup
      \ConverseConditionalAscendent \Delta_{\Complementary{\AgentSubsetW}}
      \cup \ConditionalCommonCause} \Delta_{\Complementary{\AgentSubsetW}}
      \tag{by~\eqref{eq:CEmEsubset}}
    \\
    &=
      \Cousinhood \Delta_{\AgentSubsetW}\bp{
      \Delta \cup
      \ConverseConditionalAscendent \Delta_{\Complementary{\AgentSubsetW}}
      \cup \ConditionalCommonCause} \Delta_{\Complementary{\AgentSubsetW}}
      \intertext{as $\Cousinhood$ ends with $\Delta_{\AgentSubsetW}$
      in~\eqref{eq:Cousinhood}, hence \( \Cousinhood=\Cousinhood\Delta_{\AgentSubsetW}\)}
%    
   % &=
   %    \Cousinhood \Delta_{\AgentSubsetW}
   %    \ConverseConditionalAncestor \ConditionalAncestor
   %    \Delta_{\Complementary{\AgentSubsetW}}
   %    \tag{as $\Cousinhood$ ends with $\Delta_{\AgentSubsetW}$ in~\eqref{eq:Cousinhood}}
   %  \\
   %  &=
   %    \Cousinhood \Delta_{\AgentSubsetW}
   %    \bp{\Delta 
   %    \cup
   %    \Delta_{\Complementary{\AgentSubsetW}} \ConditionalAscendent 
   %    \cup \ConverseConditionalAscendent \Delta_{\Complementary{\AgentSubsetW}}
   %    \cup \ConditionalCommonCause}
   %    \Delta_{\Complementary{\AgentSubsetW}}
   %    \tag{by~\eqref{eq:AA}}
   %  \\
    % &=
    %   \Cousinhood 
    %   \bp{ \Delta_{\AgentSubsetW}\Delta\Delta_{\Complementary{\AgentSubsetW}} 
    %   \cup
    %   \Delta_{\AgentSubsetW}\Delta_{\Complementary{\AgentSubsetW}} \ConditionalAscendent\Delta_{\Complementary{\AgentSubsetW}}
    %   \cup
    %   \Delta_{\AgentSubsetW}\ConverseConditionalAscendent \Delta_{\Complementary{\AgentSubsetW}}\Delta_{\Complementary{\AgentSubsetW}}
    %   \cup
    %   \Delta_{\AgentSubsetW}\ConditionalCommonCause\Delta_{\Complementary{\AgentSubsetW}} }
    % \\
    &=
      \Cousinhood \Delta_{\AgentSubsetW}
      \bp{ \ConverseConditionalAscendent
      \cup \ConditionalCommonCause}
      \Delta_{\Complementary{\AgentSubsetW}}
      \tag{as $\Delta_{\AgentSubsetW}\Delta_{\Complementary{\AgentSubsetW}}=\emptyset$}
    \\
    &=
      \Cousinhood \bp{
      \ConverseConditionalAscendent 
      \cup \ConditionalCommonCause}
      \Delta_{\Complementary{\AgentSubsetW}}
      \eqfinp
      \tag{as \( \Cousinhood=\Cousinhood\Delta_{\AgentSubsetW}\)}
  \end{align*}
  \noindent $\bullet$
  Equation~\eqref{eq:DeltaAAgamma} is obtained from Equation~\eqref{eq:gammaAADelta}
  by switching to converse relation.
  \medskip
  
  This ends the proof. 
\end{proof}


\subsubsection{Definitions and properties of new binary relations}
\label{appendix_Definitions_binary_relations}

We refer the reader to Definition~\ref{de:all_the_relations}
for the definitions of basic binary relations.
% 
We add two new ones, where we recall that
\(    \TransitiveReflexiveClosureOfSet{\AgentSubsetW}=
\TransitiveReflexiveClosure{\EDGE}\AgentSubsetW \) by~\eqref{eq:TransitiveReflexiveClosure_AgentSubsetW}:
% \begin{subequations}
\begin{align}
  \CousinhoodStar
  &=
    \bpTransitiveClosure{ \Delta_{\TransitiveReflexiveClosureOfSet{\AgentSubsetW}}
    \ConditionalCommonCause \Delta_{\TransitiveReflexiveClosureOfSet{\AgentSubsetW}} }
    \cup 
    \Delta_{\TransitiveReflexiveClosureOfSet{\AgentSubsetW}}
    \eqfinv
    \label{eq:CousinhoodStar}
  \\
  \ConditionalActiveStar
  &=
    \Delta \cup 
    \ConditionalAscendent \cup \ConverseConditionalAscendent \cup \ConditionalCommonCause
    \cup
    \bp{\ConditionalAscendent \cup \ConditionalCommonCause}
    % \bp{\ConverseConditionalAscendent \cup \ConverseConditionalCommonCause}\RelTheta
    \CousinhoodStar
    \bp{\ConverseConditionalAscendent \cup \ConverseConditionalCommonCause}
    \eqfinp
    \label{eq:conditional_active_relationStar}
\end{align}
% \end{subequations}
Notice that the \emph{star conditional cousinhood relation}~\( \CousinhoodStar \) in~\eqref{eq:CousinhoodStar} is
the relation~\( \Cousinhood \) in~\eqref{eq:Cousinhood} with
\( \AgentSubsetW \) replaced by~\(
\TransitiveReflexiveClosureOfSet{\AgentSubsetW} \),
and the \emph{star conditional active relation}~$\ConditionalActiveStar$ 
in~\eqref{eq:conditional_active_relationStar} is
the relation~\( \ConditionalActive \) in~\eqref{eq:conditional_active_relation} with
\( \Cousinhood \) replaced by~\( \CousinhoodStar \).




\begin{lemma}
  The two following relations
  \begin{subequations}
    \begin{align}
      \ConditionalActivePlus
      &=
        \ConditionalAscendent \cup \ConditionalCommonCause
        \cup 
        \Bp{\bp{\ConditionalAscendent \cup \ConditionalCommonCause}
        \CousinhoodStar % \bp{
        % \np{\Delta_{\TransitiveReflexiveClosureOfSet{\AgentSubsetW}} \ConditionalCommonCause \Delta_{\TransitiveReflexiveClosureOfSet{\AgentSubsetW}}}^{+}
        % \cup
        % \Delta_{\TransitiveReflexiveClosureOfSet{\AgentSubsetW}}
        % }
        {\ConditionalCommonCause}}
        \eqfinv
        \label{eq:ConditionalActivePlus}
      \\
      \ConditionalActiveMinus
      &= \ConverseConditionalAscendent \cup
        \Bp{
        \bp{\ConditionalAscendent \cup \ConditionalCommonCause}
        \CousinhoodStar %      \bp{
        % \np{\Delta_{\TransitiveReflexiveClosureOfSet{\AgentSubsetW}} \ConditionalCommonCause \Delta_{\TransitiveReflexiveClosureOfSet{\AgentSubsetW}}}^{+}
        % \cup
        % \Delta_{\TransitiveReflexiveClosureOfSet{\AgentSubsetW}}
        % }
        \ConverseConditionalAscendent
        }
        \eqfinv
        \label{eq:ConditionalActiveMinus}      
    \end{align}
  \end{subequations}
  satisfy the following properties
  \begin{subequations}
    \begin{align}
      \ConditionalActiveStar
      &=
        \Delta \cup \ConditionalActivePlus \cup \ConditionalActiveMinus
        \eqfinv
        \label{eq:ConditionalActivePlus_cup_Minus}      
      \\
      \ConditionalActivePlus
      & \supset 
        \ConditionalActiveMinus \Delta_{\Complementary{\AgentSubsetW}}
        {\Precedence}
        \eqfinv
        \label{eq:ConditionalActivePlus_supset}
      \\
      \ConditionalActiveMinus
      & \supset 
        \ConditionalActivePlus \Delta_{\TransitiveReflexiveClosureOfSet{\AgentSubsetW}}
        \Converse{\Precedence}
        \eqfinp 
        \label{eq:ConditionalActiveMinus_supset}        
    \end{align}
  \end{subequations}  
\end{lemma}

\begin{proof}
  \quad

  \noindent $\bullet$
  We prove~\eqref{eq:ConditionalActivePlus_cup_Minus} as follows:
  \begin{align*}
\ConditionalActiveStar
      &=
    \Delta \cup 
    \ConditionalAscendent \cup \ConverseConditionalAscendent \cup \ConditionalCommonCause
    \cup
    \bp{\ConditionalAscendent \cup \ConditionalCommonCause}
    \CousinhoodStar
        \bp{\ConverseConditionalAscendent \cup \ConverseConditionalCommonCause}
        \tag{by definition~\eqref{eq:conditional_active_relationStar} of~$\ConditionalActiveStar$}
\\
      &=
    \Delta \cup 
  \underbrace{  \ConditionalAscendent \cup \ConditionalCommonCause
       \cup 
        \bp{\ConditionalAscendent \cup \ConditionalCommonCause}
        \CousinhoodStar {\ConditionalCommonCause} }_{=\ConditionalActivePlus \textrm{by~\eqref{eq:ConditionalActivePlus}}}
        \cup
 \underbrace{\ConverseConditionalAscendent \cup
        \bp{\ConditionalAscendent \cup \ConditionalCommonCause}
        \CousinhoodStar  \ConverseConditionalAscendent  }_{=\ConditionalActiveMinus \textrm{by~\eqref{eq:ConditionalActiveMinus}}}
\eqfinp               
  \end{align*}

  \noindent $\bullet$  
  We prove~\eqref{eq:ConditionalActivePlus_supset} as follows:     
  \begin{align*}
    \ConditionalActiveMinus\Delta_{\Complementary{\AgentSubsetW}} {\Precedence}
    &=
      \bgp{ \ConverseConditionalAscendent \cup
      \Bp{
      \bp{\ConditionalAscendent \cup \ConditionalCommonCause}
      \CousinhoodStar \ConverseConditionalAscendent } }
      \Delta_{\Complementary{\AgentSubsetW}} {\Precedence}
      \tag{by definition~\eqref{eq:ConditionalActiveMinus} of~$\ConditionalActiveMinus$}
    \\
    &=
      \bgp{ \Delta \cup 
      \Bp{
      \bp{\ConditionalAscendent \cup \ConditionalCommonCause}
      \CousinhoodStar } } 
      \ConverseConditionalAscendent \Delta_{\Complementary{\AgentSubsetW}} {\Precedence}
      \tag{by factorizing $\ConverseConditionalAscendent$}
    \\
    &\subset 
      \bgp{ \Delta \cup 
      \Bp{
      \bp{\ConditionalAscendent \cup \ConditionalCommonCause}
      \CousinhoodStar } }
      \ConverseConditionalAscendent \Delta_{\Complementary{\AgentSubsetW}} 
      \ConditionalAscendent 
      \tag{as \( \Precedence \subset \ConditionalAscendent \)
      by~\eqref{eq:conditional_ascendent_relation} }
    \\
    & =
      \bgp{ \Delta \cup 
      \Bp{
      \bp{\ConditionalAscendent \cup \ConditionalCommonCause}
      \CousinhoodStar } } 
      \ConditionalCommonCause
      \tag{by definition~\eqref{eq:common_cause} of $\ConditionalCommonCause$}
    \\
    & =
      \ConditionalCommonCause \cup 
      \bp{\ConditionalAscendent \cup \ConditionalCommonCause}
      \CousinhoodStar \ConditionalCommonCause 
      \tag{by developing}
    \\        
    &\subset \ConditionalActivePlus
      \eqfinp
      \tag{by definition~\eqref{eq:ConditionalActivePlus} of~$\ConditionalActivePlus$}
      % \label{eq:subsetC+}
  \end{align*}

  \noindent $\bullet$  
  We prove~\eqref{eq:ConditionalActiveMinus_supset} as follows:
  % \jpc{developed as asked by mdl}
  \begin{align*}
    \ConditionalActivePlus \Delta_{\TransitiveReflexiveClosureOfSet{\AgentSubsetW}}
    \Converse{\Precedence}
    &=
      \bgp{  \ConditionalAscendent \cup \ConditionalCommonCause
      \cup 
      \Bp{\bp{\ConditionalAscendent \cup \ConditionalCommonCause}
      \CousinhoodStar
      {\ConditionalCommonCause}} } \Delta_{\TransitiveReflexiveClosureOfSet{\AgentSubsetW}}
      \Converse{\Precedence}
      \tag{by definition~\eqref{eq:ConditionalActivePlus} of~$\ConditionalActivePlus$}
    \\
    &=
      \bp{
      \np{  \ConditionalAscendent \cup \ConditionalCommonCause}
      \Delta_{\TransitiveReflexiveClosureOfSet{\AgentSubsetW}} \Converse{\Precedence}}
      \cup
      \bgp{
      \bp{\ConditionalAscendent \cup \ConditionalCommonCause}
      \CousinhoodStar
      {\ConditionalCommonCause}
      \Delta_{\TransitiveReflexiveClosureOfSet{\AgentSubsetW}} \Converse{\Precedence}}
      \tag{by developing}
      \eqfinp 
  \end{align*}
  We treat each of the two terms in the union separately.
  We are going to show that each term is included in~$\ConditionalActiveMinus$. 
  
  For the first term, we have that
  \begin{align*}
    \np{\ConditionalAscendent \cup \ConditionalCommonCause }
    \Delta_{\TransitiveReflexiveClosureOfSet{\AgentSubsetW}}
    \Converse{\Precedence}
    &\subset
      \bp{\ConditionalAscendent \cup \ConditionalCommonCause}
      \CousinhoodStar
      \ConverseConditionalAscendent
      \intertext{as $\Delta_{\TransitiveReflexiveClosureOfSet{\AgentSubsetW}}
      \subset \CousinhoodStar$ by~\eqref{eq:CousinhoodStar},
      and as  $\Converse{\Precedence} \subset
      \ConverseConditionalAscendent$
      by~\eqref{eq:converse_conditional_ascendent_relation}}
    % \\
    &\subset \ConditionalActiveMinus
      \eqfinp
      \tag{as  $\ConditionalActiveMinus
      = \ConverseConditionalAscendent \cup
      \bp{
      \np{\ConditionalAscendent \cup \ConditionalCommonCause} \CousinhoodStar
      \ConverseConditionalAscendent
      }$ by definition~\eqref{eq:ConditionalActiveMinus} }
  \end{align*}
  For the second term, we have that
  \begin{align*}
    &\bp{\ConditionalAscendent \cup \ConditionalCommonCause}
      \CousinhoodStar
      {\ConditionalCommonCause}
      \Delta_{\TransitiveReflexiveClosureOfSet{\AgentSubsetW}} \Converse{\Precedence}
    \\
    &\hspace{1cm}=
      \bp{\ConditionalAscendent \cup \ConditionalCommonCause}
      \Bp{
      \np{\Delta_{\TransitiveReflexiveClosureOfSet{\AgentSubsetW}} \ConditionalCommonCause \Delta_{\TransitiveReflexiveClosureOfSet{\AgentSubsetW}}}^{+}
      \cup
      \Delta_{\TransitiveReflexiveClosureOfSet{\AgentSubsetW}}
      }
      {\ConditionalCommonCause}
      \Delta_{\TransitiveReflexiveClosureOfSet{\AgentSubsetW}} \Converse{\Precedence}
      \tag{by definition~\eqref{eq:CousinhoodStar} of $\CousinhoodStar$}
    \\
    &\hspace{1cm}=
      \bp{\ConditionalAscendent \cup \ConditionalCommonCause}
      \bp{
      \np{\Delta_{\TransitiveReflexiveClosureOfSet{\AgentSubsetW}} \ConditionalCommonCause \Delta_{\TransitiveReflexiveClosureOfSet{\AgentSubsetW}}}^{+}
      \cup
      \Delta_{\TransitiveReflexiveClosureOfSet{\AgentSubsetW}}
      }
      \bp{\Delta_{\TransitiveReflexiveClosureOfSet{\AgentSubsetW}} {\ConditionalCommonCause}
      \Delta_{\TransitiveReflexiveClosureOfSet{\AgentSubsetW}}
      }
      \Converse{\Precedence}
      \tag{by inserting $\Delta_{\TransitiveReflexiveClosureOfSet{\AgentSubsetW}}$}
    \\
    &\hspace{1cm}\subset
      \bp{\ConditionalAscendent \cup \ConditionalCommonCause}
      \bp{
      \np{\Delta_{\TransitiveReflexiveClosureOfSet{\AgentSubsetW}} \ConditionalCommonCause \Delta_{\TransitiveReflexiveClosureOfSet{\AgentSubsetW}}}^{+}
      \cup
      \Delta_{\TransitiveReflexiveClosureOfSet{\AgentSubsetW}}
      }
      \Converse{\Precedence}
      \intertext{as $ \np{ \TransitiveClosure{\relation} \cup 
      \Delta_{\TransitiveReflexiveClosureOfSet{\AgentSubsetW}} } \relation \subset 
      \TransitiveClosure{\relation} $ for any relation~$\relation$ such that 
      $\relation\Delta_{\TransitiveReflexiveClosureOfSet{\AgentSubsetW}}=
      \Delta_{\TransitiveReflexiveClosureOfSet{\AgentSubsetW}}\relation=\relation$,
      which is the case for \( \relation=\Delta_{\TransitiveReflexiveClosureOfSet{\AgentSubsetW}} \ConditionalCommonCause
      \Delta_{\TransitiveReflexiveClosureOfSet{\AgentSubsetW}} \)}
    % \\
    &\hspace{1cm}=
      \bp{\ConditionalAscendent \cup \ConditionalCommonCause}
      \CousinhoodStar
      \Converse{\Precedence}
      \tag{by definition~\eqref{eq:CousinhoodStar} of $\CousinhoodStar$}
    \\
    &\hspace{1cm}\subset
      \bp{\ConditionalAscendent \cup \ConditionalCommonCause}
      \CousinhoodStar
      \ConverseConditionalAscendent
      \tag{as $ \Converse{\Precedence}
      \subset\ConverseConditionalAscendent$
      by~\eqref{eq:converse_conditional_ascendent_relation}}
    \\
    &\hspace{1cm}\subset \ConditionalActiveMinus
      \eqfinp
      \tag{as  $\ConditionalActiveMinus
      = \ConverseConditionalAscendent \cup
      \bp{
      \np{\ConditionalAscendent \cup \ConditionalCommonCause} \CousinhoodStar
      \ConverseConditionalAscendent
      }$ by definition~\eqref{eq:ConditionalActiveMinus} }
  \end{align*}
  We conclude that 
  $\ConditionalActivePlus
  \Delta_{\TransitiveReflexiveClosureOfSet{\AgentSubsetW}}
  \Converse{\Precedence} \subset \ConditionalActiveMinus$.
  \medskip

  This ends the proof. 
\end{proof}



% \subsubsubsection{$\ConditionalActive =\ConditionalActive_{*}$}

\begin{lemma}
  \label{le:cond-act-equals-cond-act-star}
  We have that
  \begin{equation}
    \ConditionalActive =\ConditionalActiveStar
    \eqfinp
  \end{equation}
\end{lemma}


\begin{proof}
  The proofs is in three steps.
  \medskip

  \noindent $\bullet$
  We prove that
  \begin{align}
    \forall \relation \subset \AGENT\times\AGENT \eqsepv
    \forall \Bgent \subset \AGENT \eqsepv
    \TransitiveReflexiveClosure{\relation}{\Bgent}
    & =
      \npTransitiveReflexiveClosure{\Delta_{\Complementary{\Bgent}}\relation} \Bgent 
      \eqfinp
      \label{eq:R-kleene-G-equals}
  \end{align}
  % for any relation $\relation \subset \VERTEX{\times}\VERTEX$ and any
  % subset $\Bgent \subset \VERTEX$ of vertices we have 
  % \begin{equation}
  %   \TransitiveReflexiveClosure{\relation}{\Bgent}
  %   =  \npTransitiveReflexiveClosure{\Delta_{\Complementary{\Bgent}}\relation}
  %   \Bgent
  %   \eqfinp
  %   \label{eq:foreset-property}
  % \end{equation}
  For this purpose, we prove
  the following induction assumption~\( {\cal H}_n \): 
  for any $n\ge 1$, we have that $\np{\cup_{k=0}^n \relation^k}\Bgent
  =  \bp{ \cup_{k=0}^n \np{\Delta_{\Complementary{\Bgent}} \relation}^k
  }\Bgent$, where we recall the convention $\relation^0=\Delta$.

  As a preliminary result, for any \( \Bgent, \Cgent\subset \AGENT \),
  from the sequence of equalities
  \( \np{\Delta \Bgent} \cup \np{\relation \Cgent}
  = \Bgent \cup \bp{ \np{\relation \Cgent} \backslash \Bgent}
  % \\
  = \Bgent \cup  \np{\Delta_{\Complementary{\Bgent}} \relation \Cgent}
  % \\
  =\np{\Delta \Bgent} \cup  \np{\Delta_{\Complementary{\Bgent}} \relation
    \Cgent} \), we deduce that
  \begin{equation}
    \np{\Delta \Bgent} \cup \np{\relation \Cgent}=
    \np{\Delta \Bgent} \cup  \np{\Delta_{\Complementary{\Bgent}} \relation \Cgent}
    \eqfinp    
    \label{eq:preliminary_result}
    \end{equation}
  Thus, with \( \Cgent=\Bgent \), we obtain that $\np{\cup_{k=0}^1 \relation^k}\Bgent
  =  \bp{ \cup_{k=0}^1 \np{\Delta_{\Complementary{\Bgent}} \relation}^k
  }\Bgent$, that is, assumption~\( {\cal H}_1 \) holds true. 
  
  Now, we suppose that, for a given $n\ge 1$, the induction assumption~\( {\cal H}_n
  \) holds true. Then, we have that 
  \begin{align*}
    \np{\cup_{k=0}^{n+1} \relation^k}\Bgent
    &=
      \np{\Delta \Bgent} \cup
      \Bp{\relation \bp{\np{\cup_{k=0}^n \relation^k}}\Bgent}
      \tag{using the convention $\relation^0=\Delta$}
    \\
    &=
      \np{\Delta \Bgent} \cup
      \Bp{\Delta_{\Complementary{\Bgent}} \relation \bp{\np{\cup_{k=0}^n \relation^k}}\Bgent}
      \intertext{using the preliminary result~\eqref{eq:preliminary_result}
      but with the binary relation \( \relation \bp{\np{\cup_{k=0}^n
      \relation^k}} \)}
 %   \\
    &=
      \np{\Delta \Bgent} \cup
      \bgp{\Delta_{\Complementary{\Bgent}} \relation
      \bp{ \cup_{k=0}^n \np{\Delta_{\Complementary{\Bgent}} \relation}^k }\Bgent}
      \tag{using the induction assumption~\( {\cal H}_n \)}
    \\
    &=
      \np{\Delta \Bgent} \cup \Bp{
      \bp{ \cup_{k=1}^{n+1} \np{\Delta_{\Complementary{\Bgent}} \relation}^k }\Bgent}
    \\
    &=
      \bp{ \cup_{k=0}^{n+1} \np{\Delta_{\Complementary{\Bgent}} \relation}^k }\Bgent
      \eqfinp
      \tag{as \( \np{\Delta_{\Complementary{\Bgent}} \relation}^0 =\Delta\)}
  \end{align*}
  Thus, we have proven the induction assumption~\( {\cal H}_{n+1} \).
\medskip
  % Now, for $n=1$ we have using the preliminary equality that
  % $\np{\Delta \cup \relation}\Bgent=\np{\Delta \Bgent} \cup \np{\relation \Bgent} =
  % \np{\Delta \Bgent} \cup \bp{\np{\Delta_{\Complementary{\Bgent}}\relation} \Bgent}=
  % \bp{\Delta \cup \np{\Delta_{\Complementary{\Bgent}}\relation}}\Bgent$.

  Now, let us suppose that $\bgent \in
  \TransitiveReflexiveClosure{\relation}\Bgent$. Then, there exists an integer $n\ge 1$ 
  such that $\bgent \in \np{\cup_{k=0}^n \relation^k}\Bgent$; using the just
  proven property~\( {\cal H}_n \), we get that
  $\bgent \in \bp{ \cup_{k=0}^{n} \np{\Delta_{\Complementary{\Bgent}} \relation}^k }\Bgent$ and, therefore,
  $\bgent \in
  \npTransitiveReflexiveClosure{\Delta_{\Complementary{\Bgent}}\relation}\Bgent$.
  Thus, we have shown that
  $ \TransitiveReflexiveClosure{\relation}\Bgent \subset
  \npTransitiveReflexiveClosure{\Delta_{\Complementary{\Bgent}}\relation}\Bgent$.
  The converse inclusion is easier to prove as
  \( \Delta_{\Complementary{\Bgent}}\relation \subset\relation \).
  Finally, we have shown the equality
  $ \TransitiveReflexiveClosure{\relation}\Bgent =
  \npTransitiveReflexiveClosure{\Delta_{\Complementary{\Bgent}}\relation}\Bgent$,
  which is~\eqref{eq:R-kleene-G-equals}.
  \medskip
  
  \noindent $\bullet$
  The following inclusion is easy to prove:
  \begin{align}
    \forall \relation \subset \AGENT\times\AGENT \eqsepv
    \forall \Bgent \subset \AGENT \eqsepv
    \Delta_{\relation\Bgent} 
    &\subset 
      \relation\Delta_{\Bgent}\Converse{\relation}
      \eqfinp
      \label{eq:delta-R-G-subset-R-delta-G-Rinv}
  \end{align}
%  \medskip
  
  \noindent $\bullet$
  We prove that
  \begin{align}
    \TransitiveReflexiveClosureParentalPrecedence  \Delta_{\AgentSubsetW}
    \TransitiveReflexiveClosureConverseParentalPrecedence
    & =
      \TransitiveReflexiveClosureParentalPrecedence \Delta_{\np{\TransitiveReflexiveClosure{\Precedence}\AgentSubsetW}}
      \TransitiveReflexiveClosureConverseParentalPrecedence\eqfinp
      \label{eq:star-no-star}
  \end{align}
  Using Equation~\eqref{eq:R-kleene-G-equals} with $\relation{=}\Precedence$ and
  $\Bgent{=}\AgentSubsetW$ gives
  $\TransitiveReflexiveClosure{\Precedence}\AgentSubsetW
  =\npTransitiveReflexiveClosure{\Delta_{\Complementary{\AgentSubsetW}}\Precedence}
  \AgentSubsetW = \TransitiveReflexiveClosureParentalPrecedence \AgentSubsetW$.
  Combined with the Inclusion~\eqref{eq:delta-R-G-subset-R-delta-G-Rinv}, we get
  \( \Delta_{\TransitiveReflexiveClosure{\Precedence}\AgentSubsetW} 
  = \Delta_{\TransitiveReflexiveClosureParentalPrecedence \AgentSubsetW} 
  \subset \TransitiveReflexiveClosureParentalPrecedence \Delta_{\AgentSubsetW}
  \TransitiveReflexiveClosureConverseParentalPrecedence
  \). 
  Thus, we obtain that 
  \begin{align*}
    \TransitiveReflexiveClosureParentalPrecedence  \Delta_{\np{\TransitiveReflexiveClosure{\Precedence}\AgentSubsetW}}\TransitiveReflexiveClosureConverseParentalPrecedence
    & \subset\TransitiveReflexiveClosureParentalPrecedence
      \np{\TransitiveReflexiveClosureParentalPrecedence \Delta_{\AgentSubsetW} \TransitiveReflexiveClosureConverseParentalPrecedence} 
      \TransitiveReflexiveClosureConverseParentalPrecedence
      =%\subset
      {\TransitiveReflexiveClosureParentalPrecedence \Delta_{\AgentSubsetW} \TransitiveReflexiveClosureConverseParentalPrecedence}\eqfinp
  \end{align*}
  Thus, we have obtained the inclusion
  \(
  \TransitiveReflexiveClosureParentalPrecedence  \Delta_{\AgentSubsetW}
  \TransitiveReflexiveClosureConverseParentalPrecedence
  \supset
  \TransitiveReflexiveClosureParentalPrecedence \Delta_{\np{\TransitiveReflexiveClosure{\Precedence}\AgentSubsetW}}
  \TransitiveReflexiveClosureConverseParentalPrecedence
  \).
  The reverse inclusion  follows from the fact that
  \(\AgentSubsetW \subset  \TransitiveReflexiveClosureOfSet{\AgentSubsetW}=
  \TransitiveReflexiveClosure{\EDGE}\AgentSubsetW \)
  by~\eqref{eq:TransitiveReflexiveClosure_AgentSubsetW},   
  which gives
  \(
  \TransitiveReflexiveClosureParentalPrecedence  \Delta_{\AgentSubsetW}\TransitiveReflexiveClosureConverseParentalPrecedence
  \subset 
  \TransitiveReflexiveClosureParentalPrecedence \Delta_{\np{\TransitiveReflexiveClosure{\Precedence}\AgentSubsetW}}\TransitiveReflexiveClosureConverseParentalPrecedence
  \).

  \medskip
  \noindent $\bullet$
  Finally, we prove that
  $\ConditionalActive =\ConditionalActiveStar$. For that purpose, it suffices
  to show that replacing the subexpressions $\Delta_{\AgentSubsetW}$ by
  $\Delta_{\np{\TransitiveReflexiveClosure{\Precedence}\AgentSubsetW}}$ in the
  expression~\eqref{eq:conditional_active_relation} of $\ConditionalActive$ does not change the relation. Using the
  definition of $\ConditionalActive$ in
  Equation~\eqref{eq:conditional_active_relation}, we obtain that
  $\Delta_{\AgentSubsetW}$ appears only in subexpressions of the form
  $\ConditionalAscendent \Delta_{\AgentSubsetW} \ConverseConditionalAscendent$
  or
  $\ConditionalCommonCause \Delta_{\AgentSubsetW} \ConverseConditionalAscendent$
  or $\ConditionalAscendent \Delta_{\AgentSubsetW} \ConditionalCommonCause$ or
  $\ConditionalCommonCause \Delta_{\AgentSubsetW} \ConditionalCommonCause$. Now,
  using the fact that the two relations $\ConditionalAscendent$ and
  $\ConditionalCommonCause$ always end with
  $\TransitiveReflexiveClosureParentalPrecedence$ and the two relation
  $\ConverseConditionalAscendent$ and $\ConditionalCommonCause$ always start with
  $\TransitiveReflexiveClosureConverseParentalPrecedence$ we obtain that
  $\Delta_{\AgentSubsetW}$ appears only in subexpressions of the form
  $\TransitiveReflexiveClosureParentalPrecedence \Delta_{\AgentSubsetW}
  \TransitiveReflexiveClosureConverseParentalPrecedence$.  We conclude, using
  Equation~\eqref{eq:star-no-star}, that $\Delta_{\AgentSubsetW}$ can be
  replaced by
  $\Delta_{\np{\TransitiveReflexiveClosure{\Precedence}\AgentSubsetW}}$ in
  $\ConditionalActive$ without changing the relation.
  \medskip

  This ends the proof.
\end{proof}



\section{Proof of Theorem~\ref{th:ConditionalDirectionalSeparation_IFF_relation}}
\label{Proof_of_Theorem}

The proof of Theorem~\ref{th:ConditionalDirectionalSeparation_IFF_relation}
relies on the following
Proposition~\ref{pr:ConditionalDirectionalSeparation_seilpmi_relation}
(itself based on Lemmas~\ref{lem:BlockedUndirectedPaths_elementary} and~\ref{lem:ConditionalDirectionalSeparation_seilpmi_relation-two})
and on Lemma~\ref{lem:induction-lemma}, that we are going to prove. 

The following Lemma~\ref{lem:BlockedUndirectedPaths_elementary}
displays elementary relational patterns (composition of relations)
whose deployment in paths contain active \undirectedEdgePaths.
% 
The binary relations below have been introduced in Definition~\ref{de:all_the_relations},
except for the two additional ones defined in~\eqref{eq:CousinhoodStar}
and in~\eqref{eq:conditional_active_relationStar}. 
% 
For any integer \( n\geq 1\), we denote by \( \1_{n}=\np{+1,\ldots,+1} \)
(resp. \( -\1_{n}=\np{-1,\ldots,-1} \)) the vector of length~$n$ made of~$+1$ (resp. of~$-1$).
%
We recall that the deployment~\( \DeploymentInUPaths{\relation}{\OrientedGraph}
\) in \undirectedEdgePaths\ of a binary relation~$\relation$ has been defined
in~\eqref{eq:DeploymentInUPaths}%
\ifpreprint
, and the intermediate vertices~$\Internals_{\UPATH}$ 
of an \undirectedEdgePath\ in~\eqref{eq:Internals_UPATH}
\fi
.
\ifpreprint 
\begin{lemma}
  \label{lem:BlockedUndirectedPaths_elementary}
  Let \( \npOrientedGraph \) be a graph, 
  and $\AgentSubsetW\subset\VERTEX$ be a subset of vertices.
  For any two vertices $\bgent$, $\cgent\in \VERTEX$, we have 
  \begin{subequations}
    \begin{align}
      \bgent \ConditionalAscendent\cgent 
      \implies & 
                 \mtext{there exists } \UndirectedPath \in \UPATH\np{\OrientedGraph}
                 \mtext{ such that }
                 \nonumber \\
               & 
                 \bp{\path, \1_{\cardinal{\path}} } 
                 \in 
                 \DeploymentInUPaths{\np{\bgent,\cgent}}{\OrientedGraph} 
                 \cap \ActiveUndirectedPaths{\OrientedGraph} %\neq\emptyset
                 \nonumber 
      \\
               & 
                 \text{and }
                 \Internals_{\UPATH}\np{\UndirectedPath} \subset \TransitiveClosureParentalPrecedence \cgent
                 \eqfinv
                 \label{eq:BlockedUndirectedPaths_elementary_Delta_ComplementaryAgentSubsetW_EDGE} 
      \\
      \bgent \ConverseConditionalAscendent\cgent 
      \implies
               & 
                 \mtext{there exists }\UndirectedPath \in \UPATH\np{\OrientedGraph}
                 \mtext{ such that }
                 \nonumber \\
               & 
                 \bp{\path, -\1_{\cardinal{\path}} } 
                 \in 
                 \DeploymentInUPaths{\np{\bgent,\cgent}}{\OrientedGraph} 
                 \cap \ActiveUndirectedPaths{\OrientedGraph}
                 \nonumber \\                                      % PREPRINTONLY
               &                                                   % PREPRINTONLY
                 \text{and }                                       % PREPRINTONLY
                 \Internals_{\UPATH}\np{\UndirectedPath} \subset    % PREPRINTONLY
                 \TransitiveClosureParentalPrecedence \bgent       % PREPRINTONLY
                 \eqfinv
                 \label{eq:BlockedUndirectedPaths_elementary_Converse_EDGE_Delta_ComplementaryAgentSubsetW} 
      \\
      \bgent\ConditionalCommonCause\cgent
      \implies
               & 
                 \mtext{there exists }\UndirectedPath=\np{\path,\orient} \in
                 \UPATH\np{\OrientedGraph},
                 \mtext{with } \cardinal{\UndirectedPath} \geq 2
                 \nonumber \\
               & \mtext{and } \orient=\np{-1,\ldots,+1} 
                 \in \na{-1}\times\na{-1,+1}^{\cardinal{\UndirectedPath}-2}\times\na{+1},
                 \mtext{ such that}
                 \nonumber \\
               & 
                 \UndirectedPath=\np{\path,\orient}=\bp{\path, \np{-1,\ldots,+1} } 
                 \in 
                 \DeploymentInUPaths{\np{\bgent,\cgent}}{\OrientedGraph}
                 \cap \ActiveUndirectedPaths{\OrientedGraph} %\neq\emptyset
                 \nonumber \\  % PREPRINTONLY
               & % PREPRINTONLY
                 \text{and }% PREPRINTONLY
                 \Internals_{\UPATH}\np{\UndirectedPath} \subset% PREPRINTONLY
                 \TransitiveClosureParentalPrecedence \na{\bgent,\cgent}% PREPRINTONLY
                 % \TopologicalClosure{\AgentSubsetW}{\na{\bgent,\cgent}} % PREPRINTONLY
                 \eqfinv
                 \label{eq:BlockedUndirectedPaths_elementary_TransitiveClosure_Converse_EDGE_Delta_ComplementaryAgentSubsetW} 
      \\
      \bgent\CousinhoodStar\cgent
      \implies
               & 
                 \bgent \in \TransitiveReflexiveClosureOfSet{\AgentSubsetW}, \,
                 \cgent \in \TransitiveReflexiveClosureOfSet{\AgentSubsetW} \mtext{ and }
                 \bgent = \cgent \mtext{ or }
                 \nonumber \\
               & 
                 \mtext{there exists }\UndirectedPath=\np{\path,\orient} \in \UPATH\np{\OrientedGraph},
                 \mtext{with } \cardinal{\UndirectedPath} \geq 2                
                 \nonumber \\
               & \mtext{and } \orient=\np{-1,\ldots,+1} 
                 \in \na{-1}\times\na{-1,+1}^{\cardinal{\UndirectedPath}-2}\times\na{+1},
                 \mtext{ such that}
                 \nonumber \\
               & 
                 \UndirectedPath=\np{\path,\orient}=\bp{\path, \np{-1,\ldots,+1} } 
                 \in 
                 \DeploymentInUPaths{\np{\bgent,\cgent}}{\OrientedGraph}
                 \cap \ActiveUndirectedPaths{\OrientedGraph} % \neq\emptyset
                 \nonumber \\ % PREPRINTONLY
               & % PREPRINTONLY
                 \text{and } \Internals_{\UPATH}\np{\UndirectedPath} \subset% PREPRINTONLY
                 \TransitiveReflexiveClosureParentalPrecedence \AgentSubsetW% PREPRINTONLY
                 \eqfinp
                 \label{eq:BlockedUndirectedPaths_elementary_TransitiveClosure_TransitiveClosure}
    \end{align}
    \ifpreprint 
      Moreover, in Implication~\eqref{eq:BlockedUndirectedPaths_elementary_TransitiveClosure_TransitiveClosure},
      if a vertice $\dgent\in\Internals_{\UPATH}\np{\UndirectedPath} $ belongs
      to~$\AgentSubsetW$,
      it necessarily appears in a subpath of $\UndirectedPath$ of the form
      \(
      { \Bp{ \bp{ \np{\tail{\vertex},\dgent},  \np{\dgent, \head{\vertex} }
          },\np{+1,-1} } }
      \).
    \fi 
  \end{subequations}
\end{lemma}

\else 

\begin{lemma}
  \label{lem:BlockedUndirectedPaths_elementary}
  Let \( \npOrientedGraph \) be a graph, 
  and $\AgentSubsetW\subset\VERTEX$ be a subset of vertices.
  For any two vertices $\bgent$, $\cgent\in \VERTEX$, we have 
  \begin{subequations}
    \begin{align}
      \bgent \ConditionalAscendent\cgent 
      \implies & 
                 \mtext{there exists } \UndirectedPath \in \UPATH\np{\OrientedGraph}
                 \mtext{ such that }
                 \nonumber \\
               & 
                 \bp{\path, \1_{\cardinal{\path}} } 
                 \in 
                 \DeploymentInUPaths{\np{\bgent,\cgent}}{\OrientedGraph} 
                 \cap \ActiveUndirectedPaths{\OrientedGraph} %\neq\emptyset
                 % \nonumber 
                 % \\
                 % & 
                 % \text{and }
                 % \Internals_{\UPATH}\np{\UndirectedPath} \subset \TransitiveClosureParentalPrecedence \cgent
                     \eqfinv
                     \label{eq:BlockedUndirectedPaths_elementary_Delta_ComplementaryAgentSubsetW_EDGE} 
      \\
      \bgent \ConverseConditionalAscendent\cgent 
      \implies
               & 
                 \mtext{there exists }\UndirectedPath \in \UPATH\np{\OrientedGraph}
                 \mtext{ such that }
                 \nonumber \\
               & 
                 \bp{\path, -\1_{\cardinal{\path}} } 
                 \in 
                 \DeploymentInUPaths{\np{\bgent,\cgent}}{\OrientedGraph} 
                 \cap \ActiveUndirectedPaths{\OrientedGraph}
                 % \nonumber \\                                      % PREPRINTONLY
                 % &                                                   % PREPRINTONLY
                 % \text{and }                                       % PREPRINTONLY
                 % \Internals_{\UPATH}\np{\UndirectedPath} \subset    % PREPRINTONLY
                 % \TransitiveClosureParentalPrecedence \bgent       % PREPRINTONLY
                     \eqfinv
                     \label{eq:BlockedUndirectedPaths_elementary_Converse_EDGE_Delta_ComplementaryAgentSubsetW} 
      \\
      \bgent\ConditionalCommonCause\cgent
      \implies
               & 
                 \mtext{there exists }\UndirectedPath=\np{\path,\orient} \in
                 \UPATH\np{\OrientedGraph},
                 \mtext{with } \cardinal{\UndirectedPath} \geq 2
                 \nonumber \\
               & \mtext{and } \orient=\np{-1,\ldots,+1} 
                 \in \na{-1}\times\na{-1,+1}^{\cardinal{\UndirectedPath}-2}\times\na{+1},
                 \mtext{ such that}
                 \nonumber \\
               & 
                 \UndirectedPath=\np{\path,\orient}=\bp{\path, \np{-1,\ldots,+1} } 
                 \in 
                 \DeploymentInUPaths{\np{\bgent,\cgent}}{\OrientedGraph}
                 \cap \ActiveUndirectedPaths{\OrientedGraph} %\neq\emptyset
                 % \nonumber \\  % PREPRINTONLY
                 % & % PREPRINTONLY
                 % \text{and }% PREPRINTONLY
                 % \Internals_{\UPATH}\np{\UndirectedPath} \subset% PREPRINTONLY
                 % \TransitiveClosureParentalPrecedence \na{\bgent,\cgent}% PREPRINTONLY
                     \eqfinv
                     \label{eq:BlockedUndirectedPaths_elementary_TransitiveClosure_Converse_EDGE_Delta_ComplementaryAgentSubsetW} 
      \\
      \bgent\CousinhoodStar\cgent
      \implies
               & 
                 \bgent \in \TransitiveReflexiveClosureOfSet{\AgentSubsetW}, \,
                 \cgent \in \TransitiveReflexiveClosureOfSet{\AgentSubsetW} \mtext{ and }
                 \nonumber \\
               & 
                 \mtext{there exists }\UndirectedPath=\np{\path,\orient} \in \UPATH\np{\OrientedGraph},
                 \mtext{with } \cardinal{\UndirectedPath} \geq 2                
                 \nonumber \\
               & \mtext{and } \orient=\np{-1,\ldots,+1} 
                 \in \na{-1}\times\na{-1,+1}^{\cardinal{\UndirectedPath}-2}\times\na{+1},
                 \mtext{ such that}
                 \nonumber \\
               & 
                 \UndirectedPath=\np{\path,\orient}=\bp{\path, \np{-1,\ldots,+1} } 
                 \in 
                 \DeploymentInUPaths{\np{\bgent,\cgent}}{\OrientedGraph}
                 \cap \ActiveUndirectedPaths{\OrientedGraph} % \neq\emptyset
                 % \nonumber \\ % PREPRINTONLY
                 % & % PREPRINTONLY
                 % \text{and } \Internals_{\UPATH}\np{\UndirectedPath} \subset% PREPRINTONLY
                 % \TransitiveReflexiveClosureParentalPrecedence \AgentSubsetW% PREPRINTONLY
                     \eqfinp
                     \label{eq:BlockedUndirectedPaths_elementary_TransitiveClosure_TransitiveClosure}
    \end{align}
    \ifpreprint 
    Moreover, in Implication~\eqref{eq:BlockedUndirectedPaths_elementary_TransitiveClosure_TransitiveClosure},
    if a vertice $\dgent\in\Internals_{\UPATH}\np{\UndirectedPath} $ belongs
    to~$\AgentSubsetW$,
    it necessarily appears in a subpath of $\UndirectedPath$ of the form
    \(
    { \Bp{ \bp{ \np{\tail{\vertex},\dgent},  \np{\dgent, \head{\vertex} }
        },\np{+1,-1} } }
    \).
    \fi 
  \end{subequations}
\end{lemma}
\fi 

\begin{proof} \quad 

  \noindent $\bullet$ 
  We prove the implication~\eqref{eq:BlockedUndirectedPaths_elementary_Delta_ComplementaryAgentSubsetW_EDGE}.
  Let $\bgent$, $\cgent\in \VERTEX$ be such that
  \( \bgent \ConditionalAscendent\cgent \).   
  By~\eqref{eq:conditional_ascendent_relation}, we have that
  \( \ConditionalAscendent\cgent =
  \EDGE\npTransitiveReflexiveClosure{\Delta_{\Complementary{\AgentSubsetW}}\EDGE}\),
  hence that \( \bgent
  \EDGE\npTransitiveReflexiveClosure{\Delta_{\Complementary{\AgentSubsetW}}\EDGE}\cgent
  \).
  %
    As \( \npTransitiveReflexiveClosure{\Delta_{\Complementary{\AgentSubsetW}}\EDGE}=
  \Delta \cup \cup_{n=1}^{\infty}
  \np{\Delta_{\Complementary{\AgentSubsetW}}\EDGE}^{n} \) by definition, 
  if \( \bgent \EDGE\npTransitiveReflexiveClosure{\Delta_{\Complementary{\AgentSubsetW}}\EDGE}\cgent \),
  then either 
  \( \bgent \EDGE \cgent \)
  or there exists $n\geq 1$ such that 
  \( \bgent \EDGE\np{\Delta_{\Complementary{\AgentSubsetW}}\EDGE}^{n} \cgent \).
  Thus, we consider two cases.

  If \( \bgent \EDGE \cgent \),
  the \undirectedEdgePath\ \( \bp{ \np{\bgent,\cgent}, +1 } \)
  is both in \( \DeploymentInUPaths{\np{\bgent,\cgent}}{\OrientedGraph} \),
  by definition~\eqref{eq:DeploymentInUPaths},
  and belongs to~\( \ActiveUndirectedPaths{\OrientedGraph} \),
  as it is of length~1 hence is active (see Definition~\ref{de:ActivePaths}). 

  If \( \bgent\EDGE \np{\Delta_{\Complementary{\AgentSubsetW}}\EDGE}^{n} \cgent \),
  with $n\geq 1$, then 
  there exists a sequence \( \sequence{\vertex_i}{i\in\ic{0,n+1}} \) in~$\VERTEX$ such that 
  \( \vertex_{0}=\bgent \), \( \vertex_{n+1}=\cgent \), 
  and 
  \( \vertex_{i-1} \EDGE \vertex_{i} \),
  \( \vertex_{i} \in \Complementary{\AgentSubsetW} \),
  \( \vertex_{i} \EDGE \vertex_{i+1} \),
  for $i\in\ic{1,n}$.
  The following 
  \( \UndirectedPath= \bp{\sequence{ \np{\vertex_i,\vertex_{i+1}}
    }{i\in\ic{0,n}}, \1_{n+1} } \)
  is an \undirectedEdgePath\ that 
  belongs to
  \( \DeploymentInUPaths{\np{\bgent,\cgent}}{\OrientedGraph} \),
  as \( \Projection\np{\UndirectedPath}=\np{\bgent,\cgent} \),
  and also to~\( \ActiveUndirectedPaths{\OrientedGraph} \).
  Indeed, all the \undirectedEdgeSubPaths \\
  \( \Bp{ \bp{ \np{\vertex_{i-1},\vertex_{i}},
      \np{\vertex_{i},\vertex_{i+1}} }, \np{+1,+1} } \)
  for $i\in\ic{1,n}$, satisfy Item~\ref{it:ActivePaths_case1} in
  Definition~\ref{de:ActivePaths} 
  because \( \vertex_{i} \in \Complementary{\AgentSubsetW} \).
  \ifpreprint
  It remains to show that
  $\Internals_{\UPATH}\np{\UndirectedPath} \subset\TransitiveClosureParentalPrecedence \cgent$.
  This inclusion easily follows from the fact that,
  by definition~\eqref{eq:Internals_PATH} of the intermediate vertices,
  we have that \( \Internals_{\UPATH}\np{\UndirectedPath} =
  \na{ \vertex_{2}, \ldots, \vertex_{n-1} } \) where
  \( \vertex_{n-1} \in \Complementary{\AgentSubsetW}\cap\EDGE\cgent
  = \Delta_{\Complementary{\AgentSubsetW}}\EDGE\cgent
  = \ParentalPrecedence\cgent  \subset \TransitiveClosureParentalPrecedence
  \cgent \),
  \ldots,
  \( \vertex_{n-k} \in \np{\ParentalPrecedence}^k\cgent  \subset \TransitiveClosureParentalPrecedence \cgent \), 
  \ldots,   \( \vertex_{2} \in \np{\ParentalPrecedence}^{n-2}\cgent
  \subset \TransitiveClosureParentalPrecedence \cgent \).
  \fi 
  \medskip
  
  
  \noindent $\bullet$ 
  We prove the implication~\eqref{eq:BlockedUndirectedPaths_elementary_Converse_EDGE_Delta_ComplementaryAgentSubsetW}
  in the same way
  (here, all the \undirectedEdgeSubPaths\ satisfy Item~\ref{it:ActivePaths_case2} in
  Definition~\ref{de:ActivePaths}).
  \medskip

  
  \noindent $\bullet$ 
  We
  prove the implication~\eqref{eq:BlockedUndirectedPaths_elementary_TransitiveClosure_Converse_EDGE_Delta_ComplementaryAgentSubsetW}.
  Let $\bgent$, $\cgent\in \VERTEX$ be such that
  \( \bgent\ConditionalCommonCause\cgent \).
  By definition~\eqref{eq:common_cause} of \(\ConditionalCommonCause\),
  we obtain that \( \bgent \ConverseConditionalAscendent
  \Delta_{\Complementary{\AgentSubsetW}} \ConditionalAscendent \cgent \).
  As a consequence of the definition of the composition of relations, 
  % Therefore,
  % if \( \bgent \npTransitiveClosure{\Converse{\EDGE} \Delta_{\Complementary{\AgentSubsetW}} }
  % \npTransitiveClosure{\Delta_{\Complementary{\AgentSubsetW}}\EDGE}\cgent \),
  % then, 
  there exists \( \dgent \in \Complementary{\AgentSubsetW} \) such that 
  \( \bgent
  \ConverseConditionalAscendent
  \dgent \)
  and
  \( \dgent  \ConditionalAscendent \cgent \).
  Thus, by~\eqref{eq:BlockedUndirectedPaths_elementary_Converse_EDGE_Delta_ComplementaryAgentSubsetW},
  there exists \( \UndirectedPathbis =\bp{\pathbis, -\1_{\cardinal{\pathbis}} } 
  \in \DeploymentInUPaths{\np{\bgent,\dgent}}{\OrientedGraph} 
  \cap \ActiveUndirectedPaths{\OrientedGraph}\) and
  by~\eqref{eq:BlockedUndirectedPaths_elementary_Delta_ComplementaryAgentSubsetW_EDGE},
  there exists \( \UndirectedPathter =\bp{\pathter,  \1_{\cardinal{\pathter}} }
  \in \DeploymentInUPaths{\np{\dgent,\cgent}}{\OrientedGraph} 
  \cap \ActiveUndirectedPaths{\OrientedGraph} \).
  % 
  We consider the \undirectedEdgePath\ 
  \( \UndirectedPath = \UndirectedPathbis \ltimes \UndirectedPathter  \in
  \DeploymentInUPaths{\np{\bgent,\cgent}}{\OrientedGraph} \)
  obtained by concatenation as in~\eqref{eq:concatenation_UndirectedPath},
  and which is such that \( \cardinal{\UndirectedPath} \geq 2 \). 
  We claim that \( \UndirectedPath \in
  \ActiveUndirectedPaths{\OrientedGraph} \).
  Indeed, \( \UndirectedPathbis \in
  \ActiveUndirectedPaths{\OrientedGraph} \)
  and \(\UndirectedPathter \in
  \ActiveUndirectedPaths{\OrientedGraph} \)
  by assumption, so that, by
  Equation~\eqref{eq:ActiveUndirectedPaths_subpaths_junctions_when_reconcatenating} in
  Lemma~\ref{lem:ActiveUndirectedPaths_subpaths_junctions_when_reconcatenating},
  it only remains to show that 
  \[
    \Bp{ \bp{ \np{\vertexbis,\dgent},  \np{\dgent, \vertexter } },
      \np{-1,+1} } =
    {\Head{\ProjectionUndirectedPath}\np{\UndirectedPathbis}
      \ltimes
      \Tail{\ProjectionUndirectedPath}\np{\UndirectedPathter}}
    \in \ActiveUndirectedPaths{\OrientedGraph}
    \eqfinv
  \]
  where 
  % the triplet \( \np{\vertexbis,\dgent, \vertexter } \), where 
  \( \np{\vertexbis,\dgent} \) is the first edge of the \undirectedEdgePath\ 
  \( \UndirectedPathbis = \bp{\pathbis, -\1_{\cardinal{\pathbis}} } \) and 
  \( \np{\dgent, \vertexter } \) is the last edge of the \undirectedEdgePath\ 
  \( \UndirectedPathter =\bp{\pathter,  \1_{\cardinal{\pathter}} } \).
  % By assumption on \( \UndirectedPathbis \) and \( \UndirectedPathter\),
  % the \undirectedEdgePath\ 
  % \( \UndirectedPath = \UndirectedPathbis \ltimes \UndirectedPathter\) contains the couple
  % \( \Bp{ \bp{ \np{\vertexbis,\dgent},  \np{\dgent, \vertexter } },
  % \np{-1,+1} } \), where \( \dgent \in \Complementary{\AgentSubsetW} \).
  Now, the above subpath satisfies Item~\ref{it:ActivePaths_case3} in
  Definition~\ref{de:ActivePaths} 
  because \( \vertex_{i} \in \Complementary{\AgentSubsetW} \).
  % 
  % Now, the above subpath pattern corresponds to
  % none of the four cases in
  % Definition~\ref{de:ActivePaths},
  % because \( \dgent \in \Complementary{\AgentSubsetW} \)
  % for the first three cases,
  % and because of the orientation \( \np{-1,+1} \)
  % for the last fourth case.
  We deduce that \( \UndirectedPath = \UndirectedPathbis \ltimes
  \UndirectedPathter
  = \Bp{ \np{\pathbis,\pathter}, 
    \bp{ -\1_{\cardinal{\pathbis}}, \1_{\cardinal{\pathter}} } } \in 
  \DeploymentInUPaths{\np{\bgent,\cgent}}{\OrientedGraph}
  \cap \ActiveUndirectedPaths{\OrientedGraph} \).
  \ifpreprint 
  It remains to show that 
  $\Internals_{\UPATH}\np{\UndirectedPath} \subset
  \TransitiveClosureParentalPrecedence \na{\bgent,\cgent} $
  but this comes from~\eqref{eq:Internals_PATH_concatenation} which gives
  \[
    \Internals_{\UPATH}\np{\UndirectedPath} = 
    \underbrace{\Internals_{\UPATH}\np{\UndirectedPathbis}}_{\subset
      \TransitiveClosureParentalPrecedence\bgent}
    \cup
    \overbrace{ \na{\dgent} }^{\subset \TransitiveClosureParentalPrecedence\bgent
      \cap \TransitiveClosureParentalPrecedence\cgent}
    \cup
    \underbrace{\Internals_{\UPATH}\np{\UndirectedPathter}}_{\subset
      \TransitiveClosureParentalPrecedence\cgent}
    \subset
    \TransitiveClosureParentalPrecedence\bgent
    \cup \np{ \TransitiveClosureParentalPrecedence\bgent
      \cap \TransitiveClosureParentalPrecedence\cgent}
    \cup
    \TransitiveClosureParentalPrecedence\cgent
    = \TransitiveClosureParentalPrecedence \na{\bgent,\cgent} 
    \eqfinp
  \]
  % This inclusion follows from the fact that
  % the intermediate and first vertice of the path $\UndirectedPathter$ are in
  % $\Complementary{\AgentSubsetW}$ and are all ancestors of the node ${\cgent}$ for the relation $\Precedence$
  % while the intermediate and last vertice of the path $\UndirectedPathbis$ are in
  % $\Complementary{\AgentSubsetW}$ and are all ancestors of the node ${\bgent}$ for the relation $\Precedence$.
  \fi
  \medskip

  \noindent $\bullet$ 
  We
  prove the implication~\eqref{eq:BlockedUndirectedPaths_elementary_TransitiveClosure_TransitiveClosure}.
  We suppose that \( \bgent\CousinhoodStar\cgent \).
  As \( \CousinhoodStar =
  \Delta_{\TransitiveReflexiveClosureOfSet{\AgentSubsetW}}
  \cup  \bpTransitiveClosure{ \Delta_{\TransitiveReflexiveClosureOfSet{\AgentSubsetW}}
    \ConditionalCommonCause \Delta_{\TransitiveReflexiveClosureOfSet{\AgentSubsetW}} } \)
  by~\eqref{eq:CousinhoodStar},
  we consider three cases:
  either \( \bgent\Delta_{\TransitiveReflexiveClosureOfSet{\AgentSubsetW}}\cgent \),
  or  \( \bgent
  \Delta_{\TransitiveReflexiveClosureOfSet{\AgentSubsetW}}\ConditionalCommonCause
  \Delta_{\TransitiveReflexiveClosureOfSet{\AgentSubsetW}} \cgent \)
  or there exists $n \geq 1$ such that 
  \( \bgent \bp{\Delta_{\TransitiveReflexiveClosureOfSet{\AgentSubsetW}}
    \ConditionalCommonCause
    \Delta_{\TransitiveReflexiveClosureOfSet{\AgentSubsetW}}}^{n+1} \cgent \).

  Suppose that \( \bgent\Delta_{\TransitiveReflexiveClosureOfSet{\AgentSubsetW}}\cgent \).
  Then \( \bgent=\cgent \), and thus the \undirectedEdgePath\ 
  \( \UndirectedPath = \bp{\np{\bgent,\cgent}, +1} \in \UPATH^0\np{\OrientedGraph} \) is active as
  any \undirectedEdgePath\ of length~0 or 1 is active by
  Definition~\ref{de:ActivePaths}
  and by~\eqref{eq:undirectedEdgePaths_of_length0or1}. 
  
  Suppose that \( \bgent
  \Delta_{\TransitiveReflexiveClosureOfSet{\AgentSubsetW}}\ConditionalCommonCause
  \Delta_{\TransitiveReflexiveClosureOfSet{\AgentSubsetW}} \cgent \).
  Then, \( \bgent \in \TransitiveReflexiveClosureOfSet{\AgentSubsetW} \), 
  \( \cgent \in \TransitiveReflexiveClosureOfSet{\AgentSubsetW} \) and
  \( \bgent\ConditionalCommonCause \cgent \).
  Therefore, 
  by~\eqref{eq:BlockedUndirectedPaths_elementary_TransitiveClosure_Converse_EDGE_Delta_ComplementaryAgentSubsetW}
  there exists \( \UndirectedPath \in \UPATH\np{\OrientedGraph} \),
  with \( \cardinal{\UndirectedPath} \geq 2 \),
  and \( \orient %\np{-1,\ldots,+1}
  \in\na{-1}\times\na{-1,+1}^{\cardinal{\UndirectedPath}-2}\times\na{+1} \)
  such that \( \bp{\path, \orient %\np{-1,\ldots,+1}
  } \in \DeploymentInUPaths{\np{\bgent,\cgent}}{\OrientedGraph}
  \cap \ActiveUndirectedPaths{\OrientedGraph} \)
  \ifpreprint
  and that
  \(\Internals_{\UPATH}\np{\UndirectedPath} \subset
  \TransitiveClosureParentalPrecedence\na{\bgent,\cgent} \).
  As \( \na{\bgent,\cgent} \subset \TransitiveReflexiveClosureOfSet{\AgentSubsetW} \),
  we get that \( \Internals_{\UPATH}\np{\UndirectedPath} \subset
  \TransitiveClosureParentalPrecedence
  \TransitiveReflexiveClosureOfSet{\AgentSubsetW} = \TransitiveClosureParentalPrecedence \TransitiveReflexiveClosureParentalPrecedence
  \AgentSubsetW = \TransitiveClosureParentalPrecedence \AgentSubsetW \),
  since \(    \TransitiveReflexiveClosureOfSet{\AgentSubsetW}=
  \TransitiveReflexiveClosure{\EDGE}\AgentSubsetW \)
  by~\eqref{eq:TransitiveReflexiveClosure_AgentSubsetW}
  and \( \TransitiveClosureParentalPrecedence
  \TransitiveReflexiveClosureParentalPrecedence =
  \TransitiveClosureParentalPrecedence \)%
  \fi
  .

  Suppose that
  \( \bgent \bp{\Delta_{\TransitiveReflexiveClosureOfSet{\AgentSubsetW}}
    \ConditionalCommonCause
    \Delta_{\TransitiveReflexiveClosureOfSet{\AgentSubsetW}}}^{n+1} \cgent \).
  Then, there exists a sequence 
  \( \sequence{\dgent_i}{i\in\ic{0,n+1}} \) in~\( \TransitiveReflexiveClosureOfSet{\AgentSubsetW} \)
  such that \( \dgent_0=\bgent \), \( \dgent_{n+1}=\cgent \) 
  and \( \dgent_i \ConditionalCommonCause \dgent_{i+1} \) for $i\in\ic{0,n}$.
  Therefore, 
  by~\eqref{eq:BlockedUndirectedPaths_elementary_TransitiveClosure_Converse_EDGE_Delta_ComplementaryAgentSubsetW},
  there exists a sequence 
  \( \sequence{\UndirectedPath_i}{i\in\ic{0,n}} \) in~\( \UPATH\np{\OrientedGraph}\)
  of \undirectedEdgePaths\ such that
  \( \cardinal{\UndirectedPath_i} \geq 2 \) for $i\in\ic{0,n}$,
  and that 
  \( \UndirectedPath_i = \np{\path_i, \orient_i }
  \in \DeploymentInUPaths{\np{\dgent_i,\dgent_{i+1}}}{\OrientedGraph}
  \cap \ActiveUndirectedPaths{\OrientedGraph} \)
  for $i\in\ic{0,n}$, where \( \orient_i \) is of the form~\( \np{-1,\ldots,+1}
  \) 
  \ifpreprint 
  and such that $\Internals_{\UPATH}\np{\UndirectedPath_i} \subset \TransitiveClosureParentalPrecedence \AgentSubsetW$
  for $i\in\ic{0,n}$%
  \fi
  .
  We define the \undirectedEdgePath\ 
  \( \UndirectedPath = \UndirectedPath_0 \ltimes \cdots \ltimes
  \UndirectedPath_n  \in
  \DeploymentInUPaths{\np{\bgent,\cgent}}{\OrientedGraph} \),
  by iterated (associative) concatenation as
  in~\eqref{eq:concatenation_UndirectedPath}, which is such that
  \( \cardinal{\UndirectedPath} \geq 2 \) and that
  \[ \UndirectedPath = \bp{\path_0 \ltimes \cdots \ltimes \path_n,
      \np{ \orient_0, \ldots, \orient_n} }
    % \bp{\np{-1,\ldots,+1 }, \ldots, \np{-1,\ldots,+1 } } }
    \mtext{ where }
    \orient_i \in \na{-1} \times \na{-1,+1}^{\cardinal{\path_i}-2} \times \na{+1}
    \eqsepv
    \forall i\in\ic{0,n}
    \eqfinp 
  \]
  We claim that \( \UndirectedPath \in
  \ActiveUndirectedPaths{\OrientedGraph} \).
  Indeed, \( \UndirectedPath_i \in \ActiveUndirectedPaths{\OrientedGraph} \)
  for $i\in\ic{0,n}$,
  so that, by
  Equation~\eqref{eq:ActiveUndirectedPaths_subpaths_junctions_when_reconcatenating} in
  Lemma~\ref{lem:ActiveUndirectedPaths_subpaths_junctions_when_reconcatenating},
  it only remains to show that, for $i\in\ic{0,n-1}$, 
  \begin{align}
    \Bp{ \bp{ \np{\tail{\vertex_{i}},\dgent_{i+1}},  \np{\dgent_{i+1}, \head{\vertex_{i+1}} } },
    \np{+1,-1} }
    &= 
      {\Head{\ProjectionUndirectedPath}\np{\UndirectedPath_{i}}
      \ltimes
      \Tail{\ProjectionUndirectedPath}\np{\UndirectedPath_{i+1}}}
      \in \ActiveUndirectedPaths{\OrientedGraph}
      \eqfinv
  \end{align}
  where 
  \( \np{\tail{\vertex}_{i},\dgent_{i+1}} \) is the last edge of the \undirectedEdgePath\ 
  \( \UndirectedPath_i = \np{\path_i, \orient_i } \) and 
  \( \np{\dgent_{i+1}, \head{\vertex}_{i+1}} \) is the first edge of the \undirectedEdgePath\ 
  \( \UndirectedPath_{i+1} =\np{\path_{i+1}, \orient_{i+1} } \).
  As \( \dgent_{i+1} \in \TransitiveReflexiveClosureOfSet{\AgentSubsetW} \), for $i\in\ic{0,n-1}$,
  and because of the orientation \( \np{-1,+1} \),
  all the above subpaths satisfy Item~\ref{it:ActivePaths_case4} in
  Definition~\ref{de:ActivePaths}. 
  % 
  % Now, the above subpath pattern corresponds to
  % none of the four cases in
  % Definition~\ref{de:ActivePaths},
  % because \( \dgent \in \Complementary{\AgentSubsetW} \)
  % for the first three cases,
  % and because of the orientation \( \np{-1,+1} \)
  % for the last fourth case.
  We conclude that \( \UndirectedPath = \UndirectedPath_0 \ltimes \cdots \ltimes \UndirectedPath_n \in 
  \DeploymentInUPaths{\np{\bgent,\cgent}}{\OrientedGraph}
  \cap \ActiveUndirectedPaths{\OrientedGraph} \).
  \ifpreprint
  Finally, from~\eqref{eq:Internals_PATH_concatenation}, we get that
  \begin{equation}
    \Internals_{\UPATH}\np{\UndirectedPath}
    = \underbrace{\Bp{ \bigcup_{i=0}^n
      \overbrace{\Internals_{\UPATH}\np{\UndirectedPath_i}}^{\subset
        \TransitiveClosureParentalPrecedence\AgentSubsetW
      }}}_{\subset \TransitiveClosureParentalPrecedence\AgentSubsetW
  \subset \complementary{\AgentSubsetW}}
    \cup
    \underbrace{\overbrace{ \na{\dgent_0,\ldots,\dgent_n}}^{\subset \TransitiveReflexiveClosureParentalPrecedence \AgentSubsetW}}
    _{\subset
      \TransitiveReflexiveClosureParentalPrecedence \AgentSubsetW
      = \TransitiveClosureParentalPrecedence \AgentSubsetW \cup \AgentSubsetW
    }
    \subset
    \TransitiveClosureParentalPrecedence\AgentSubsetW
    \cup 
    \TransitiveReflexiveClosureParentalPrecedence\AgentSubsetW
    \subset \TransitiveReflexiveClosureParentalPrecedence\AgentSubsetW
    \label{eq:InternalsofRhoasUnion}
  \end{equation}
  since \(\Internals_{\UPATH}\np{\UndirectedPath_i}\subset
  \TransitiveClosureParentalPrecedence \AgentSubsetW \) and
  $\delta_i \in \TransitiveReflexiveClosureParentalPrecedence \AgentSubsetW$
  for $i\in\ic{0,n}$. % The last statement comes from the fact that intermediate vertice which are in
  % $\AgentSubsetW$ must be in the subset $\na{\dgent_0,\ldots,\dgent_n}$ since $\TransitiveClosureParentalPrecedence \AgentSubsetW \subset \Complementary\AgentSubsetW$ and the vertices $\sequence{\delta_i}{i\in\ic{0,n}}$
  % satisfy the requested subpath property as given in Equation~\eqref{eq:subpat-property}. 
  \fi
  \medskip

  \ifpreprint
  \noindent $\bullet$ 
  Moreover, in the implication~\eqref{eq:BlockedUndirectedPaths_elementary_TransitiveClosure_TransitiveClosure},
  if a vertice $\dgent\in\Internals_{\UPATH}\np{\UndirectedPath} $ belongs
  to~$\AgentSubsetW$,
  it necessarily appears in a subpath of $\UndirectedPath$ of the form
  \(
  \Bp{ \bp{ \np{\tail{\vertex},\dgent},  \np{\dgent, \head{\vertex} }},
    \np{+1,-1} }
  \). 
  Indeed, using Equation~\eqref{eq:InternalsofRhoasUnion}, we obtain that
  a vertice $\dgent\in\Internals_{\UPATH}\np{\UndirectedPath} $ possibly belong
  to~$\AgentSubsetW$ only when $\delta \in \na{\dgent_1,\ldots,\dgent_n}$
  which gives the result.
  \medskip
  \fi 
      
  This ends the proof.
\end{proof}


\begin{lemma}
  Let \( \npOrientedGraph \) be a graph, 
  and $\AgentSubsetW\subset\VERTEX$ be a subset of vertices.
  Let \( \bgent, \cgent \in\AGENT \) be two vertices and assume that
  $\bgent \ConditionalActive \cgent$ where $\ConditionalActive$ is the conditional active
  relation~\eqref{eq:conditional_active_relation}.
  Then, there exists an active
  \undirectedEdgePath\, $\UndirectedPath$, joining the two vertices~$\bgent$ and $\cgent$%
  \ifpreprint
  ,
  whose intermediate vertices belong to
  \(\TransitiveClosureParentalPrecedence \na{\bgent,\cgent}
  \cup
  \TransitiveReflexiveClosureParentalPrecedence \AgentSubsetW
  \), that is, 
  \begin{align}
    \bgent \ConditionalActive \cgent
    \implies&
              \exists \UndirectedPath \in
              \DeploymentInUPaths{\na{\np{\bgent,\cgent}}}{\graph} \cap \ActiveUndirectedPaths{\graph}
              % \text{ s.t }
              % \Projection_{\UPATH}\np{\UndirectedPath}= (\bgent,\cgent)
              \nonumber 
    \\
            &\text{ and }
              \Internals_{\UPATH}\np{\UndirectedPath}
              \subset
              \TransitiveClosureParentalPrecedence \na{\bgent,\cgent}
              \cup
              \TransitiveReflexiveClosureParentalPrecedence \AgentSubsetW
              \eqfinp
              \label{eq:awtopath}
  \end{align}
  Moreover, an intermediate vertice $\vertex$ is in $\AgentSubsetW$ only if it appears
  in a $2$-length subpath of $\UndirectedPath$ of the form 
  \( \Bp{ \bp{\np{\vertex_{i-1},\vertex},\np{\vertex,\vertex_{i+1}}} , \np{+1,-1} } \).
  \else
  .
  \fi
  \label{lem:ConditionalDirectionalSeparation_seilpmi_relation-two}
\end{lemma}

\begin{proof}
  As the relation $\ConditionalActiveStar$ defined
  in~\eqref{eq:conditional_active_relationStar} equals the relation
  $\ConditionalActive$ in~\eqref{eq:conditional_active_relation}
  (as proved in Lemma~\ref{le:cond-act-equals-cond-act-star}),
  we suppose that we are given two vertices  \( \bgent, \cgent \in\AGENT \) such that
  $\bgent \ConditionalActiveStar \cgent$
  and we prove the existence of $\UndirectedPath \in
  \DeploymentInUPaths{\na{\np{\bgent,\cgent}}}{\graph} \cap \ActiveUndirectedPaths{\graph}$
  that joins $\bgent$ and $\cgent$%
  \ifpreprint
  and such that \( \Internals\np{\UndirectedPath}
  \subset
  \TransitiveClosureParentalPrecedence \na{\bgent,\cgent}
  \cup
  \TransitiveReflexiveClosureParentalPrecedence \AgentSubsetW
  \)
  \fi
  .
  By~\eqref{eq:conditional_active_relationStar}, giving
  $\ConditionalActiveStar$, 
  we have that 
  \[
    \bgent \bgp{\Delta \cup \ConditionalAscendent \cup
      \ConverseConditionalAscendent \cup \ConditionalCommonCause \cup
      \ConditionalActiveThreeStar} \cgent
    \eqfinp
  \]
  
  We consider the five cases, one by one.
  More precisely, for each case, we are going to show that
  there exists \(  \UndirectedPath \in \DeploymentInUPaths{\na{\np{\bgent,\cgent}}}{\graph} 
  \cap \ActiveUndirectedPaths{\graph} \)
  \ifpreprint
  such that
  $\Internals_{\UPATH}\np{\UndirectedPath}
  \subset
  \TransitiveClosureParentalPrecedence \na{\bgent,\cgent}
  \cup
  \TransitiveReflexiveClosureParentalPrecedence \AgentSubsetW$%
  \fi
  .
  
  The first case is \( \bgent \Delta \cgent \), that is, \(\bgent = \cgent\).
  As any \undirectedEdgePath\ of length~0 or~1 is active by definition
  (see the comment following Definition~\ref{de:ActivePaths}),
  we conclude that \( \bp{\np{\bgent,\cgent},+1} \in \DeploymentInUPaths{\na{\np{\bgent,\cgent}}}{\graph} 
  \cap \ActiveUndirectedPaths{\graph} \)%
  \ifpreprint
  and, as 
  $\Internals_{\UPATH}\bp{\bp{\np{\bgent,\cgent},+1}}=\emptyset$, the second assertion is satisfied
  \fi
  .
  
  The second case is \( \bgent \ConditionalAscendent \cgent \).
  % , that is,
  % \( \bgent \ConditionalDown \cgent \)
  % by~\eqref{eq:conditional_ascendent_relation}.
  We conclude that there exists \( \UndirectedPath\in \DeploymentInUPaths{\np{\bgent,\cgent}}{\graph} 
  \cap \ActiveUndirectedPaths{\graph} \)
  \ifpreprint
  such that
  $\Internals_{\UPATH}\np{\UndirectedPath} \subset \TransitiveClosureParentalPrecedence \cgent$
  \fi
  thanks
  to~\eqref{eq:BlockedUndirectedPaths_elementary_Delta_ComplementaryAgentSubsetW_EDGE}
  in Lemma~\ref{lem:BlockedUndirectedPaths_elementary}.
  
  The third case is \( \bgent \ConverseConditionalAscendent \cgent \).
  % , that is,
  % \( \bgent \ConditionalUp \cgent \)
  % by~\eqref{eq:converse_conditional_ascendent_relation}.
  We conclude that there exists \( \UndirectedPath\in \DeploymentInUPaths{\np{\bgent,\cgent}}{\graph} 
  \cap \ActiveUndirectedPaths{\graph} \)
  \ifpreprint
  such that
  $\Internals_{\UPATH}\np{\UndirectedPath} \subset \TransitiveClosureParentalPrecedence \bgent$
  \fi
  thanks
  to~\eqref{eq:BlockedUndirectedPaths_elementary_Converse_EDGE_Delta_ComplementaryAgentSubsetW}
  in Lemma~\ref{lem:BlockedUndirectedPaths_elementary}.
  
  The fourth case is \( \bgent \ConditionalCommonCause \cgent \).
  % , that is,
  % \( \bgent \TransitiveClosureConverseParentalPrecedence
  % \TransitiveClosureParentalPrecedence \cgent \)
  % by~\eqref{eq:common_cause}. 
  We conclude that there exists \( \UndirectedPath\in \DeploymentInUPaths{\np{\bgent,\cgent}}{\graph} 
  \cap \ActiveUndirectedPaths{\graph} \)
  \ifpreprint
  such that
  $\Internals_{\UPATH}\np{\UndirectedPath} \subset \TransitiveClosureParentalPrecedence
  \na{\bgent,\cgent}$
  \fi
  thanks
  to~\eqref{eq:BlockedUndirectedPaths_elementary_TransitiveClosure_Converse_EDGE_Delta_ComplementaryAgentSubsetW}
  in Lemma~\ref{lem:BlockedUndirectedPaths_elementary}.
  
  It remains to tackle the fifth case. 
  % \item 
  Suppose that $\bgent \ConditionalActiveThreeStar \cgent$.
  Therefore, there
  exist \( \dgent_1 \) and \( \dgent_2 \) in~$\VERTEX$ such that
  \[
    \bgent \bp{\ConditionalAscendent \cup \ConditionalCommonCause}\dgent_1
    \text{ and }
    \dgent_1 \CousinhoodStar \dgent_2
    \text{ and }
    \dgent_2 \bp{\ConverseConditionalAscendent \cup \ConditionalCommonCause} \cgent
    \eqfinp
  \]
  % We consider the two subcases 
  % \( \dgent_1 \Delta_{\TransitiveReflexiveClosureOfSet{\AgentSubsetW}}\dgent_2 \)
  % and
  % \( \dgent_1 \bp{\Delta_{\TransitiveReflexiveClosureOfSet{\AgentSubsetW}}
  % \ConditionalCommonCause
  % \Delta_{\TransitiveReflexiveClosureOfSet{\AgentSubsetW}}}^{+}\dgent_2 \).
  
  % First, we focus on $\dgent_1 
  % \Bp{
  % \bp{\Delta_{\TransitiveReflexiveClosureOfSet{\AgentSubsetW}} \ConditionalCommonCause \Delta_{\TransitiveReflexiveClosureOfSet{\AgentSubsetW}}}^{+}
  % \cup
  % \Delta_{\TransitiveReflexiveClosureOfSet{\AgentSubsetW}}
  % }\dgent_2$
  %   (and later on the left junction  with
  %   $\bgent \bp{\ConditionalAscendent \cup \ConditionalCommonCause}\dgent_1$
  %   and
  %   on the right junction with 
  %   $\dgent_2 \bp{\ConverseConditionalAscendent \cup \ConditionalCommonCause}
  %   \cgent$),
  %   with

  
  We % suppose that
  % \[
  %   \bgent \bp{\ConditionalAscendent \cup \ConditionalCommonCause}\dgent_1
  %   \text{ and }
  %   \dgent_1 \Delta_{\TransitiveReflexiveClosureOfSet{\AgentSubsetW}} \dgent_2
  %   \text{ and }
  %   \dgent_2 \bp{\ConverseConditionalAscendent \cup \ConditionalCommonCause} \cgent
  % \]
  % and we 
  are going to display an \undirectedEdgePath\ \( \UndirectedPath \in 
  \DeploymentInUPaths{\np{\dgent,\bgent}}{\graph} 
  \cap \ActiveUndirectedPaths{\graph} \).
  % 
  \begin{itemize}
  \item
    % As \( \dgent_1 \Delta_{\TransitiveReflexiveClosureOfSet{\AgentSubsetW}} \dgent_2
    % \), we get that
    % \( \dgent_1= \dgent_2 \in \TransitiveReflexiveClosureOfSet{\AgentSubsetW} \).
    % Thus, we are in the situation where there exists \( \dgent \in \TransitiveReflexiveClosureOfSet{\AgentSubsetW} \)
    % such that \( \bgent\bp{\ConditionalAscendent \cup \ConditionalCommonCause} \dgent\) and
    % \( \dgent \bp{\ConverseConditionalAscendent \cup \ConditionalCommonCause}
    % \cgent \).
    
    Considering the left hand side 
    \( \bgent\bp{\ConditionalAscendent \cup \ConditionalCommonCause} \dgent_1\) and
    using Lemma~\ref{lem:BlockedUndirectedPaths_elementary}, 
    we obtain --- either by~\eqref{eq:BlockedUndirectedPaths_elementary_Delta_ComplementaryAgentSubsetW_EDGE} applied to
    \( \bgent \ConditionalAscendent \dgent_1 \), or
    by~\eqref{eq:BlockedUndirectedPaths_elementary_TransitiveClosure_Converse_EDGE_Delta_ComplementaryAgentSubsetW}
    applied to \( \bgent \ConditionalCommonCause \dgent_1\) --- 
    that there exists
    \[
      \UndirectedPath_{1}= \np{\path_{1}, \orient_{1} }
      =\bp{\path_{1}, \np{\ldots,+1} } \in 
      \DeploymentInUPaths{\np{\bgent,\dgent_1}}{\graph} 
      \cap \ActiveUndirectedPaths{\graph}
      \ifpreprint  \eqfinv \else \eqfinp \fi
    \]
    \ifpreprint
    and we have \(\Internals_{\UPATH}\np{\UndirectedPath_{1}} \subset
    \TransitiveClosureParentalPrecedence \na{\bgent,\dgent_1}\).
    \fi
    

  \item 
    In the same way, considering the right hand side
    \( \dgent_2 \bp{\ConverseConditionalAscendent \cup \ConditionalCommonCause} \cgent \)
    and using Lemma~\ref{lem:BlockedUndirectedPaths_elementary}, 
    we obtain  --- either by~\eqref{eq:BlockedUndirectedPaths_elementary_Converse_EDGE_Delta_ComplementaryAgentSubsetW}
    applied to  \( \dgent_2 \ConverseConditionalAscendent \cgent \),
    or by~\eqref{eq:BlockedUndirectedPaths_elementary_TransitiveClosure_Converse_EDGE_Delta_ComplementaryAgentSubsetW}
    applied to \( \dgent_2 \ConditionalCommonCause \cgent \) --- 
    that there exists
    \[
      \UndirectedPath_{2} =\np{\path_{2}, \orient_{2} }
      =\bp{\path_{2}, \np{-1,\ldots} } \in 
      \DeploymentInUPaths{\np{\dgent_2,\bgent}}{\graph} 
      \cap \ActiveUndirectedPaths{\graph}
      \ifpreprint  \eqfinv \else \eqfinp \fi
    \]
    \ifpreprint
    and we have \(\Internals_{\UPATH}\np{\UndirectedPath_{2}} \subset
    \TransitiveClosureParentalPrecedence \na{\dgent_2,\cgent}\).
    \fi
  \item
    Considering the middle expression 
    \( \dgent_1 \CousinhoodStar \dgent_2 \) and 
    using Lemma~\ref{lem:BlockedUndirectedPaths_elementary}, 
    we obtain
    by~\eqref{eq:BlockedUndirectedPaths_elementary_TransitiveClosure_TransitiveClosure}
    that \( \dgent_1 \in \TransitiveReflexiveClosureOfSet{\AgentSubsetW} \), 
    \( \dgent_2 \in \TransitiveReflexiveClosureOfSet{\AgentSubsetW} \),
    and that there exists
    \[
      \UndirectedPath_{1,2}
      =\np{\path_{1,2}, \orient_{1,2} } =\bp{\path_{1,2}, \np{-1,\ldots,+1} } 
      \in        \DeploymentInUPaths{\np{\dgent_1,\dgent_2}}{\graph} 
      \cap \ActiveUndirectedPaths{\graph}
      \ifpreprint  \eqfinv \else \eqfinp \fi
    \]
    \ifpreprint
    and we have \(\Internals_{\UPATH}\np{\UndirectedPath_{1,2}} \subset
    \TransitiveReflexiveClosureParentalPrecedence \AgentSubsetW\).
    \fi
    % 
  \end{itemize}
  
  We consider the \undirectedEdgePath\ 
  \( \UndirectedPath = \UndirectedPath_1 \ltimes \UndirectedPath_{1,2} \ltimes \UndirectedPath_2 \)
  obtained by concatenation as in~\eqref{eq:concatenation_UndirectedPath}.
  By construction, we have that \( \UndirectedPath \in
  \DeploymentInUPaths{\np{\bgent,\cgent}}{\graph} \).
  We claim that \( \UndirectedPath \in
  \ActiveUndirectedPaths{\graph} \).
  Indeed, \( \UndirectedPath_1, \UndirectedPath_{1,2}, \UndirectedPath_2
  \in \ActiveUndirectedPaths{\graph} \) 
  by assumption,
  so that, by
  Equation~\eqref{eq:ActiveUndirectedPaths_subpaths_junctions_when_reconcatenating} in
  Lemma~\ref{lem:ActiveUndirectedPaths_subpaths_junctions_when_reconcatenating},
  it only remains to show that
  \begin{align}
    \Bp{ \bp{ \np{\tail{\vertex_{1}},\dgent_{1}},  \np{\dgent_{1}, \head{\vertex_{1,2}} } },
    \np{+1,-1} }
    &= 
      {\Head{\ProjectionUndirectedPath}\np{\UndirectedPath_{1}}
      \ltimes
      \Tail{\ProjectionUndirectedPath}\np{\UndirectedPath_{1,2}}}
      \in \ActiveUndirectedPaths{\OrientedGraph}
      \eqfinv
      \label{eq:dgent-one-property}
      \intertext{where 
      \( \np{\tail{\vertex_{1}},\dgent_{1}} \) is the last edge of the \undirectedEdgePath\ 
      \( \UndirectedPath_{1}\) and 
      \( \np{\dgent_{1}, \head{\vertex_{1,2}} } \) is the first edge of the \undirectedEdgePath\ 
      \( \UndirectedPath_{1,2} \), and that }
    % 
      \Bp{ \bp{  \np{\tail{\vertex_{1,2}},\dgent_{2} }, \np{\dgent_{2}, \head{\vertex_{2}} }  },
      \np{+1,-1} }
    &= 
      {\Head{\ProjectionUndirectedPath}\np{\UndirectedPath_{1,2}}
      \ltimes
      \Tail{\ProjectionUndirectedPath}\np{\UndirectedPath_{2}}}
      \in \ActiveUndirectedPaths{\OrientedGraph}
      \eqfinv
      \label{eq:dgent-two-property}
  \end{align}
  where 
  \( \np{\tail{\vertex_{1,2}},\dgent_{2}} \) is the last edge of the \undirectedEdgePath\ 
  \( \UndirectedPath_{1,2}\) and 
  \( \np{\dgent_{2}, \head{\vertex_{2}} } \) is the first edge of the \undirectedEdgePath\ 
  \( \UndirectedPath_{2} \). 
  As \( \dgent_1,  \dgent_2 \in \TransitiveReflexiveClosureOfSet{\AgentSubsetW} \)
  and because of the orientation \( \np{-1,+1} \),
  the two subpaths hereabove satisfy Item~\ref{it:ActivePaths_case4} in
  Definition~\ref{de:ActivePaths}. 
  We conclude that \( \UndirectedPath \in 
  \DeploymentInUPaths{\np{\bgent,\cgent}}{\graph}
  \cap \ActiveUndirectedPaths{\graph} \).%
  \ifpreprint
  Now, we have that
  \begin{align*}
    \Internals_{\UPATH}\np{\UndirectedPath}
    &=
      \Internals_{\UPATH}\np{\UndirectedPath_1} \cup \na{\dgent_1}
      \cup
      \Internals_{\UPATH}\np{\UndirectedPath_{1,2}} \cup \na{\dgent_2}
      \cup
      \Internals_{\UPATH}\np{\UndirectedPath_2}
    \\
    &\subset
      \TransitiveClosureParentalPrecedence \na{\bgent,\dgent_1}
      \cup
      \na{\dgent_1}
      \cup
      \TransitiveReflexiveClosureParentalPrecedence \AgentSubsetW
      \cup
      \na{\dgent_2}
      \cup
      \TransitiveClosureParentalPrecedence \na{\dgent_2,\cgent}
      \intertext{as \(\Internals_{\UPATH}\np{\UndirectedPath_{1}} \subset
      \TransitiveClosureParentalPrecedence \na{\bgent,\dgent_1}\),
      \(\Internals_{\UPATH}\np{\UndirectedPath_{1,2}} \subset
      \TransitiveReflexiveClosureParentalPrecedence \AgentSubsetW\) and
      \(\Internals_{\UPATH}\np{\UndirectedPath_{2}} \subset
      \TransitiveClosureParentalPrecedence \na{\dgent_2,\cgent}\)}      
    % \\
    &=
      \TransitiveClosureParentalPrecedence\bgent
      \cup
      \TransitiveClosureParentalPrecedence\dgent_1
      \cup
      \na{\dgent_1}
      \cup
      \TransitiveReflexiveClosureParentalPrecedence\AgentSubsetW
      \cup
      \na{\dgent_2}
      \cup
      \TransitiveClosureParentalPrecedence\dgent_2
      \cup
      \TransitiveClosureParentalPrecedence\cgent
    \\
    &=
      \TransitiveClosureParentalPrecedence\bgent
      \cup
      \TransitiveReflexiveClosureParentalPrecedence\dgent_1
      \cup
      \TransitiveReflexiveClosureParentalPrecedence\AgentSubsetW
      \cup
      \TransitiveReflexiveClosureParentalPrecedence\dgent_2
      \cup
      \TransitiveClosureParentalPrecedence\cgent
    \\
    & \subset
      \TransitiveClosureParentalPrecedence\bgent
      \cup
      \TransitiveReflexiveClosureParentalPrecedence\AgentSubsetW
      \cup
      \TransitiveClosureParentalPrecedence\cgent
      \tag{as        \( \dgent_1 \in \TransitiveReflexiveClosureOfSet{\AgentSubsetW} \), 
      \( \dgent_2 \in \TransitiveReflexiveClosureOfSet{\AgentSubsetW} \)}
    % \tag{as $\bgent$, $\cgent \in \TransitiveReflexiveClosureOfSet{\AgentSubsetW}$ }
    \\
    &=
      \TransitiveClosureParentalPrecedence \na{\bgent,\cgent}
      \cup \TransitiveReflexiveClosureParentalPrecedence \AgentSubsetW
      \eqfinp
  \end{align*}
  The last assertion of the Proposition --- namely, that
  an intermediate vertice $\vertex$ is in~$\AgentSubsetW$ only if it appears
  in a $2$-length subpath of $\UndirectedPath$ of the form 
  \( \Bp{ \bp{\np{\vertex_{i-1},\vertex},\np{\vertex,\vertex_{i+1}}} ,
    \np{+1,-1} } \) ---
  follows from the last assertion
  of Lemma~\ref{lem:BlockedUndirectedPaths_elementary}
  and from Equations~\eqref{eq:dgent-one-property} and~\eqref{eq:dgent-two-property}.
  \fi
  \medskip 
  
  This ends the proof.
  
\end{proof}

The following Proposition~\ref{pr:ConditionalDirectionalSeparation_seilpmi_relation}
is half of the proof of
Theorem~\ref{th:ConditionalDirectionalSeparation_IFF_relation}. 

\begin{proposition}
  Let \( \npOrientedGraph \) be a graph, 
  and $\AgentSubsetW\subset\VERTEX$ be a subset of vertices.
  % 
  Let \( \bgent, \cgent \in\AGENT \) be two vertices.
  We have the implication
  \begin{equation}
    \bgent \ConditionalActive \cgent \implies 
    \neg \bp{\bgent \ConditionalDirectionalSeparation \cgent \mid \AgentSubsetW}
    \label{eq:rel-to-d-sep}
  \end{equation}
  where $\ConditionalActive$ is the conditional active
  relation~\eqref{eq:conditional_active_relation}
  and $\ConditionalDirectionalSeparation$ is the
  conditional directional separation relation~\eqref{eq:vertices-d-separated}.
  % 
  \label{pr:ConditionalDirectionalSeparation_seilpmi_relation}  
\end{proposition}

\begin{proof} The proof is an easy consequence of
  Lemma~\ref{lem:ConditionalDirectionalSeparation_seilpmi_relation-two}.
 Let \( \bgent, \cgent \in\AGENT \) be two vertices.  We have that
  \begin{align*}
    \bgent \ConditionalActive \cgent
    &\implies
      \exists \UndirectedPath \in
      \DeploymentInUPaths{\na{\np{\bgent,\cgent}}}{\graph} \cap \ActiveUndirectedPaths{\graph}
      \nonumber
      \tag{by  Lemma~\ref{lem:ConditionalDirectionalSeparation_seilpmi_relation-two}}
    \\
    &\implies
      \neg \Bp{\bcDeploymentInUPaths{\na{\np{\bgent,\cgent}}}{\graph} \subset
      \BlockedUndirectedPaths{\graph}}
      \tag{by definition of $\BlockedUndirectedPaths{\graph}$}
    \\
    &\implies
      \neg \bp{\bgent \ConditionalDirectionalSeparation \cgent \mid \AgentSubsetW}
      \eqfinp
      \tag{by~\eqref{eq:vertices-d-separated} in Definition~\ref{de:vertices-d-separated}}
  \end{align*}
  This ends the proof.
\end{proof}

The following  Lemma~\ref{lem:induction-lemma} is instrumental in the second half of the proof of
Theorem~\ref{th:ConditionalDirectionalSeparation_IFF_relation}. 

\begin{lemma}
  Let \( \npOrientedGraph \) be a graph, 
  and $\AgentSubsetW\subset\VERTEX$ be a subset of vertices.
  % 
  For any $n\ge 2$ and $\bgent$, $\cgent \in \AGENT$, the following statement
  holds true.
  % be fixed. We say that
  % the induction assumption ${\mathbb H}'_n$ is satisfied by $(\bgent,\cgent)$ if
  For any
  \undirectedEdgePath\ \( \UndirectedPath = \np{\path,\orient} \in \UPATH^{n}\np{\graph} \) of length~$n$
  joining two vertices~$\bgent$ and $\cgent$ (that is,
  \( \BoundaryU{\na{\UndirectedPath}}{\graph}=\na{ \np{\bgent,\cgent} } \) as in~\eqref{eq:BoundaryU}), and
  such that \( \UndirectedPath \) is active (that is,
  \( \UndirectedPath \in \ActiveUndirectedPaths{\graph} \) as in
  Definition~\ref{de:ActivePaths}), 
  one of the two following properties is fullfiled:
  \begin{enumerate}
  \item
    Either $\bgent\ConditionalActivePlus \cgent$ and
    the last orientation of $\UndirectedPath$ is $\orient_n=+1$,
    \label{it:lem:induction-lemma_+1}
  \item
    Or $\bgent\ConditionalActiveMinus \cgent$ and
    the last orientation of $\UndirectedPath$ is $\orient_n=-1$.
    \label{it:lem:induction-lemma_-1}
  \end{enumerate}
  \label{lem:induction-lemma}
\end{lemma}


\begin{proof}
  We call ${\mathcal H}_{n}$ the statement in Lemma~\ref{lem:induction-lemma}
  and we prove by induction that it is satisfied for all $n\ge 2$.
  \medskip

  We prove ${\mathcal H}_{2}$. For this purpose, we consider an \undirectedEdgePath~$\UndirectedPath$ of length~$2$,
  joining two vertices $\bgent$ and $\cgent$ in the graph~$(\graph)$, and 
  which is active, that is, by definition~\eqref{eq:BoundaryU} of the
  endpoint relation~\( \BoundaryU{\SubsetPath}{\OrientedGraph} \) on~$\VERTEX$, 
  \begin{equation*}
    \UndirectedPath \in \UPATH^{2}\np{\graph}
    \text{ and }
    \BoundaryU{\na{\UndirectedPath}}{\graph}=\na{ \np{\bgent,\cgent} }
    \text{ and }
    \UndirectedPath \in \ActiveUndirectedPaths{\graph}
    \eqfinp 
  \end{equation*}
  Using Lemma~\ref{le:active-length-2}, there are two cases to consider.
  % 
  In the first case, the \undirectedEdgePath\ ends with orientation~$+1$  and is
  such that
  either \(\bgent  \Precedence \Delta_{\Complementary{\AgentSubsetW}}\Precedence  \cgent\)
  or \(\bgent   \Converse{\Precedence} \Delta_{\Complementary{\AgentSubsetW}} \Precedence  \cgent\).
  Using the properties that
  \( \Precedence \Delta_{\Complementary{\AgentSubsetW}}\Precedence
  \subset \ConditionalAscendent
  \subset \ConditionalActivePlus\)
  (by~\eqref{eq:conditional_ascendent_relation} and~\eqref{eq:ConditionalActivePlus})
  and that 
  \( \Converse{\Precedence} \Delta_{\Complementary{\AgentSubsetW}} \Precedence
  \subset \ConditionalCommonCause \subset \ConditionalActivePlus\)
  (by~\eqref{eq:common_cause} and~\eqref{eq:ConditionalActivePlus}),
  we obtain that
  the case~\ref{it:lem:induction-lemma_-1} of~${\mathcal H}_2$ is satisfied.
  % 
  In the second case, the \undirectedEdgePath\ path ends with orientation~$-1$ and is
  such that
  either 
  \( \bgent\Precedence \Delta_{\TransitiveReflexiveClosureOfSet{\AgentSubsetW}} \Converse{\Precedence}\cgent \)
  or \( \bgent \Converse{\Precedence} \Delta_{\Complementary{\AgentSubsetW}}\Converse{\Precedence} \cgent\).
  Using the properties that
  \(\Precedence \Delta_{\TransitiveReflexiveClosureOfSet{\AgentSubsetW}} \Converse{\Precedence}
  \subset   \ConditionalAscendent  \Delta_{\TransitiveReflexiveClosureOfSet{\AgentSubsetW}} \ConverseConditionalAscendent
  \subset \ConditionalActiveMinus \)
  (by~\eqref{eq:conditional_ascendent_relation}
  and~\eqref{eq:ConditionalActiveMinus})
  and that
  \(\Converse{\Precedence} \Delta_{\Complementary{\AgentSubsetW}}\Converse{\Precedence}
  \subset \ConverseConditionalAscendent\subset \ConditionalActiveMinus\)
  (by~\eqref{eq:converse_conditional_ascendent_relation}
  and~\eqref{eq:ConditionalActiveMinus}),
  we obtain that the case~\ref{it:lem:induction-lemma_-1} of~${\mathcal H}_2$ is
  satisfied.
  
  \medskip
  We suppose that the induction assumption~${\mathcal H}_{n-1}$ holds true, where $n-1 \ge 2$,
  and we are going to show that ${\mathcal H}_{n}$ holds true. 
  For this purpose, we consider an \undirectedEdgePath~$\UndirectedPath$ of length~$n$ ($n \geq 2$), 
  joining two vertices $\bgent$ and $\cgent$ in the graph~$(\graph)$, and 
  which is active, that is, 
  \begin{equation*}
    \UndirectedPath \in \UPATH^{n}\np{\graph}
    \text{ and }
    \BoundaryU{\na{\UndirectedPath}}{\graph}=\na{ \np{\bgent,\cgent} }
    \text{ and }
    \UndirectedPath \in \ActiveUndirectedPaths{\graph}
    \eqfinp 
  \end{equation*}
  We decompose the \undirectedEdgePath~$\UndirectedPath$ as
  \[
    \UndirectedPath=\np{\path,\orient}=\bp{\nseqp{(\tail{\vertex_i},\head{\vertex_i})}{i \in
        \ic{1,n}},\nseqp{\orient_i}{i \in \ic{1,n}}}
    = \UndirectedPath_{n{-}1} \ltimes \overline{\UndirectedPath}
    \eqfinv
  \]
  where
  \(
  \UndirectedPath_{n{-}1}=\bp{\nseqp{(\tail{\vertex_i},\head{\vertex_i})}{i \in
      \ic{1,n{-}1}},\nseqp{\orient_i}{i \in \ic{1,n{-}1}}}
  \in \UPATH^{n{-}1}\np{\graph}
  \)
  is an \undirectedEdgePath\ of length~$n{-}1$,
  and where $\overline{\UndirectedPath}=\bp{\np{\tail{\vertex_n},\head{\vertex_n}}, \orient_{n}}\in \UPATH^{1}\np{\graph}$
  is an \undirectedEdgePath\ of length~$1$.
  We have that \( \tail{\vertex_1}=\bgent \) and \( \head{\vertex_n}=\cgent \).
  It is clear that the \undirectedEdgePath~$\UndirectedPath_{n{-}1}$ is active, 
  that is, \(   \UndirectedPath_{n{-}1} \in \ActiveUndirectedPaths{\graph}\).
  Indeed, otherwise, the \undirectedEdgePath~$\UndirectedPath_{n{-}1}$
  would be in one of the four cases listed in 
  Definition~\ref{de:vertices-d-separated}, hence so would be the \undirectedEdgePath~$\UndirectedPath$.
  But this would contradict the assumption that
  \(  \UndirectedPath \in \ActiveUndirectedPaths{\graph} \).
  As the \undirectedEdgePath~$\UndirectedPath_{n{-}1}$ is active and of length~$n{-}1$,
  it satisfies the induction assumption~${\mathcal H}_{n-1}$. We deduce that
  either $\bgent \ConditionalActivePlus \head{\vertex_{n-1}}$
  and the last orientation of $\UndirectedPath_{n-1}$ is
  $\orient_{n-1}=+1$, or
  $\bgent \ConditionalActiveMinus \head{\vertex_{n-1}}$ and the last orientation of $\UndirectedPath_{n-1}$ is
  $\orient_{n-1}=-1$.
  We analyze the two cases separately.

  \medskip
  \noindent $\bullet$
  Assume that we have
  $\bgent \ConditionalActivePlus \head{\vertex_{n-1}}$ and that
  the last orientation of~$\UndirectedPath_{n-1}$ is $\orient_{n-1}=+1$,
  that is, $\UndirectedPath_{n-1}$ ends with \(
  \bp{\np{\tail{\vertex_{n-1}},\head{\vertex_{n-1}}}, +1} \). 
  There are two possibilities for the
  \undirectedEdgePath~$\overline{\UndirectedPath}=\bp{\np{\tail{\vertex_n},\head{\vertex_n}}, \orient_{n}}$.
  
  \begin{itemize}
    
  \item
    Suppose that $\overline{\UndirectedPath}=\bp{\np{\tail{\vertex_n},\head{\vertex_n}}, +1
    }= \bp{\np{\tail{\vertex_{n}},\cgent}, +1}$, that
    is, \( \np{\tail{\vertex_{n}},\cgent} \in \Precedence \) by~\eqref{eq:undirected_(edge)_paths_of_length_n}.
    % 
    As the path $\UndirectedPath$ is active by assumption, the pattern
    \( \Bp{ \bp{\np{\tail{\vertex_{n-1}},\head{\vertex_{n-1}}}, +1},
      \bp{\np{\tail{\vertex_n},\head{\vertex_n}}, +1} } \) must satisfy Item~\ref{it:ActivePaths_case1} in
    Definition~\ref{de:ActivePaths}. We deduce that 
    $\head{\vertex_{n-1}}=\tail{\vertex_{n}} \in \Complementary{\AgentSubsetW}$. 
    Now, we wrap up the results obtained so far.
    % 
    On the one hand, from $\bgent \ConditionalActivePlus \head{\vertex_{n-1}}$,
    $\head{\vertex_{n-1}}=\tail{\vertex_{n}} \in \Complementary{\AgentSubsetW}$
    and \( \np{\tail{\vertex_{n}},\cgent} \in \Precedence \),
    we get that  $\bgent\ConditionalActivePlus
    \Delta_{\Complementary{\AgentSubsetW}}
    \Precedence\cgent$, hence that 
    $\bgent \ConditionalActivePlus\cgent$ because
    the relation~$\ConditionalActivePlus$ in~\eqref{eq:ConditionalActivePlus}
    ends with the relation
    $\ConditionalAscendent$ and as 
    $\ConditionalAscendent \Delta_{\Complementary{\AgentSubsetW}}\Precedence
    =\ConditionalDown (\Delta_{\Complementary{\AgentSubsetW}}\Precedence)
    \subset \ConditionalAscendent$ 
    by~\eqref{eq:conditional_ascendent_relation}.
    % 
    On the other hand, the \undirectedEdgePath~$\UndirectedPath$ ends with~$+1$,
    as it is the case for~$\overline{\UndirectedPath}$.
    % 
    We conclude that the \undirectedEdgePath~$\UndirectedPath$ of length~$n$
    satisfies the case~\ref{it:lem:induction-lemma_+1} of~${\mathcal H}_{n}$,
    since it ends with~$+1$ and its endpoints are such 
    that $\bgent \ConditionalActiveMinus\cgent$.
    Therefore, we have proven the case~\ref{it:lem:induction-lemma_+1} of
    the induction assumption~${\mathcal H}_{n}$
    for the \undirectedEdgePaths\ of length~$n$.
    % 
  \item
    Suppose that $\overline{\UndirectedPath}=
    \bp{\np{\tail{\vertex_n},\head{\vertex_n}}, -1}=
    \bp{\np{\tail{\vertex_{n}},\cgent}, -1}$, that
    is, \( \np{\tail{\vertex_{n}},\cgent} \in \Converse{\Precedence} \)
    by~\eqref{eq:undirected_(edge)_paths_of_length_n}.
    % 
    As the path $\UndirectedPath$ is active, the pattern
    \( \Bp{ \bp{\np{\tail{\vertex_{n-1}},\head{\vertex_{n-1}}}, +1},
      \bp{\np{\tail{\vertex_n},\head{\vertex_n}}, -1} } \) must satisfy Item~\ref{it:ActivePaths_case4} in
    Definition~\ref{de:ActivePaths}. We deduce that 
    $\head{\vertex_{n-1}}=\tail{\vertex_{n}} \in \TransitiveReflexiveClosureOfSet{\AgentSubsetW}$.
    Now, we wrap up the results obtained so far.
    % 
    On the one hand, from $\bgent \ConditionalActivePlus \head{\vertex_{n-1}}$,
    $\head{\vertex_{n-1}}=\tail{\vertex_{n}} \in \TransitiveReflexiveClosureOfSet{\AgentSubsetW}$
    and \( \np{\tail{\vertex_{n}},\cgent} \in \Converse{\Precedence} \),
    we get that  $\bgent\ConditionalActivePlus
    \Delta_{\TransitiveReflexiveClosureOfSet{\AgentSubsetW}}
    \Converse{\Precedence}\cgent$, hence that 
    $\bgent \ConditionalActiveMinus\cgent$  by~\eqref{eq:ConditionalActiveMinus_supset}.
    % 
    On the other hand, the \undirectedEdgePath~$\UndirectedPath$ ends with~$-1$,
    as it is the case for~$\overline{\UndirectedPath}$.
    % 
    We conclude that the \undirectedEdgePath~$\UndirectedPath$ of length~$n$
    satisfies the case~\ref{it:lem:induction-lemma_-1} of~${\mathcal H}_{n}$,
    since it ends with~$-1$ and its endpoints are such 
    that $\bgent \ConditionalActiveMinus\cgent$.
    Therefore, we have proven the case~\ref{it:lem:induction-lemma_-1} of
    the induction assumption~${\mathcal H}_{n}$
    for the \undirectedEdgePaths\ of length~$n$.
    % 
  \end{itemize}

  \medskip
  \noindent $\bullet$
  Assume that we have
  $\bgent \ConditionalActiveMinus \head{\vertex_{n-1}}$
  and that the last orientation of~$\UndirectedPath_{n-1}$ is
  $\orient_{n-1}=-1$, that
  is, \( \np{\tail{\vertex_{n}},\cgent} \in \Converse{\Precedence} \). 
  There are two possibilities for the 
  \undirectedEdgePath~$\overline{\UndirectedPath}=\bp{\np{\tail{\vertex_n},\head{\vertex_n}},
    \orient_{n}}$.
  
  \begin{itemize}
  \item
    Suppose that $\overline{\UndirectedPath}=
    \bp{\np{\tail{\vertex_n},\head{\vertex_n}}, +1}= \bp{\np{\tail{\vertex_{n}},\cgent}, +1}$, that
    is, \( \np{\tail{\vertex_{n}},\cgent} \in \Precedence \) by~\eqref{eq:undirected_(edge)_paths_of_length_n}.
    As the path $\UndirectedPath$ is active, the pattern
    \( \Bp{ \bp{\np{\tail{\vertex_{n-1}},\head{\vertex_{n-1}}}, -1},
      \bp{\np{\tail{\vertex_n},\head{\vertex_n}}, +1} } \) must satisfy Item~\ref{it:ActivePaths_case3} in
    Definition~\ref{de:ActivePaths}. We deduce that 
    $\head{\vertex_{n-1}}=\tail{\vertex_{n}} \in \Complementary{\AgentSubsetW}$. 
    Now, we wrap up the results obtained so far.
    % 
    On the one hand, from $\bgent \ConditionalActiveMinus \head{\vertex_{n-1}}$,
    $\head{\vertex_{n-1}}=\tail{\vertex_{n}} \in \Complementary{\AgentSubsetW}$
    and \( \np{\tail{\vertex_{n}},\cgent} \in \Precedence \),
    we get that  $\bgent\ConditionalActiveMinus
    \Delta_{\Complementary{\AgentSubsetW}}
    \Converse{\Precedence}\cgent$, hence that 
    $\bgent \ConditionalActivePlus\cgent$  by~\eqref{eq:ConditionalActiveMinus_supset}.
    % 
    On the other hand, the \undirectedEdgePath~$\UndirectedPath$ ends with~$+1$,
    as it is the case for~$\overline{\UndirectedPath}$.
    % 
    We conclude that the \undirectedEdgePath~$\UndirectedPath$ of length~$n$
    satisfies the case~\ref{it:lem:induction-lemma_+1} of~${\mathcal H}_{n}$,
    since it ends with~$+1$ and its endpoints are such 
    that $\bgent \ConditionalActivePlus\cgent$.
    Therefore, we have proven the case~\ref{it:lem:induction-lemma_+1} of
    the induction assumption~${\mathcal H}_{n}$
    for the \undirectedEdgePaths\ of length~$n$.
    % 
  \item
    Suppose that $\overline{\UndirectedPath}=
    \bp{\np{\tail{\vertex_n},\head{\vertex_n}}, -1}= \bp{\np{\tail{\vertex_{n}},\cgent}, -1}$, that
    is, \( \np{\tail{\vertex_{n}},\cgent} \in \Converse{\Precedence} \)
    by~\eqref{eq:undirected_(edge)_paths_of_length_n}.
    % 
    As the path $\UndirectedPath$ is active, the pattern
    \( \Bp{ \bp{\np{\tail{\vertex_{n-1}},\head{\vertex_{n-1}}}, -1},
      \bp{\np{\tail{\vertex_n},\head{\vertex_n}}, -1} } \) must satisfy Item~\ref{it:ActivePaths_case2} in
    Definition~\ref{de:ActivePaths}. We deduce that 
    $\head{\vertex_{n-1}}=\tail{\vertex_{n}} \in \Complementary{\AgentSubsetW}$. 
    Now, we wrap up the results obtained so far.
    % 
    On the one hand, from $\bgent \ConditionalActiveMinus \head{\vertex_{n-1}}$,
    $\head{\vertex_{n-1}}=\tail{\vertex_{n}} \in \Complementary{\AgentSubsetW}$
    and \( \np{\tail{\vertex_{n}},\cgent} \in \Converse{\Precedence} \),
    we get that  $\bgent\ConditionalActiveMinus
    \Delta_{\Complementary{\AgentSubsetW}}
    \Converse{\Precedence}\cgent$, hence that 
    $\bgent \ConditionalActiveMinus\cgent$ because
    the relation~$\ConditionalActiveMinus$ ends with the relation
    $\ConverseConditionalAscendent$ and we have that
    $\ConverseConditionalAscendent \Delta_{\Complementary{\AgentSubsetW}}\Converse{\Precedence}
    = \ConditionalUp\Delta_{\Complementary{\AgentSubsetW}}\Converse{\Precedence}
    \subset \ConverseConditionalAscendent$  which implies that
    $\ConditionalActiveMinus \Delta_{\Complementary{\AgentSubsetW}}\Converse{\Precedence}
    \subset \ConditionalActiveMinus$. 
    % 
    On the other hand, the \undirectedEdgePath~$\UndirectedPath$ ends with~$-1$,
    as it is the case for~$\overline{\UndirectedPath}$.
    % 
    We conclude that the \undirectedEdgePath~$\UndirectedPath$ of length~$n$
    satisfies the case~\ref{it:lem:induction-lemma_-1} of~${\mathcal H}_{n}$,
    since it ends with~$-1$ and its endpoints are such 
    that $\bgent \ConditionalActiveMinus\cgent$.
    Therefore, we have proven the case~\ref{it:lem:induction-lemma_-1} of
    the induction assumption~${\mathcal H}_{n}$
    for the \undirectedEdgePaths\ of length~$n$.
    % 
  \end{itemize}
  \medskip

  This ends the proof. 
\end{proof}

\ifpreprint 
\section{Moral relation}
\label{Moral_relation}

% \subsection{Definition of conditional moral separation}

Let \( \npOrientedGraph \) be a graph.
The \emph{moral relation}%associated with~$\Precedence$ is the relation
~$\Moral$ on $\np{\VERTEX,\Precedence}$ is defined in~\citep{Lauritzen-et-al-1990} by
\begin{equation}
  \Moral= \bp{\Precedence\cup \Precedence\Converse{\Precedence}} \cup
  \bp{\Converse\Precedence \cup \Precedence\Converse{\Precedence}}
  \subset \Precedence\times\Precedence
  \eqfinp
  \label{eq:moral-definition}
\end{equation}

In Proposition~\ref{pr:from-active-path-to-moral-graph}, which is a consequence
of Lemma~\ref{lem:ConditionalDirectionalSeparation_seilpmi_relation-two},
we prove that, when two vertices $\bgent$ and $\cgent$ are in relation by the
conditional active relation~\eqref{eq:conditional_active_relation}, then 
there exists an (edge) path in the graph \( (\VERTEX, \Moral) \)
joining the two vertices $\bgent$ and $\cgent$ 
with intermediate vertices in 
$\TransitiveClosureParentalPrecedence
\bp{\na{\bgent,\cgent}\cup\AgentSubsetW}$.

\begin{proposition}
  \label{pr:from-active-path-to-moral-graph}
    Let \( \npOrientedGraph \) be a graph, 
  and $\AgentSubsetW\subset\VERTEX$ be a subset of vertices.  
  Let $\ConditionalActive$ be the conditional active
  relation~\eqref{eq:conditional_active_relation}
  and $\Moral$ be the relation~\eqref{eq:moral-definition}. 
  Then, for all $\bgent, \cgent \in \VERTEX$, we have that
  \begin{align}
    \bgent \ConditionalActive \cgent
    \implies&
              \exists \path \in
              \DeploymentInPaths{\na{\np{\bgent,\cgent}}}{\moralgraph}
              \text{ and  }
              \Internals\np{\path}
              \subset
              \TransitiveClosureParentalPrecedence \bp{\na{\bgent,\cgent}\cup\AgentSubsetW}
              \eqfinp
              \label{eq:awtopathMoral}
  \end{align}
\end{proposition}

%\jpc{is it clear now ?}
\begin{proof}
  Let  $\bgent, \cgent \in \VERTEX$. Using
  Lemma~\ref{lem:ConditionalDirectionalSeparation_seilpmi_relation-two},
  there exists an active \undirectedEdgePath\ $\UndirectedPath$ joining the two
  vertices~$\bgent$ and $\cgent$ and such that
  $\Internals_{\UPATH}\np{\UndirectedPath} \subset
  \TransitiveClosureParentalPrecedence \na{\bgent,\cgent} \cup
  \TransitiveReflexiveClosureParentalPrecedence \AgentSubsetW$.
%  \jpc{I use $\vertex$ as the notation $\delta$ is used for the special vertices of Lemma~\ref{lem:ConditionalDirectionalSeparation_seilpmi_relation-two}}
  We denote by
  $\sequence{\vertex_i}{i=0,\ldots,\cardinal{\UndirectedPath}}$ the sequence of
  intermediate vertices of $\UndirectedPath$.  For
  $i=0,\ldots,\cardinal{\UndirectedPath}-1$, we have that $\vertex_i \Moral \vertex_{i+1}$
  as, by definition~\eqref{eq:undirectedEdgePaths} of an \undirectedEdgePath, we have that 
  $\vertex_i \Precedence \vertex_{i+1}$ or
  $\vertex_i \Converse\Precedence \vertex_{i+1}$.  In the same way, we have that
  $\bgent \Moral \vertex_0$ and $\vertex_{\cardinal{\UndirectedPath}} \Moral \cgent$.
  Thus, we build an (edge) path 
  \( \path'=\sequence{ \np{\vertex_i,\vertex_{i+1}} }{i\in\ic{1,\cardinal{\UndirectedPath}-1}} \)
  which belongs to $\DeploymentInPaths{\na{\np{\bgent,\cgent}}}{\moralgraph}$ as defined in Equation~\eqref{eq:DeploymentInPaths}.
  However, the (edge) path $\path'$ does not satisfy the assumption 
  $\Internals\np{\path'}
  \subset
  \TransitiveClosureParentalPrecedence \bp{\na{\bgent,\cgent}\cup\AgentSubsetW}$ as some intermediate vertex 
  of $\path'$ may belong to $\AgentSubsetW$. But using the last statement of 
  Lemma~\ref{lem:ConditionalDirectionalSeparation_seilpmi_relation-two}, we
  have that, given an intermediate vertex of $\path'$, $\vertex_i \in \AgentSubsetW$ is only possible when $\vertex_i$ is a
  ``$\delta$'' vertex, that is, $\vertex_i$ appears in a subpath of~$\UndirectedPath$ of the form
  \( \Bp{ \bp{\np{\vertex_{i-1},\vertex_i},\np{\vertex_i,\vertex_{i+1}}} , \np{+1,-1} } \) which implies that
  $\vertex_{i-1} \Precedence \vertex_i$ and
  $\vertex_{i} \Converse\Precedence \vertex_{i+1}$ hence that
  $\vertex_{i-1} \Moral \vertex_{i+1}$ by definition of the moral relation
  in~\eqref{eq:moral-definition}.
  We define the \EdgePath~$\path''$ as obtained from the \EdgePath~$\path'$ by
  removing all the ``$\delta$'' vertices.
  We readily get that $\path''$ belongs, as does~$\path'$, to
  $\DeploymentInPaths{\na{\np{\bgent,\cgent}}}{\moralgraph}$
  and has the same internal vertices in $\Complementary\AgentSubsetW$ as does~$\path'$.
  % Thus $\path$, the \EdgePath\ obtained from $\path'$ by removing all the ``$\delta$'' vertices remains in
  % $\DeploymentInPaths{\na{\np{\bgent,\cgent}}}{\moralgraph}$ and have internal vertices
  % in $\Complementary\AgentSubsetW$.
  We conclude that
  \[
    \Internals\np{\path''} \subset
    \Internals\np{\path'} \cap \Complementary\AgentSubsetW
    \subset 
    \bp{\TransitiveClosureParentalPrecedence \na{\bgent,\cgent} \cup
      \TransitiveReflexiveClosureParentalPrecedence \AgentSubsetW}\cap \Complementary\AgentSubsetW
    \subset
    \TransitiveClosureParentalPrecedence{\na{\bgent,\cgent}}
    \cup
    \TransitiveClosureParentalPrecedence \AgentSubsetW
    = \TransitiveClosureParentalPrecedence{\bp{\na{\bgent,\cgent}\cup\AgentSubsetW}}
    \eqfinp
  \]
  This ends the proof.
\end{proof}


Let $\AgentSubsetW$, $\Bgent$, $\Cgent \subset \AGENT$ be three subsets of
vertices, that are pairwise disjoints.  We define the subset 
\( \MoralSubset= \TransitiveReflexiveClosure{\Precedence}
\np{\Bgent\cup\Cgent\cup\AgentSubsetW} \subset \AGENT \)
of vertices, and the binary relation
\( \Precedence_\MoralSubset= \Precedence \cap \np{\MoralSubset\times\MoralSubset}
\) on~\( \MoralSubset \). 
The subgraph
$(\MoralSubset,\Precedence_\MoralSubset)$ is composed of the set of vertices
$\MoralSubset$ and of the set~$\Precedence_\MoralSubset$ of edges, given by the
restriction of the relation~$\Precedence$ to the subset~$\MoralSubset$.
The \emph{moral relation}~$\Moral_{\MoralSubset}$ on the subgraph
$(\MoralSubset,\Precedence_\MoralSubset)$ %associated with~$\Precedence_{\MoralSubset}$
is the subset of~$\MoralSubset\times\MoralSubset$ defined by
\begin{equation}
  \Moral_{\MoralSubset} =
  \bp{\MoralPrecedence\cup \MoralPrecedence\Converse{\MoralPrecedence}} \cup
  \bpConverse{\MoralPrecedence \cup \MoralPrecedence\Converse{\MoralPrecedence}}
  \subset \MoralSubset\times\MoralSubset
  \eqfinp
  \label{eq:moral-subset-definition}
\end{equation}

Now, under the assumptions of
Proposition~\ref{pr:from-active-path-to-moral-graph}
and the additional assumption that
$\bgent \in \Bgent$ and $\cgent \in \Cgent$, 
we obtain a stronger result.
% that there exists an
% active \undirectedEdgePath\ $\UndirectedPath$ joining them in the subgraph
% $(\MoralSubset,\Precedence_\MoralSubset)$.

\begin{corollary}
  Let $\AgentSubsetW$, $\Bgent$, $\Cgent \subset \AGENT$ be three subsets of
vertices, that are pairwise disjoints.
  Let $\bgent \in \Bgent$ and $\cgent \in \Cgent$ be given. Then, we have that 
  \begin{align}
    \bgent \ConditionalActive \cgent
    &\implies
      \exists \path \in
      \DeploymentInPaths{\na{\np{\bgent,\cgent}}}{\MoralSubset,\Moral_{\MoralSubset}}
      \text{ and  }
      \Internals\np{\path}
      \subset \complementary{\AgentSubsetW}
      \eqfinp
      \label{co:active-to-moral-deploy}
  \end{align}
\end{corollary}

\begin{proof} The proof is a simple consequence of
  Proposition~\ref{pr:from-active-path-to-moral-graph} as the path 
  obtained in Equation~\eqref{eq:awtopathMoral} is a path in the subgraph 
  $(\MoralSubset,\Precedence_\MoralSubset)$ and its intermediate vertices
  satisfy 
  $\Internals\np{\path}
  \subset
  \TransitiveClosureParentalPrecedence \bp{\na{\bgent,\cgent}\cup\AgentSubsetW}
  \subset \Complementary{\AgentSubsetW}$.
\end{proof}

\begin{proposition}
  \label{pr:lauritzen-2}
      Let \( \npOrientedGraph \) be a graph, 
 and let $\AgentSubsetW$, $\Bgent$, $\Cgent \subset \AGENT$ be three subsets of
  vertices, that are pairwise disjoints.
  We have the implication:
  \begin{align}
    \exists \path \in
    \DeploymentInPaths{\Bgent\times\Cgent}{\bp{\MoralSubset,\Moral_{\MoralSubset}}}
    &
      \text{ such that }
      \Internals\np{\path}
      \subset
      \Complementary\AgentSubsetW
      \nonumber % \label{eq:s-ms-active}
    \\
    &\implies
      \exists \UndirectedPath \in
      \DeploymentInUPaths{\Bgent\times\Cgent}{\graph} \cap \ActiveUndirectedPaths{\graph}
      \eqfinp
      \label{eq:awtopath-ter}
  \end{align}
\end{proposition}


\begin{proof} Let
  $\path \in
  \DeploymentInPaths{\Bgent\times\Cgent}{\bp{\MoralSubset,\Moral_{\MoralSubset}}}$
  such that
  $\Internals\np{\path} \subset \Complementary\AgentSubsetW$.
  Then, there exists a sequence
  $\sequence{\vertex_i}{i\in\ic{1,\cardinal{\path}+1}}$ of vertices in $\MoralSubset$ such that
  \( \path=\sequence{ \np{\vertex_i,\vertex_{i+1}}
  }{i\in\ic{1,\cardinal{\path}}}\) with $\vertex_1 \in \Bgent$,
  $\vertex_{\cardinal{\path}+1}\in \Cgent$ and such that
  $\vertex_i \Moral_{\MoralSubset} \vertex_{i+1}$ and
  $\vertex_i \in \Complementary \AgentSubsetW$ for all
  $i=2,\ldots,\cardinal{\path}$. Using the definition of $\Moral_{\MoralSubset}$,
  there are two cases: either for all
  $i=2,\ldots,\cardinal{\path}$ there exists $\orient_i \in \na{-1,+1}$ such that
  $\vertex_i \Precedence^{\np{\orient_i}} \vertex_{i+1}$ and, in that case, we
  build the \undirectedEdgePath\
  $\UndirectedPath_i = \bp{(\vertex_i,\vertex_{i+1}),\orient_i}$;
  or there exists $\dgent_i \in \MoralSubset$ such that
  $\vertex_i \Precedence^{\np{1}} \dgent_i \Precedence^{\np{-1}}\vertex_{i+1}$
  and, in that case, we build the \undirectedEdgePath\ 
  $\UndirectedPath_i=\Bp{\bp{(\vertex_i,\dgent_i),(\dgent_i,\vertex_{i+1})},\na{1,-1}}$
  of length~$2$.
  By concatenation, we obtain the \undirectedEdgePath\
  $\UndirectedPath = \UndirectedPath_1 \ltimes \ldots \ltimes
  \UndirectedPath_{\cardinal{\path}}$.

  It remains to show that the \undirectedEdgePath~$\UndirectedPath$ is active.
  As, by assumption, the vertices
  $\sequence{\vertex_i}{i\in\ic{1,\cardinal{\path}+1}}$ are in $\Complementary
  \AgentSubsetW$, then Definition~\ref{de:ActivePaths} implies that $\UndirectedPath$
  can only be blocked at the vertices appearing in $2$-length
  \undirectedEdgeSubPath\ of the form
  $\Bp{\bp{(\vertex,\dgent),(\dgent,\vertex')},\na{1,-1}}$ with
  $\dgent \in \MoralSubset$.
  If $\dgent \in \TransitiveReflexiveClosure{\Precedence}\AgentSubsetW$, then the
  subpath is active.  If $\dgent \in \MoralSubset$ and
  $\dgent \not\in \TransitiveReflexiveClosure{\Precedence}\AgentSubsetW$, then
  it easily follows that
  $\dgent \in \TransitiveClosureParentalPrecedence \bp{\Bgent \cup \Cgent} =
  \TransitiveClosureParentalPrecedence \Bgent \cup
  \TransitiveClosureParentalPrecedence\Cgent $ and the subpath is blocked. We
  denote by $n$ the number of blocked subpaths of length~2 of $\UndirectedPath$
  for which %the middle vertice
  $\dgent \in \TransitiveClosureParentalPrecedence \Bgent \cup
  \TransitiveClosureParentalPrecedence\Cgent$ and we prove that, when $n >0$, we
  can build a new~$\UndirectedPath$ which has a strictly smaller number of
  blocked subpaths. We consider only the case where the vertice at the center of the blocked subpath of length 2
  is such that $\dgent \in  \TransitiveClosureParentalPrecedence \Bgent$ as the case
  $\dgent \in  \TransitiveClosureParentalPrecedence \Cgent$ can be treated in a similar way.
  Denote by $\UndirectedPath'$ the subpath of $\UndirectedPath$ starting at $\dgent$ and ending
  in $\Cgent$. The subpath $\UndirectedPath$ contains a number of blocked
  subpaths which is stricly smaller than
  $n$. As $\dgent \in  \TransitiveClosureParentalPrecedence \Bgent$, there exists
  an active \undirectedEdgeSubPath\ $\UndirectedPath''$ starting in~$\Bgent$ and ending at~$\dgent$.
  Then, by concatenation, we obtain that $\UndirectedPath'' \ltimes \UndirectedPath' \in 
  \DeploymentInUPaths{\Bgent\times\Cgent}{\graph}$ for which the number of  blocked subpaths 
  is strictly smaller that~$n$, since it is the case for $\UndirectedPath'$, and since
  $\UndirectedPath''$ does not contain blocked subpaths and 
  the connection at node~$\dgent$ is active in the concatenated path $\UndirectedPath'' \ltimes \UndirectedPath'$.
  We can iterate the process a finite number of times (at most $n$) to obtain at the end
  an active \undirectedEdgeSubPath.
  This ends the proof.
\end{proof}

We relate in Proposition~\ref{p:conditionalactiveinmoral} the conditional active relation 
$\ConditionalActive$ to the moral relation $\Moral_{\MoralSubset}$ defined on the subset 
$\MoralSubset$ in Equation~\eqref{eq:moral-subset-definition}.

\begin{proposition}
  \label{p:conditionalactiveinmoral}
  Let \( \npOrientedGraph \) be a graph, 
  and let $\AgentSubsetW$, $\Bgent$, $\Cgent \subset \AGENT$ be three subsets of
  vertices, that are pairwise disjoints.
  We have the equivalence:
  \begin{align}
    \Delta_{\Bgent} \ConditionalActive \Delta_{\Cgent}
    \not= \emptyset
    &\iff 
      \Delta_{\Bgent} \Moral_{\MoralSubset}
      \npTransitiveReflexiveClosure{\Delta_{\Complementary\AgentSubsetW}\Moral_{\MoralSubset}}
      \Delta_{\Cgent}
      \not= \emptyset
      \eqfinp
  \end{align}
\end{proposition}

\begin{proof}\quad
  
  \noindent ($\implies$) Let $\bgent\in \Bgent$ and  $\cgent\in \Cgent$ be two given vertices such that
  $\bgent \ConditionalActive\cgent$. 
  Then, using Equation~\eqref{eq:awtopathMoral},
  there exists $\path \in \DeploymentInPaths{\na{\np{\bgent,\cgent}}}{\moralgraph}$ such that
  $\Internals\np{\path}
  \subset
  \TransitiveClosureParentalPrecedence\bp{\na{\bgent,\cgent}\cup\AgentSubsetW}$.
  As $\bgent\in \Bgent$ and  $\cgent\in \Cgent$,
  we get that $\Internals\np{\path} \subset \MoralSubset \cap \Complementary\AgentSubsetW$ and, thus,
  that $\path$ is a path in the subgraph $\np{\MoralSubset,
    \Moral_{\MoralSubset}}$ with intermediate vertices
  in~$\Complementary\AgentSubsetW$. Hence, we obtain that
  $\path \in \DeploymentInPaths{\na{\np{\bgent,\cgent}}}{\MoralSubset, \Moral_{\MoralSubset}}$ and
  $\Internals\np{\path} \subset \Complementary\AgentSubsetW$. Then, we easily
  get that 
  $\bgent \Moral_{\MoralSubset}
  \npTransitiveReflexiveClosure{\Delta_{\Complementary\AgentSubsetW}\Moral_{\MoralSubset}}\cgent$.
  
  \noindent ($\impliedby$)  Let $\bgent\in \Cgent$ and  $\cgent\in \Cgent$ be two vertices such that
  $\bgent \Moral_{\MoralSubset}\npTransitiveReflexiveClosure{\Delta_{\Complementary\AgentSubsetW}\Moral_{\MoralSubset}}\cgent$. 
  Then, 
  there exists $\path \in \DeploymentInPaths{\na{\np{\bgent,\cgent}}}{\MoralSubset, \Moral_{\MoralSubset}}$ with
  $\Internals\np{\path} \subset \Complementary\AgentSubsetW$.
  Thus we get, by Proposition~\ref{pr:lauritzen-2}, that
  \(
  \UndirectedPath \in
  \DeploymentInUPaths{\Bgent\times\Cgent}{\graph} \cap \ActiveUndirectedPaths{\graph}
  \) and, as proved in Theorem~\ref{th:ConditionalDirectionalSeparation_IFF_relation}, it gives that
  $\bgent \ConditionalActive\cgent$.
\end{proof}

\begin{definition}
      Let \( \npOrientedGraph \) be a graph, 
 and let $\AgentSubsetW$, $\Bgent$, $\Cgent \subset \AGENT$ be three subsets of
 vertices, that are pairwise disjoints.
    The two subsets $\Bgent$, $\Cgent \subset \AGENT$ of vertices are said to be
  \emph{conditionally morally active} (w.r.t the subset~$\AgentSubsetW$) if there exists an \emph{(edge) path} in
  the moral graph $(\MoralSubset, \Moral_{\MoralSubset})$ with endpoints in
  $\Bgent$ and  $\Cgent$ and such that all the intermediate vertices are in
  $\Complementary{\AgentSubsetW}$.

  The two subsets \( \Bgent, \Cgent \subset \AGENT \) of vertices %$\Bgent$ and $\Cgent$ 
are said to be \emph{conditionally morally blocked} (w.r.t the subset~$\AgentSubsetW$), denoted by
  $\Bgent \protect\ConditionalMoralSeparation \Cgent\mid \AgentSubsetW$ when they are not
  \emph{conditionally morally active} (w.r.t the subset~$\AgentSubsetW$), that is,
  \begin{equation}
    {\Bgent \protect\ConditionalMoralSeparation \Cgent\mid \AgentSubsetW}
    \iff
    \bset{
      \path \in
      \DeploymentInPaths{\Bgent\times\Cgent}{{\MoralSubset,\Moral_{\MoralSubset}}}
    }{ \Internals\np{\path}\subset \Complementary\AgentSubsetW} = \emptyset
    \eqfinp
    \label{eq:moraliy-conditionaly-separated}
  \end{equation}
\end{definition}

\begin{theorem}
     Let \( \npOrientedGraph \) be a graph, 
 and let $\AgentSubsetW$, $\Bgent$, $\Cgent \subset \AGENT$ be three subsets of
 vertices, that are pairwise disjoints.
  We have the equivalence: 
  \begin{equation}
    \Bgent \protect\ConditionalMoralSeparation \Cgent\mid \AgentSubsetW
    \iff
    \Bgent \ConditionalDirectionalSeparation \Cgent \mid \AgentSubsetW
    \qquad \bp{ \forall \Bgent, \Cgent \subset \AGENT}
    \eqfinp
  \end{equation}
\end{theorem}

\begin{proof} We have
  \begin{align*}
    \neg\bp{\Bgent \protect\ConditionalMoralSeparation \Cgent\mid \AgentSubsetW}
    &\iff
      \exists 
      \path \in
      \DeploymentInPaths{\Bgent\times\Cgent}{\bp{\MoralSubset,\Moral_{\MoralSubset}}}
      \text{ such that }
      \Internals\np{\path}\subset \Complementary\AgentSubsetW
      \tag{by~\eqref{eq:moraliy-conditionaly-separated}}
    \\
    &\implies
      \exists \UndirectedPath \in
      \DeploymentInUPaths{\Bgent\times\Cgent}{\graph} \cap \ActiveUndirectedPaths{\graph}
      \tag{by~\eqref{eq:awtopath-ter}}
    \\
    &\iff
      \neg\bp{\Bgent \ConditionalDirectionalSeparation \Cgent \mid \AgentSubsetW}
      \tag{by Theorem~\ref{th:ConditionalDirectionalSeparation_IFF_relation}}
      \eqfinp
      % \\
    \intertext{and}
    \neg\bp{\Bgent \ConditionalDirectionalSeparation \Cgent \mid \AgentSubsetW}
    &\iff \Delta_{\Bgent} \ConditionalActive \Delta_{\Cgent} \not=\emptyset
    \\
    &\implies
      \exists \path \in
      \DeploymentInPaths{\na{\np{\bgent,\cgent}}}{\MoralSubset,\Moral_{\MoralSubset}}
      \text{ such that }
      \Internals\np{\path}
      \subset \complementary{\AgentSubsetW}
      \tag{by~\eqref{co:active-to-moral-deploy}}
    \\
    &\iff
      \neg\bp{\Bgent \ConditionalMoralSeparation \Cgent \mid \AgentSubsetW}
      \tag{by~\eqref{eq:moraliy-conditionaly-separated}}
      \eqfinp
  \end{align*}
  This ends the proof.
\end{proof}

% \jpc{remarque pour michel}\mdl{où veux tu en venir ?}

% Consider the relation $ \Delta_{X} \relation \npTransitiveReflexiveClosure{ \Delta_{Y} \relation} \Delta_{Z}$ where
% $\relation$ is a symmetric relation. We have, assuming that $X\subset Y$ and $Z\subset Y$ and $X\cap Z =\emptyset$
% \begin{equation}
%   \Delta_{X} \relation \npTransitiveReflexiveClosure{ \Delta_{Y} \relation} \Delta_{Z}
%   = \Delta_{X}\npTransitiveReflexiveClosure{ \relation \Delta_{Y}} \npTransitiveReflexiveClosure{ \Delta_{Y} \relation} \Delta_{Z}
% \end{equation}


% \begin{proof}
%   \begin{subequations}
%   \begin{align}
%     \Delta_{X}
%     &\npTransitiveReflexiveClosure{ \relation \Delta_{Y}}
%       \npTransitiveReflexiveClosure{ \Delta_{Y} \relation} \Delta_{Z}
%       =
%       \Delta_{X} \bp{\npTransitiveClosure{ \relation \Delta_{Y}} \cup \Delta}
%       \bp{\npTransitiveClosure{ \Delta_{Y} \relation} \cup \Delta} \Delta_{Z}
%     \\
%     &= \Delta_{X} \Delta_{Z}
%       \cup
%       \Delta_{X} \npTransitiveClosure{ \relation \Delta_{Y}}\Delta_{Z}
%       \cup
%       \Delta_{X} \npTransitiveClosure{ \Delta_{Y} \relation} \Delta_{Z}
%       \cup
%       \Delta_{X} \npTransitiveClosure{ \relation \Delta_{Y}}  \npTransitiveClosure{ \Delta_{Y} \relation} \Delta_{Z}
%     \\
%     &= \emptyset
%       \cup \Delta_{X}  \npTransitiveReflexiveClosure{ \relation \Delta_{Y}} \relation\Delta_{Y} \Delta_{Z}
%       \cup \Delta_{X}\Delta_{Y} \relation \npTransitiveReflexiveClosure{ \Delta_{Y} \relation} \Delta_{Z}
%       \cup
%       \Delta_{X} \npTransitiveClosure{ \relation \Delta_{Y}}  \npTransitiveClosure{ \Delta_{Y} \relation} \Delta_{Z}
%     \\
%     &= 
%       \Delta_{X}  \npTransitiveReflexiveClosure{ \relation \Delta_{Y}} \relation \Delta_{Z}
%       \cup \Delta_{X} \relation \npTransitiveReflexiveClosure{ \Delta_{Y} \relation} \Delta_{Z}
%       \cup
%       \Delta_{X} \npTransitiveClosure{ \relation \Delta_{Y}}  \npTransitiveClosure{ \Delta_{Y} \relation} \Delta_{Z}
%     \\
%     &=
%       \Delta_{X}  \npTransitiveReflexiveClosure{ \relation \Delta_{Y}} \relation \Delta_{Z}
%       \cup
%       \Delta_{X} \npTransitiveClosure{ \relation \Delta_{Y}}  \npTransitiveClosure{ \Delta_{Y} \relation} \Delta_{Z}
%     \\
%     &= 
%       \Delta_{X}  \npTransitiveReflexiveClosure{ \relation \Delta_{Y}} \relation \Delta_{Z}
%   \end{align}    
%   \end{subequations}
% %
% \end{proof}

\fi % end of ifpreprint 


% START
\bibliographystyle{plain}%{abbrv}
\bibliography{DeLara}
% START

\end{document}

% START
\subsubsection{Graph vocabulary}

From Lauritzen

A graph, as we use it throughout this paper, is a pair $(\VERTEX,\EDGE)$, where $\VERTEX$ is a
finite set of vertices and the set of edges $\EDGE$ is a subset of the set $\VERTEX\times\VERTEX$ of
ordered pairs of \rouge{distinct} vertices.

ils interdisent les loops (In graph theory, a loop (also called a self-loop or a buckle) is an edge that connects a vertex to itself. A simple graph contains no loops).

simple graph veut dire no multiple edges between two ordered distinct vertices
and no loops (v->v)

Edges with both $(\cgent,\bgent)\in \EDGE$ and $(\bgent, \cgent)\in \EDGE$ are called undirected edges 
whereas an edge $(\cgent,\bgent)\in \EDGE$ such that $(\bgent, \cgent)\not\in \EDGE$ is called a directed edge.

un graph est undirected si il n'a que des undirected edges
Il est dit directed si il n'a que des directed edges 

On autorise nous un mix des deux ce qui a un nom ?

On peut considerer le undirected grah associé à un graphe (pour nous celui
obtenut en symétrisant E)

Wikipedia: 

A directed graph is called an oriented graph if none of its pairs of vertices is linked by two symmetric edges. 
Among directed graphs, the oriented graphs are the ones that have no 2-cycles (that is at most one of (x, y) and (y, x) may be arrows of the graph).[1]

La definition de Wikipedia est étrange car un directed graph peut aussi etre
undirected !  Ex  E={(a,b),(b,a)};

Lauritzen (2002) Chain graph models and their causal interpretations

Chain graphs, which admit both directed and undirected edges, but no partially directedcycles,  were  introduced  as  a  natural  generalization  of  both  undirected  graphs  and  acyclicdirected  graphs
Lauritzen  and  Wermuth,  1989)  .One  of  the  original  motivations  forintroducing chain graphs was that the inclusion of undirected edges allowed the modelling

A partially directed cycle: est un cycle dans un graphe qui contient des directed et des undorected edges avec au moins 1 directed edges.

A chain graphis a graph in which there are no partially directed cycles .A chain
graph in which there are no undirected edges is a DAG.The chain components T of
a chain graph are the connected components of the undirectedgraph obtained by
removing all directed edges from the chain graph .In a DAG, all chain components
are singletons.For $A\subset V$ ,$G_A$ denotes the subgraph which has $A$ as
vertex set and all edges inheritedfrom G .Such a subgraph is said to
beinduced by A. A minimal complex in a chain graph is an induced subgraph of the
. Minimal complexes play a fundamental role for the chain graph
Markov property, to be further described below

